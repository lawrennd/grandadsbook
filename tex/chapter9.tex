
There were about twenty five of us in the reinforcement detail of
Royal Engineers led by \lieutenant Dewacre. The only ones I knew in the group
were the four from my tent. They were \corporal Keogh, \sappers Coleman
and Tandy and one who had been Shakespeare. He was called Shakespeare
because he was always reciting bits form the Bard's playes and
claiming to have been a Shakespearian actor in civy life. A bum actor
I reckon and an awful bore in company.

After boat frill we sat chatting and smoking in the dimly lit cabin
and listening to the various announcements coming over the tannoy
system, normaly directed to the Canadian infantry battalian which we
had seen going aboard while waiting at the dockside at Newhaven. The
battalion and its equipment appeared to fill the Thanet and we
wondered why we even mixed with them.

Stong vibrations throught the ships structure indicated that we were
on ur waty to France and, during quierter periods when wveryone
apperated to be snatichin some sleep my imagination began to run riot
with such thoughts as could we be torpedoed, was it possible to hit an
undetected stray mine and aove all what are the Normandy Beaches going
to be like? News about the invasion had benn rather scanty for us
since no newspapers reached our camp at Aldershot and since D-Day we
had been confined to camp. All we knew was that our troops had landed
in Normandy, some advances inland had been made and fierce fighting
was now taking place to hold the postions they had gained.

When I no longer felt the ships engine vibrations I assumed we must
have crossed the channel. This was soon confrimed when we Royal
Engineers were ordered on deck.

On reaching deck I was met with such a scene that was entirely
different from anything my fanciful mind had imagined during the
crossing. The early morning sun was burning away the sea mist. The
`Isle of Thanet' was anchored about two hundred yards away from the
coast, one craft in an armarda of shipping of all kinds. Above this
armarda floated a cloud of barrage ballons to deter low flying enemy
aircraft from making attacks. Looking seawards it was possible to see
the occasional flash of gun fire as the Royal Navies big guns pounded
at some inland strong point. The water between the larger ships and
shore was alive with smaller craft ferrying me, stores and wquipment
ashore. Most of the activity was away on our right in the area I later
learned was Courselles.

On shore the was none of the exploding shells and morta bombs that my
imagination had built up during the crossing, but there was howver
plenty of evidence of what had taken place six days ago. Wrecked
armoured vehicles and landing craft were scatted about the shore line
and on the beach and squads of men with bulldozers were working hard
to clear them away. Anti-invasion obstackes could be seen stickin out
of the water and through cleared ways amphibious D.U.W.K.S. were
coming ashore with stores.

A motor boat hooked into a ladder built on the side of the Thanet and
we clambered down inot the boat. Not an easy task with our gear. While
heading to some improvised jettied built from empty oil drums with
boards lashed on top, a piec of loose rafting cordage wrapped itself
round the boats propellor and stalled the engine. Without power the
helmsman was unable to steer alongside the jetty and the boat drifted
to a halt a few yards form the beach. After paddling ashore through
ten inches of water we were met by a beach master who had no knowledge
about the unit we had to report to but insisted that we `kept moving
and got off the beach.'

To get to the coast road we had to go through a velt of sand dunes
which Jerry had mined. Gaps had been cleared through the minefield
which were marked by tapes and \lieutenant Dewacre led us to one of these gaps
where the tapes had become a bit loose and were blowing about in the
breeze. In single file and keeping to a path that was central to the
pegs of the marking tapes we reached the road which was fill of filled
in shell holes and flanked one side with battered buildings.

Sapppwrs were working to make the road more useale and to them
\lieutenant Dewacre hopefully went ot get information. From the officer in
charge he learned that we were in the 3rd Canadian sector. Now we
realised why we were on a shop bringing in Canadian Troops. We should
have been in the 3rd British sector the other side of Courselles. We
had been put on the wrong ship. It was suggested that we travelled
across country using sens of direction to get into the 3rd British
sector and find our unit.

We crossed into what had been a cornfield which sloped gently away
from the road. The corn was now all smashed to the ground by vehicles
and trmpling feet. There were some ctaters in the ground and several
rifles with helmets hanging on their butts marked the graves of allied
and German soldiers killed during the landing. After battl debris lay
around. Evidence of the resistance put up by the Germans to hald the
inbasion. Except fot thos sandwiches we ate at Newhaven while wating
to embark we had not eaten since leavig Aldershot some thirty six
hours ago and it was decided to find out what the emergency rations
had to offer. Each pack designed to provide sustenance for twenty-four
hours contained a tablet about $3" {\rm x} 1 \frac{1}{2}" {\rm x} 1
\frac{1}{2}"$ of dehydrated meat, a similar tablet of oatmeal, vubes
about the size of Oxo cubes: amixture of tea sweetner and powdered
milk. Hard biscuits, a bar of vitamin chocolate, a collapsable tin
stve with methaldehyde fuel tablets and a few sheets of toilet
paper. For this snack I put half my bottle of water into my mess tin
and placed it on the lighted stove to boil. With the hot water I mad
tea and soaked the oatmeal block. The tea was quite a good drink abut
the oatmeal mush hadn't a lot of taste. I tried to eat some biscuit
that had been broken into mout-size pieces by hammering them with my
jack knife. After putting a piece in my mout it took ages to suck and
soften enough for swallowing. Most of the broken pieces were put into
a pocket to be sucked at leisure. The vitamin chocolate, although
bitter and strong tasting wasn't too bad to eat. The concentration of
the food was sufficient to take away the emptiness of my stomach and
gave me renewed vigour. What a picnic site this was. From here I could
look down onto the beache where we had landed in what appeared to be a
small cove. The `Isle of Thanet' had vome inshore and at low tide was
now aground. She would refloat on the next high tide. The Canadian
Troops were disembarking, clambering down the steps in the side of the
vessel, wading ashoure through the shallow water, forming up on the
beach and marching away towards Courselles. How their equipment was to
be unloaded puzzled me. Behind I could hear the rattle of small arms
fire and machine guns together with the louder explosions of
artillery. Round about the field there were those grim reminders o
those who had lost their lives in this conflict.

Our meal over we moved across the field in what the \lieutenant thought was
the right direction. Thie inhabitants of ta small village we passed
through, mostly women, children and elderly men, eyed us rather
warily. The houses seemed to be in good order but the church tower had
lost its upper part. The people's unfriendliness at this time was
understandable. They were not sure the allies had come to stay and
were afraid to show too welcoming an attitude un case the Germans
pushed us back into the Channel and reprisals would begin.

Hindered with so much gear our progress was slow. \Lieutenant Dewacre seemed
to have twice as much as we had; besides his `marching order' he had a
folding camp bed and collapsable wash stand. It was unnaturally free
of any Milatary presence since we were so near to the front line. We
met no one after leaving that one village.

When we came to a badly battered collection of garm buildings in the
late afternoon \lieutenant Dewacre called a halt for the day. The estimated
depth of the beach head around here was about five miles and not being
sure avout how far we wer inland he decided that we should stay where
we were for the night and in the morning he waould go alone to search
around for infromation.

The farm well water was tasted and appeared to be uncontaminated so we
had a good was and shave using odd utensils we found lying around,
after cheching for boobie traps. Water wsa heated and feom the remains
of the Emergency pack another meal was made. The soaked dehydrated
meat cube produced a mess of tastell mince. I ate the remains of my
chocolate and washed it down with tea. Bits of biscuit were in my
mouth for most of the evening.

We were hoping to bed down for the night on hay in a corner of the
barn bu the lieutenant though that the farm was too obvious a target for
enemy patrols and artillery so he took us some distanve away from the
farm and we spemdt the night in a dry ditch. A watching rota was drawn
up which gave each pair of men about half an hour of alerness. The
watch wsa harldly necessary since I doublt if anyone slept that
night. I know I didn't. I lay there listening to the noises of battle
which, in the darkness sounded uncomfortably close. Occaisionally the
shy was lit up with flares and coloured fairly lights and overhead I
head the drone of planes which I hopefully assumed were our bombers
eith going on a mission or returning from one.

We had an early breakfast from our second emergency packs and
\lieutenant Dewacre left us in the charge of \corporal Keogh while he sat out
alone to try and contact a military unit. In a couple of hours he was
seen coming back along th road with a box on his shoulder. Lady Luck
had been on his side for he had made contact with a R.A.S.C. ration
company, who not only knew where our unit was but had offered
transport to takes us ther and had given him a box of rations for us
to eat while waiting for the truck.

The wooden box was a fourteen man twenty four hours ration pack which
was quickly broached. We retraced our stwps to the derelect farm, soon
had a fire going and had a meal from the variety of tinned good in the
box. These `Camper Boxes' contained quite a selection of tinned food,
such as bacon, M\&V, corned beef, puddings, fruit and sausages. There
was vitamin chocolate, biscuits (hard) and cigarettes. All boxes had
the same ammount in them but were packed together to give a variety of
diet. What we couldn't eat we shared out to be stowed in our packs for
another time.

The R.A.S.C. truch arrived and we were takedn to a field on the
outskirts of Douvres in the 3rd British sector. All military movement
appeared to be alon this road between Bayeux and Douvres and explained
the quietness of our journey through the fields and lanes between here
and the coast. H.Q. of the reinforcement unit, a little way inside the
field was in a hole about eight feet wquare and three feet
deep. covered by a flimsy roof of old corregated iron, torn ground
sheets parts of gas capes and camouflage netting, supported by a
rickety structure of timber.

The field was divided into two parts. In the large part were infantry
reinforcements, while the wmaller part to which we had been directed
was for corps reinforcements. R.E.s R.A.s etc. We were told to get
into a trence. There were some empty ones in our area, or
alternatively dig one for ourselves. We were not to bunch togeterh or
stand in groups about the field, and always to be ready for
action. Not liking the look of the trenches already dig. I decided to
dig my own shelter. The soil was quite easy to dig out and it didn't
take me long to excavate a hol tree feet by two feet and som two feet
six inches deep carefully packing the loose eath on three sides of the
hole. In the corner of one of the three feet sides I tunneled until it
was long enough to take my legs and trunk so that I could lie flat out
with my ches and head in the hole. Leaving a platform for my chest and
head I lowered the rest of the floor another twelve inches. I now had
somewhere to put my feet while sitting on the platform and also a sim
to collect water seepage. From odd pieces of wood turf and soil I
built a lean to roof ober the hole to keep out the rain, a precaution
I was truly grateful for later on. A few shelves cut into the sides of
the hole took care of my kit and I reckon I was as comfortable as any
on the field. Its major fault was the awful claustrophobic feeling
when lying down to sleep. It was a bit like lying in ones grave. All
kinds of ingenious was were used by some men to get a bit of cover
over their open trenches, but there were many who seemed to thing that
the fine weather we seemed to be now enjoying would last for ever, or
were perhaps hoping they would soon be moved to more favourable
conditions. For us a quick move from here didn't materialsise. It was
however a different story for the infantry boys. Infantry casualties
were gettin heavier. At one time during the battle for Caen and
Caumont, four hundred replacements a day were required to keep
battalions up to fighting strength so infantry reinforcements hardly
had time to unsling their equipment before being posted.

There were no parades here. \Corporal Keogh checked the number of our
party each morning, reported to \lieutenant Dewacre who then handed in his
report to H.Q. The clanging of an empty shell case hit with a piece of
iron summoned us to meals which were brought to the field by a
P.U. truch and seved outside H.Q. dug out. N.C.O's controlled the
ration quies to make sure it didn't get too big and attract the
attention of wandering enemy aircraft. We were attacked one day. I had
collected my ration of biscuit, M\&V stew and tea and was taking it
back to eat in my shelter when a cry of `Enemy aircraft!' and the
chatter of machine guns sent me flat on the ground spilling my rations
in the act. Those queing at the ration truck scattered and some, in
their hurry collided with the supports of H.Q.s roof qhich collapsed
on H.Q. personnel. A few men even fell into the dugout. No one was
injured in the raid, the raider only made one straffing run but many
meals like mine lay scattered on the ground. Dry biscuits and a few
sips of water had to do for my midday meal that day.

Drinking wate was scarce of any water come to that. A water carrier
came over once a day and issued on pint of water to each man. Not much
to wash and shave in or provide a wet for a dry, dusty throat. To trya
nd save water for drinking, washing water was saved and used over and
over again. Empty tins in which to save this water became a well
sought for commodity and were jealously garded. I managed to get two
biscuit tins about ten by six inches and two inches depe for my
washing water.

Without parades of fatigues to do the days here were quite boring
which may sound strange when we were only four or five miles from the
front line. To while away the time I moved from trench to trench
chatting with the sppers I knew or wathced the card schools at play in
someone's trench. Sometimes I went over and chatted to the crews of a
squadron of Sherman flail tanks laagered under the trees in our end of
the field. I was intereste to know how the flails had worked in the
minefields. A few of the flails looked worse for wear. The crews
stories about their action of the early days were full of interest.

All day long the noise of war came from the east and south of
Douvres. The German Panzer divisions were putting up strong resistance
to our advance from Ouistreham and Ranville. Caen, the gateway to the
Seine which Montgomery had hoped to take quickly was being fiecely
defended by the Germans. We could feel the ground tremble when our
bombers dropped their heacy loads of high explosive bombs in and
around Caen.

On afternoon I was sitting in my shelter cleaneing the dust from my
rifle when I heard a whistling noise followed by explosions. Putting
on my helmet I poked my head up to see what was happening but quickly
bobed it down again when I realised the field was being shelled. When
I thought the shelling had ceased I grabbed my rifle and ammo poiches
and climbed out ready to join in the defense of our patch.

The sherman tanks were already on the move and clanking out of the
field. Officers at H.Q. assured us that we were in no immediate
danger. The small Germand kight armoured reconnasisance patrol that
had been shelling the field were now being pursued by our amoured
behicles. Casualties from the raid were two men killed by a direct hit
on a trench and some had been injured by shrapnel. These enemy
reconnaisance patrols often caused us to be on alert byt they were
usually wiped out before doing much damage. One larger and heavier
armed patrol almost succeeded in cutting the bridge head in two a
Bayeax before it was broken up and destroyed.

When rapid advances were mase, small enemy stron points, not
considered to be of strategic importance were often left isolated for
back-up troops to eliminate them. Douvres had been an important radar
station for the Germands and was surroinded by strong points. One of
them lay between our unit and the sea and was still occupied by a
German garrison. Surrounded as it was by our troops it was being left
for them to surrender but the garrison commander was in no hurry to do
this. Perhaps he was hoping that we would be pushed back and he would
be able to rejoin his German friends again.

One of our aircraft, flying back from an inland mission, flew over the
area. His plane had been hit and was trailing black smoke, it was
loosing height all the time. The pilot thinging he was now in an area
where he could land in safety abandoned the aircraft and in his
parachute began to drift slowly to earth. A fwe shots from an
automatic weapon rang out. Some one in the bunker had fired t the
helpless pilot dangling in his parachute but luckily failed to hit
him. Everyone in the unit was furious and wanted to mount an attack on
the Germans. The squadron of Shermans did get permission to join other
tanks and shell the bnker into submission. It must have been hell
inside that bunker as shell after shell crashed into the concrete
walls. The tank crews told us that when the Jerries finally came out
their faces were whiter than the flag of surrender. I bet they had a
rought journey to the P.O.W. compound.

The dust from blasted buildings mixed with the acrid smell of
exploding shells ans bombs and the stench od death gave the air a
peculiar smell. Dead faem animals lay around unburied and two dead
cows lay fly ridden and rotting in the field behind us. Until the
animals had been registered by the authorities for compensation claims
we were not allowed to bury them. We could only pray for the wind to
blow their stink away from us.

In the middle o June the dry spell of weather came to an end and we
had periods of heavy storms. Gales in the Channel fot this time of
year were the worst for ober forty years and almost wrecked the
building of the Mulberry harbours. The harbour being built at
Aeeamanches, completed in a great struggle against the elements
enabled bital supplies of men and material to be landed in the beach
head.

Instead of the clouds of dust raised every time the wind blew our
field became a horrible slushy mess of mud. Trenches became water
logged and even those with improbised roofs rwquireedconstant bailing
out. The sump I had dug out was invaluable for this and my bed
platform remained dry or rather dryish for now everything became ver
damp. Shakespeare was one of the lazy ones who had not thought out an
idea to gice him protection from the rain. He now sat in his trench,
on a compo box, hiddled under his gas cape and ground sheet, wet
through and shivering with cold. Each day his moral appeared to sink
lowere and lower. If yo went to try and cheer him up he began crying
and moaning, `I can't go on', I can't go on'. When we moved to another
camp we managed to haul him from the trench and get his gear on. After
staggering a few yards he slipped in the mud, fell into a trench and
lay there with tears in his eyes saying `I can't go on'. He wouldn't
move out of the trench and he was left in the care of H.Q.. What
happened to him we never knew. He was a pathetic sight and among us
there was more sympathy for his condition thatn their wsa for his lack
of courage. After all we each had some fear within us if we were
honest with ourselves.

A lot of troop reorganising was taking place and \lieutenant Dewacre with our
detail plus anothe detail of sappers and \lieutenant Kingsley were moved to
another camp nearer Bayeux.

In drizzling rain we trudged across country to this new camp situated
on a fairly large farm. As we went throught the farmyard entrance we
were counted into groups of fourteen, given a compo box and directed
to tents erected round the perimeter of the farmesr field. In the
count I became seperated from Keogh, Coleman and Tandy and was with a
group of thirteen less familiar faces. The tents were and improvement
on the trenches at Douvres and certainly less claustrophobic than my
shelter there. Before claiming bed space the tent floor had to be
lowered by eighteen inches to give `blast protection' from bombs or
shells landing close by. The drop also gave more head room to the
tent. During one raid a dud anti-aircraft shell dropped and exploded
outside our tent. We were not harmed but the tent was made useless by
the shrapnel holes in the canbas. Until we had cleared out a barn and
made it a cook-house and mess room we heated up the rations of the
compo boxes in vessels of hot water on communal fires. The most messy
item in thes boxes was tinned bacon. The bacon was in a long strip
rolled up in a kind of paper between strips, to feet neatly in the
tina dntopped up with a very greasy watery liquid. Vefore placing the
can in the hot water two expansion holes were punctured in one
end. When the tn was hot the loquid wsa drained off but ther was
always sufficient of it left over to make a greasy mess for shering
round. Many cans of rations exploded in the boiling water when
expansion holes were forgotten.

For a few days I had a dour Scotsman from the Hebredies lying next to
me. He was a early riser always on the move before reveillie and his
rifle and equipment were kept immaculately clean. He was an odd
character. Although he didn't smoke himself he aways had his cigarette
ration which he would not sell or give to those who cheekily
approached him, but if he knew anyone in the group who was genuinely
out of cigarettes he would gice them some. I never heard him swear or
moan about any fatigue he was given, he just silently plodded
on. Along with many other sappers he was posted to comanies building
the Bayeax bye-pass; built to take the heavy volume of military
traffic away from the town centre and its narrow streets.

In one corner of our field there was a capt. Campbell R.E. who played
his bagpipes outside his tent every evening. Voices from the sappers
tents pitched round the field perimeter shouted barious suggestions as
to what he could do with his bag pipes. A weapons pit had been built
by his tent and a Bren gun mounted on a tripod was manned there from
dawn to dusk. I was not unusual to see a German plane being chaes by
our aircraft over the area. On one of these runs Capt. Cambell almost
went berserk because we hadn;t fired at a German plane chased by a
Spitfire when within our rife range. He calmed down when it was
pointed out that while firing at the Jerry plane we could have hit our
own aircraft.

While having a natter and a smoke one evening, a lone raider on a
suicidal mission, for the area was saturated with Bofors gus, flew
over with machine guns stuttering. Our group split up and died for the
ditch. Keogh had the misfortune to dive into a piece of ditch that had
been used as a makeshift latrine before proper latrines had been dug
out and it was only lightly covered with soil. Keogh didn't smell bery
pleasant when he climbed out of the ditch. Water was still on ration
adn he was unable to wash the smell from his clothes. He wasn't
pleasant company for many days and claimed he would have been better
off to have been hit by a Jerry bullet and in need of hospital
treatment. Being ostracised by his friends didn't please him.

From Arramanches and Port en Besson on the coast to the Bayeax area
inland there was now a large concentration of troops and
equipment. What a target it would have been for the Luftwaffer if our
air forces had not held our supremacy.

Parades were held each morning and again afer dinner for rll call,
postings and fatigue details. There was no shortage of fatigues here
and on some of my details I joined parties emptying the night bucket,
these were buckets placed at night outside each tent fot the use of
weak bladder-bods. Improvementd for camp comfort cleanliness and
defence always provided us with jobs.

After a fwe days on that kind of fatiges I wasa sent to a small
chatezu about a mile and a half from camp. The chateau was being made
into a mine school where new types of mines could be studied and
refresher courses given on those normally met. I as given quite a bit
of carpentry work to do here which I found agreable. Most of my
journey was across fields but a short part of the qay I had to walk
along th Caen-Bayeaux road, always full of trafic, and twice while
walking on this piece of rad, General Montgomery's convoy, led by two
M.P.s on motor bikes calling for the traffic to give way, passed
me. He had been to the Caen Caumont front and was returning to his
H.W. at Port-en-Beson.

This detail to the mine school was a lucky break for me. Some of my
mates back at camp had been sent on two awful fatigues. A tent
hospital in fields nearby had fatigue parties from our camp and one 
of their tasks was to attend to the incinerator where dirty dressings and
amputated limbs were burnt. They told me the stench of burning flesh
here was nauseating. Another party had been sent to dig up some
isolated bodies and take them for reinternment in a temporary
cemetary.

At the mine schol I saw dogs trained to sniff out buried
anti-personnel mines that were constructed from wood and plastic and
were difficult to locate using the conventional mine detector. The dogs
were mostly Airdales and their ability to sniff and find hidden
explosives was miraculous.

I had seen several aerial combats between our planes and the Luftwaffe
since arriving in Normandy but until now I had not witnessed the end
of a fight. As I was making my way back to the mine school afer having
my midday meal at the camp a Messerschmitt with a Spitfire on its tail
flew low over the fields and bursts of fire from the Spitfire's
machine guns were failing to score a hit. The planes momentarily
vanished from my view behind a clump of trees and when they reappeared
they were flying towards me. The Messerschmitt by some clever move was
now on the Spitfires tail with his machine guns chattering away . A
burst of fire from Jerry scored a hit and smoke began to trail from
the Spitfire. The pilot climbed until he gain a bit more height and
then bailed out. The Messerschmitt now alone in the sky was receiving
attention from the many ack-ack weapons that were in the area and
before I lost sight of the plane it also began to leave a train of
smoke.

While on picket duty one night I was patrolling the perimeter of the
Transport Park and in the darkness I saw what I thought was the dim
light of a truck shingin through the hedge. The slitted hoods fitted
over the trucks head lamps showed very little light through
them. Thinking that some unauthorised person was meddling with a truck
I slipped a live round into the breach of my rifle and called a
challenge. Nothing moved and the dim light still showed so I
cautiously approached to investigate. The light disappeared and
looking around I couldn't find anyone about or see any interference
with the vehicles. My mate hurrying from the opposite direction wanted
to know why I had shouted. After telling him avout the light we made
another search of the park but found nothing amiss. The incident was
reported to the picket sergeant whe I came off duty who laughed about
the light. He opened a tin and showed me several large glow worms. He
said in the darkess he could read the time from his watch with these
glow worms. Thwy were quite large creatures and afterwards I saw many
more glowing in the night.

I met Andy Jamieson again in this camp and we spent many hours in the
evening at the side of the Bayeax Caen rod hoping to see some one from
278 Field Company. We knew that the 15th Scottish Division was in the
area and was much involved with the battles at Tilly Caumont and Caen
to break out of the bridge head. We saw a lot of divisional transport
collecting supplies from Port-en-Beson and Arramanches but never saw
anyone we knew. I met C.S.M. DUncan from 278 COmpany one day. He was
just leacing the unit as I came in from the mine school for my midday
meal. We were both surprised with the meeting and I noticed the rather
drawn expression on his face. He asked me what I was doing here and
when I told him that both Andy Jamieson and I were here as
reinforcements awaiting posting he said he wished he had known he
would have asked for us on his replacemetns list. During the crossing
of the river Odon thecompany had lost one officer and ten O.R.'s
killed and eighteen O.R.s had been injured. Amoung thos killed was Ted
Ketnor one of my NAAFI tea drinking chums and \corporal Wilkinson of 2
platoon. The officer I didn't know. He had replaced \lieutenant Shaw of 1
platoon before the invasion. It was a chilling reminder that had I not
been sent on that course at Chatham I could have been bridging across
the Odon. S.M. Duncan didn't have another chance to get me back to 278
company for I was moved to another unit on the Arramanches side of
Bayeux.

This was a small unit of R.E.s. There were about fifty of us here
mostly about my age group. Being a small unit there were few fatigues
and for most of the day we moved about the camp and faces changed
daily as men were posted and others moved in.

In the small marquis serving as a mess tent I met a sapper I
recognised as a man I knwe in Mjor Bells company at Darlington. He was
then acting segeant major of the company. When I approached him and
said I had known him at Darlington and wondered why he was now a
sapper, he excused himself on the grounds of having to go to the
cook-house and hurried away. From then on he avoided me as if I had
the plague. He and a very tall lance corporal seemed to be on
permanent cook-house and mess tent duty here.

During my short period with tis unit I was detailed to build a black
out cover over the ovens of a R.A.S.C. mobile bakery company operating
nearby. Bread was not part of our general ration issue; we were still
on biscuits and would be for many weeks. This mobile bakery was making
bread for hospital use only and with tehe increasing demand the
company wanted to work at night. The Luftwaffe were still making some
raids over the beach head so black out restrictions were still being
enforced and the glow from the diesel fueled ovens had to be
shielded. From a pile of rough sawn wet elm and corrugated iron sheets
I managed to build a satisfactory rough shelter round the ovens with
help from R.A.S.C. buddies to lift and hold timbers in place. The
company supplied me with my midday meal and what meals they
were. Company cook had been a chef and with the aid of extra rations
he was able to get, such as flour and yeast he worked marvels with
tinned goods in the compo boxes. Compo boxes had codes on them to deno
their contents and from these codes the R.A.S.C. ration companies were
able to choose the best boxes and no doubt some of the better boxes
found their way here in exchange for a bit of bread. My tent mates
were most envious about my meals when I tantalizingly described them
in the evening.

It was now about eight weeks since I landed in Normandy and not once
had I been able to get a decent wash all over. I felt crummy and
dirty. My underwear was a disgrace. Dirty laundry had been exchanged
for clean on a few occasions but in these conditoins it was dirty
before you wore it. Whoever thought of the idea to issue us with wite
towels and underwear must have been crazy. Mine, no longer white, had
a perpetual brown appearance.

Since may I had been in six different holding and reinforcement units
and I wa beigining to wonder if I was an army misfit and with the
changing of faces round me I never had the chance to get really pally
with anyone. How my letters found me while moving about all the time
amazed me. The army postal services did a magnificent job over here
and delivered mail quite promptly. Letters hom also found their way
home to the U.K without too much delay. I had worked out a little code
on my letter to Nona and she always had a little idea of the area I
was in. It was a dangerous thing to do for had I been rumbled I could
have received a stiff prison sentence.

At last I was posted to a company where I now hoped to settle down
with regular companions. The Artisan works company to which I was
posted, its number I cannot remember, was based on the St. Lo side of
Bayeax and I became a member of its No.~2 platoon. Odd how No.~2
platoon and No.~2 Section was so often part of my Army address. No.~2
platoon's commander, \lieutenant Bailey had been left behind in England when
the company embarked for France having broken his leg a day before
embarkation adnd the platoon was being run by Sgt. Wilcox when I
joined it. \Lsergeant Webster, a Brummie was my sectionleader who was aided
by Vorporal Dawson a Geordie and two lance corporals. \Lcorporal Bates was a
southerner and \Lcorporal Munroe was a lowland Scotsman.

The platoon and personnel of H.Q. company were in tents erected in a
cider apple orchard and the H.Q. office and Officers' mess were in a
cottage on the opposite side of the road. It was at this cottage that
I had my first job with Fred Harper, another carpentere, we made and
fixed black out shutters to the windows and attended to the
doors. Hornets had nested in the masonary above the front door and it
was amazing how quickly everyone passed through that doorway. Hoping
not to attract the attention of the hornets; they were large and
vicious looking insects.

In our tent there were three Freds. Myself, Fred Harper, and Fred
Newman so nicknames were in common use `Wuff', a nickname given to me
by one of the teachers at Leamington Tech and used when I worked at
the Alvis became my title here. `Wuff' had no doggie connection, Mr
Clift, our science master gave it to me because he said `Whenever he
travelled by train and the train halted at Kenilworth station all he
could make from the porters' call of the station as they moved along
the carriages slamming doors shut was Kerwuff, Kerwuff. with emphasis
on the `Wuff'. and that was how I became know as `Wuff'. Fred Newman
our section truck driver, being a tall fellow, was called `Lanky'. In
the next field we had a platoon of Canadian Engineers for neighbours
and what a chummy crowd they were. Their rations, like the American
rations, were much better thatn ours and for the few nights we were
together they brought their beer, cigarettes and other goodies to
share with us in our camp and to have a good natter. I was sorry when
they moved on.

Our company, one of many working on lines of communication, became
involved in keeping a petrol supply route clear to the forward units,
now preparing to break for the River Seine. This entailed a lot of
travelling avoout and working in small groups. For a short period our
platoon came close to the front lines, close enough to get within the
range of Jerry's artillery, especially when working in the Tilly
area. Tilly was a shatteded town and its roads so badly mauled or
buried under debris that nwe routes were bulldozed through the
wreckage. Villages round about were now heaps of broken brick with
bits of walls poking through here and there. Looking at the
devastationone wondered if Normandy would ever be the same again. What
amazes me was the number of civilians who had survived the onslaught
and wer trying to bring a little organiszation into their disrupted
life. To see an old lady sitting milking a cow in this war torn
country side seemed nothing short of a dream.

Rubble wasn't an ideal material for filling in shell craters. The
vrick and soft stone easily crushed into dust under the wheels and
trackes of military vehickes and rwquired constant attention to keep
them filled so that the roads remained useable.

It was still possible for Jerry to lob a few shells ober into the area
where our section was now working in an attemt to disprupt supplies
and we had a narrow escape on one occasion. We were working in the
suburb of Tilly collection rabble for our road repairs when Jerry sent
a salvo over. Ther wasn't much cover available here and when a shell
exploded neaby we were covered with dust and debris. A large chunk of
masonry dropped on Wally Waldrous' leg and broke it and Chas. Timpson
dislocated his shoulder when he dropped some timbers.

It was a new experience for me to be moveing about the country on this
road repair work and wither a pick or a shovel always seemed to be in
my hand these days.

When I had my medical before call up, one of the things I was asked
was `What my hobbies were?' such as wireless, photography or first
aid. My reply was `Gardening occupied my spare time.' I now wondered
whether good at digging had gone into my records that I was engaged
with so much shovel work in Normandy.

The Royal Artillery had mobile batteries of guns that travelled about
firing salvos of shells into the German lines moveing off after each
shoot. The idea being to prevent Jerry getting a fix on them and
retailiating. From the artillery point of view that was O.K. but not
so good if they chose to shoot near where we were working, when our
fingers were crossed and we hoped Jerry had not fixed the spot for a
return firing of shells.

I had the fright of my life one day from on of our artillery postions
when we arrived at a section of road to improve its surface. Before
starting work I obtained permission from \lsergeant Webster to break away
for a nature call. Unknown to us a battery of artillery were
camouflaged behind some stindted trees and ruined buildings about two
hundred yards away. With a shovel I climbed into the field and moved
towards a wrecked bard to get provacy for my needs. As I approached my
objective their was a terrific explosion and shells whined over my
head. As I fell flat to the ground I had visions of my head joining
these shells which of course were high above me in their trajectory
towards the Germans. This was one time when I blessed the scourge of
constipaton which we were all suffereing from due to our concentrated
rations.

While workin on these road works projects, I saw the damage that our
bombers had made when a large force of Lancasters escorted by fighters
had, in daylight, bombed and destroyed a big concentration of German
armour at Villers Bocarge. I had seen this formation fly over me and
in the distance saw their circling aroud before making their runs over
the target, when I was in the holding unit between Bayezc and Vaen and
had felt the ground tremble under me as the tons of bombs exploded
amongst the armour. Jerry had cleared away some of the wreckage but
many battered hulks remained.

Wrecked German tanks were often seen among the rubble of
cottages. Thes tanks drove through the cottage walls and came to a
halt in the libing room. The remaing walls provided camouflage to
their presence. Gun flashes ecentually disclosed their position and
some accurate shooting by our artillery knocked them out.

In those early weeks of August we moved form one bivouac to another
making roads usable for the supply convoys. The rail system was so
badly damaged both by the Germans, to delay our advance and by us to
prevent reinforcements and supplies reaching German front line areas
that it would be a long time before the system could ease the traffic
load on the Normandy roads. Railway companies of the Royal Engineers
were working flat out to get the trains running again. Bombed bridges
were one of the main obstacles they had to overcome and Bailey bridge
equipment was used extensively on these repairs.

The allied advance was now moving so rapidly that we were left far
behind the fighting and our platoon returned to the camp in the cider
apple orchard, H.Q. company having been there all the time. Toad
maintenance was still our main emplyment and we made a temporary move
to a farm near Caen. In all the carnage that hd taken place one often
saw a one fof place that appeared not to have been damaged and this
farm was one of those places. Whole working in this area on work whic
occupied us for a couble of weeks we billetted on the top floor of a
barn witho only one small window too dirty for light to penetrate
reading and writing was virtually imposiible. To overcome this problem
we all had a tin of diesel oil soaked rag with a strong wick threaded
throught the lid to provide illumination. Clouds of sooty diesel smoke
filled the barn choking nostrils and throats. Belov us were the
remains of last years harvest of marigolds, many of which were
begining to rot filling the ari with a sickly stench. However these
two faults failed to make the barn less comfortable than a bivy in a
field. One evening Sgt. Wilcox, we were still without a platoon
officerm told me not to join the working partioes the following
morning but to report to him with a carpenter's kit. After roll call
and the dismissal of the parade to duties Sgt. Wilcox took me behind
the farm house to the dairy. The dairy door and frame was in a bad
state of repair and Wilcox said `Do your best with the repair, the old
boy is very co-operative about our stay here. True, for cigrettes and
tobacco, eggs, butter and milk were available.

By late afternoon I had the door working again and the farmer was
delighted. He invited Wilcox and myself into the parlour for a
drink. Three very small galsses and a bottle of light coloured liquid
came from his cupboard. When I saw the size of the glasses I thought
this isn't going to be much of a drink. The glasses were carefully
filled and with a murmured toast he tossed the drink down his throat
in one go. Fred, not to be outdone, did likewise and with disastrous
results. The liquid was like red hot metal. It took my breath away, my
eyes watered and it seemed as though steam was escaping through my
ears. The old farmer went into peals of laughter as I choked and
gasped for breath. Wilcox, more canny than I, had sipped his drink,
but he too was enjoying my discomfort. That was my introduction to the
drink of Calvados.

Petrol and oil was coming ashore at Port-en-Benson having crossed the
channel through pipes laid on the sea bed (known by the name
P.L.U.T.O.) From Port-en-Bessen it was then transproted by rad in
petrol tanker lorries to the forward areas. There were also many
reserve dumps in the bridge head area were petrol was stord in
cans. Before the introduction of the now familiar Jerry can oblong
cans holding four gallons were used to stor the petrol. These oblong
cans, built from tin plate with soldered seams were easily puctured
and the seams ofen leaked with the result that thousands of gallons of
petrol were lost from damaged cans. I was in a detail sent to one of
these dumps to dig a spillage ditch around it and improve the tracks
into the dump. Webby and Bates were N.C.O.s in charge. Now and agaoin
the odd lone German plane flew over the vast concentration of military
stores and equipment dropoing bomb here and there and sometimes coming
in low to machine gun the area. They were more of a nuisance than a
danger and I'm sure a high percentage of them never returned to
base. One of these raiders managed to start a fire at this pettrol
dump and with out special equipment fires in those dumps were
impossible to put out. There were some army fire-fighting units in
Normandy who were trainded to deal with these situations but they were
based around Port-en-Besson where there was a greater risk of
fire. Before the blaze became too hot and dangerous many loads of
petrol


AHHHH Missing 116

Drinking water was still rationed and often it was son chlorinated
that tea made from it was almost undrinkable. Less sterile water for
washing puposes wasn't too plentiful either. When we could, we
stripped off to have a wash down in any stream we happened to be
working near. All to oftern there would be a muddy patch to negotiate
so dirty feet went straight into socks to keep our cows reasonably
clean. Always at the back of my mind was the thought that some dead
farm animal might be lying in the water higher op stream. At som of
thes improptu bathing spots dust outside Bayeax the litte stream ran
under a viaduct which carried the Bayeax-Caen line of rails and while
we were having a wash down in the nude a bunch of railway company
sappers came along the railway o a bogey. They stopped to make fun of
us and quite a good humoured slanging match took pace.

To get hot water for washing we had what was known as a `Lazy-Man's
Boiler'. This contraption wnsured that hot water was abailable for
everyone while the fire was maintained. An open vessel of water over a
fire quickly went dry. Too many would tak hot water and not replace it
with cold. An empty oil drum was rapidly prepared for the boiler by
cutting a hole in the sidr of the drum for filling. At the top of one
end of the drum and on the same plane as the filling hole another hole
was made cutting in three sides only so that the flap of metal could
be bent down adn formed a rough spout. Small walls of brick or stone
weiht a fwe iron bars laid across were built to support the drum and
to raise it high enough for a fire to be built underneath the drum now
lying on its side. It was filled with water untill it ran from the
spot and while we were out on works details camp orderlioes lit the
fires to heat the water ready for our return to camp. To get hot water
you needed two containeers. An empty one was placed under the spout
and a containere of cold water was poured throught the filling
hole. The cold water immediately sank to the bottom to be heated
displacing hot water which flowed into the empty container equal in
quantity to the amount of cold water poured in. This method made sure
that as long as there was a fire buring under the drum hot water was
available for everyone.

Mobile bath units began operationg in the area nad one morning \lsergeant
Webster took the platton to one based near out camp. So great was on
demmand on these units that we only had this one unit. The shower room
was a medium sized marquee and a tarpauline covered floor and rigged
overhead with a network of perforated piping giving a jet of water for
each square yard of flooring. A diesel fueled boiler outside the
marquee heated water brought in by a relay of water truxks. The hot
water form the boiler went into a mixing tank were it was cooled to a
temperature bearable for washing and from the mixing tank it was
pumped by hand through the system of perforated piping in the marquee
for us to happily soap each others backs under the jets of
water. After almost ten weeks withouth a decent bath or shower this
was sheer luxurty. No time was allowed for soaking under the warm
water for there was quite a long queue of men waiting for there turn
under the showers. Each detail of men undressed dried off and dressed
in the open and I wondered what happened if it was a wet day,
fortunately for us the weather was fine.

To move road repair materials about on large jobs we had little dumper
truck which had a nasty habit of tipping over if one tried to turn too
sharply or went too fast into a turn and there was some competition
between us to see who could turn the fastest or the sharpest without
overturning. I was having a drice on wone of thesse dumpers whiler
attending to a road surface near a Brigade H.Q. and rolled over with a
load of gravel just as the brigadiers car was leaving. The Brigadier
wasn't very happy about being delayed while the drive entrance was
cleared and he wasn't very complementary about my driving skill
either. 

When two or three sappers and an N.C.O. were sent out on small jobs
they were oftern taken out by transport to the work site and then left
to make theri owen way back to camp. Sometiomes thes jobs only took a
few hours to complete. \Lcorporal Bates, myself and two other sappers were
returning from one of thesesmall taksks and, having reached a road
used by American transporters ferrying stores from the Mulberry
Harbour to the Carentan Peninsular, we began to look for a lift. The
`Darkie' drivers of these vehicles drove fast and furious but were
ever ready to give one a lift, they were really `great guys'. A `Mack'
truck driven fast as usual came round a bend in the road about a half
a mile from us and four thumbs went into action. As the truck
approached we noticed a flicker of flame underneath the engione. The
driver pulled up ready to gice us a lift and `Batey' told him about
the fire under the engine. When the driver heard the workd fire he
gave one yell, opened his cab door and dashed off. We stood their
speechless for a second and then lifeted up the bonnet to see what was
on fire. It was oil soaked dirt that was buring and a few shovel's of
dry earth put it out. We signalled O.K. to the driver who sheepishly
returned to his truck and we asked why he had dashed off. He said `I
have a load of explosives'. When we heard that we wondered who would
have been the front runner had we known about the explosives.

To show his appreciation for dousing the fire he went out of his way
to drop us off at our orchard camp. I had a near mishap myself on one
of these hitched lifts. The truck load was such that I needed to rest
my feet against the tail gate of the truck. As we whizzed along
bouncing and bimping at the potholes the tail gate began to fall
down. The pins had not been secured properly. I was saved from a nasty
fall by my companions who grabbed me as I slid down the load.

Bol lik sores developed on both my arms and the ointments in the
platoon medical kit failed to heal or stop them spreading. We now had
a \lieutenant MacKay as platoon commander and he sent me to the nearest
R.A.P. for treatment. Here the M.O. diagnosed a form of blood
poisoning probably caused by fly bites. The filth and decay lying
about was a paradise for them in which to breed and feed. There were
swarms of them everywhere. Pills and ointment prescribed by the
M.O.cleared up the poison leaving my arms covered in white scars which
show up when my arems get browned by the sun.

There was now a vast amount of military traffic in and around the
areas of Arrowmanches and Port-en-Beson since the bulk of the supplies
for the allied armies in Normandy were still being landed here. New
routes were constantly being organisedd to try and create a steady
circular flow of traffic and a minor road about three miles long with
exits at each end to major roads was required in the re-routing of
petrol tankers. The road wasn't wide enough for two tankers to pass so
it had to be widened and resurfaced. The road was closed to traffic
and Nos.~2 and 3 platoons with \lieutenant Charlton in charge were detailed
for the work. \Lieutenant MacKay was organiser for materials. The road was
only expected to be in use for a short time and was intended for
wheeled vehicles only.

Building rubble topped with four or fice miles of course gravel rolled
by a heavy roller was used to widen the road and the surface of the
road was then covered with about three or four inches of finer gravel,
finished off with a laywer of hot tar and sand, again well rolled. The
final finish was rather like a well kept drive to a country
mansion. Charlton was so proud of it and enjoyed waling up and dow the
finished part of the road that we called it Charlton's Drive. To keep
out unwanted traffic there were barriers at each end attended day and
night by pickets who also guarded the tools and lit the fire in the
tar boiler eand sprayer. It was one of those very old fasioned things
with iron wheels and was pulled along by a horse. We had no horse and
man-handling it when fill of boiling tar especially when coming out of
the field we used for a depot could be dangerous. One comment passed
by us one morning when dragging it onto the road was heard by
\lieutenant Charlton. The comment was something like: `One of us will get
scaleded to death with this bloddy tar splashing out.' Charlton
shouted: `Keep the damned thing moving, we have allowed for ten per
cent casualties'. The road was about three-quarters completed when the
barriers were left unmanned by the picket one night and a Churchill
tank came clanking along Charlton Drive picking up huge pieces of the
finished surface and tossing them to the rear. When Charlton arrived
with the works party the following morning and saw the trail od damage
he almost had an epileptic fit. He stamped about, crumpled his beret
into a ball and threw it on the ground muttering that he would have
the picket shot. They were dealt with by the C.O. who gave them
fourteen days of extra drills and duties. Some nights later I was a
member of the picket and on duty with Ted Rawlins. Two bulldozwes
drove up to the barrier and the R.E.M.E. sergeandt in charg demmanded
to be let through. Even though he waved a paper which he said was an
order giving him permission to use the road we still refused to move
the barrier. Our orders were no vehicles to be allowed in. After some
argument he ordered the bulldozers to crash through the barrier and
there was nothing we coudl do to stop them. In horror we watched them
drive along tearing out great chunks of road with their tracks. The
damage done by one tank was enough but the damage from two bull-dozers
was unbelievable. Ted and I wondered what our fate would be in the
morning. I almost felt like deserting. About 06:00 hrs \sergeant Wilcox
drove up with one of our trucks and collected all the members of the
picket who happened to be from No.~2 platoon, the platoon had been
ordered to move out immediatedly. When Wilcox saw the damage and heard
our story he laughed his head off. He was no friend of
\lieutenant Charlton's. Wilcox said: `You know this damage could have taken
place after we left. Jump aboard and lets get moving.' We heard no
more about the incident but I would have dearly loved to see
Charlton's face that morning.


A depot company truck took me to Darlington and dropped me outside a
small school building that had been requisitioned by the Army and was
now the H.Q. of 488 Artisan Works Company R.E. This building was
situated in a cul-de-sac and it was only large enough to provide
accommodation for Company office, quartermaster stores, officers' quarters,
cookhouse, sergeants' mess and dining hall for O.R.s. Other halls
nearby had been requisitioned for billets and I was detailed to one of
these: a meeting room over a Methodist Chapel about fifteen minutes
walk from H.Q. The concert hall of a working men's club a bit further
on had more sappers billeted in it. This was the pattern of
O.R. accommodation here in Darlington. There were about twenty of us
in the room which was a vast improvement to those dreary crowded rooms
at Preston. There was no N.C.O. in the billet and we were on trust to
ensure that our presence here would not inconvenience the Chapel
congregation. We had turns to be billet orderly, a duty that I didn't
enjoy. The chores of cleaning up the room and toilets were expected
to be completed by 10.00 hours ready to be inspected. Billets were
randomly chosen for this inspection. Since the room wasn't locked
the orderly was also responsible for its security. After the cleaning
chores were finished, time hung heavily. There was nothing to do but
read and smoke and be alert for a snap inspection. The journeys
from billets to H.Q. for parades and meals were a bind. Buses ran
along this route and had stopping places near to our billets and H.Q.
If we saw the conductor was busy taking fares at the front of the bus
we often hopped onto the platform for a free ride. Some conductors
took a dim view of this free riding and would hurry down the bus to
get our fares. A ring on the bell usually slowed down the bus
sufficiently for us to jump off the platform, often accompanied by
some flowery remarks from the conductor. 

This new company commanded by \major Bell was in an early stage of its
formation. There were now about forty sappers on the roll, very short
of officers and N.C.O.s and quite disorganized. 

There was no playground at this school so our roll call and inspection
parades were held in the road outside, which had quite a slope to it
and it was quite easy to slide and lose your balance. A sergeant who
was acting company sergeant major had a peculiar way of coming to
attention. His right foot swung outwards and came into his left foot
with quite a kick. One morning when getting the parade ready for the
major's inspection he kicked his foot from under him and sprawled on
the pavement: amusing for us but embarrassing for him.

The major, known to us as Ding Dong, was a finicky fault finder and he
always managed to get a quota of men for the 18.00 hours defaulters'
parade to do spud bashing. Twice my name appeared on this list, once
for having rust on my water bottle cork and the other occasion was for
using blacking on my boots instead of dubbin.

When we were dressed in Battle Order for route marches, and we had
plenty of these, the major always checked to see if water bottles were
full and checked the odd cork for rust. They were difficult to keep
clean and free from rust. There was a hole through the centre of the
cork and through this hole there was a wire with tin washers at either
end to hold it in place. The wire rusted inside the cork and water
sloshing round in the bottle washed out the rust stain onto the end of
the cork. That is how I earned one night's spud bashing. The other
time was through applying unpolished blacking to my boots instead of
dubbin. Dubbin is a greasy messy material, ideal for waterproofing
boots but useless for making muddy boots look clean so I often applied
blacking. This particular morning there was a bit of shine on my toe
caps and Ding Dong soon realised I hadn't used dubbin so, on the list
again went Lawrence 0452.

I had a good pair of walking feet and although I had varicose veins in
both legs they were not troublesome so I didn't mind the major's route
marches. He was always in the lead on these marches, accompanied by
his pet, a cross-bred lurcher type of dog. Being on the edge of the
Darlington sprawl we were soon in the lovely surrounding Yorkshire
countryside. The time was early spring when everywhere was beginning
to look nice and fresh again. To me it was a joy to march through
these country lanes where hedges and trees were beginning to look
green again. Quite refreshing after the winter months spent in those
grimy stone-built barracks of Maryhill and Fulwood. The dog also
enjoyed the opportunity to examine all the scents of the hedges and
ditches.

Blancoed webbing was below the major's standard one morning and we had
to spend our Saturday afternoon's free time in reblancoing. While we
were doing this the major's dog paid us a visit. We grabbed him an
daubed blanco on his coat. The major was furious when he saw his dog
covered with khaki green blanco and our route marches became longer.

Darlington was a nice town to be billeted near. It was about half an
hour's walk from our billets to the town centre. There were three
cinemas here to give a variety of films for watching, canteens to meet
the needs of hungry stomachs and, on Saturdays, there was an open
market to wander round. I enjoyed listening to the stall holders
trying to convince the large number of people milling about that they
had the best and the cheapest wares for sale. 

The Ministry of Supply had a small storage depot at Darlington and
sometimes we were taken there to sort through haystack size piles of
sand bags and camouflage netting that were beginning to rot due to
rain seeping through the tarpaulin covers. Stacks of barbed wire rolls
and coils of Danet wire were rusting in the elements and we were
supposed to sort through them and divide the good from the bad. Still
without N.C.O.s in sufficient strength to organise work parties there
was more skiving than work done on the dump.

I had a feeling that 488 Company wasn't going to be a happy one and I
wasn't looking forward to a future in it, so when I read on orders
that I was being posted I was rather pleased, but I was also a bit
apprehensive as to where I was being posted. I had read those orders
as I was leaving the dining hall after Sunday's dinner and my
instructions were to report with all my kit at company office at 08.00
hours to receive my posting documents. A few of my room mates who
were not very happy with the company thought I was lucky to get out of
it. 

We usually had a quiet hour on our beds on Sunday afternoons smoking
and reading and today we were doing just that when a D.R. came in with
orders from H.Q. The company was moving out and we had to pack our
kit and parade in Full Marching Order outside H.Q. at 15.00 hours.
Kit bags and blankets were stacked in the room to be collected by
truck. This put me in a dilemma. Where would company office be on
Monday morning, here or at our new destination?  I dashed up to
H.Q. for confirmation and was told to parade with the others, company
office and staff were also moving out.

There were some absentees from parade when the roll was called. A few
sappers had gone into town immediately after dinner and knew nothing
about this movement. What a surprise they were going to have when
they returned to find empty billets. Arrangements were made to gather
together these absentees and bring them along later. Lucky devils
they had a ride in a truck, we had to march. Our destination was
Staindrop, a village about sixteen miles from Darlington. The first
part of the march wasn't too bad but gradually my pack appeared to get
heavier and heavier. This was my first march dressed in full marching
order and the whole load weighed about one hundredweight. The
afternoon was warm. I perspired freely and my shoulder straps dug
into my shoulders. I was quite ready for the ten minute breaks. At
one halt a sapper flopped down on a wasps' nest. Weary legs were soon
forgotten in the scramble to get away from the angry wasps.

When we reached Staindrop we were marched into a field where bell
tents and small marquees had been erected. After being detailed to
tents and collecting kit bags and blankets we sat down in one of the
marquees full of benches and tables for a meal the cookhouse staff had
prepared for us. They had come by truck to get this meal ready for
our arrival. There wasn't any need for me to unpack my gear since I
was moving again in the morning so I prepared my blankets for bed and
lay on them for the remainder of the evening.


These tents were old and tatty and although the guy ropes were checked
for slackness before retiring, the night air tightened them up
sufficiently to put enough tension at the top of the pole for the pole
to push through the rotten fabric. The canvas slithered down the pole
to smother us. Fortunately the night was fine so we finished our
sleep in the open. Several other tents collapsed in the same way and
at breakfast our tent caught fire, presumably caused by a carelessly
dropped cigarette end.

I didn't fall in with the main parade at 08.00 hours but stood at the
end, surrounded by my kit, waiting to receive my orders. \Major Bell
in passing to take the parade stopped to ask why I was standing away
from the ranks. When I told him that I was waiting to get my posting
documents his memory was refreshed. He asked me if I had any reason
as to why I shouldn't be posted and I told him that I knew of none. I
didn't add that I was thankful to get away from the company. He then
told me I was being posted because the company was now over-strength
with my classification and that I was going to a field company that
had an excellent record. He wished me luck and moved to take the
parade.

When I collected my documents and found that I was going to 278 Field
Company R.E. stationed at Hurworth-on-Tees, a village about three
miles from Darlington in the opposite direction to Staindrop. I felt
like going berserk, I reckoned I need not have done yesterday's march
and could have gone straight to Hurworth. A company truck took me as
far as the bus terminus in Darlington where I caught a bus for
Hurworth. Alighting in the village I made enquiries about where I
could find an Army H.Q. and was told that some soldiers were in a
house along the road in the direction of Croft. With the addition of
my kit bag my load was now heavier and more cumbersome and I hoped
that I wouldn't have far to walk before finding 278 Field Company.

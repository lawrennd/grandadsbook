At Preston station an army truck was waiting to take us to the
barracks in Fulwood, which was a suburb of Preston. The barrack
buildings had that same unwelcoming prison-like appearance that
Maryhill had. There was the all too familiar parade ground,
surrounded by the same dirty looking, stone built, two-storey blocks
of buildings. I was detached to No.~2 section, B Company for training,
and the section was billeted in two rooms, one above the other, in one
of these blocks. Half the section in the ground floor room had
\corporal Blackwell, the section leader with them, and the other half
in a room on the first floor had \lcorporal Derbyshire as room
N.C.O. I was in this half. We were very cramped and overcrowded
here. These rooms, originally intended to accommodate about eight men
in single beds, now had twenty-seven men crowded in them. Three tier
bunks, with a two foot gangway between them filled the room and round
the open fireplace there was a space, about six feet by six feet, with
a couple of benches to sit on. In the scramble to claim bunks I went
for a top one, which was some six feet from the floor. At this height,
I could be sure of privacy and not have any old bod sitting on my bed
as they did on the lower bunks. There was lots of room between the
bunk and the ceiling and, in time, I became quite an expert dressing
there, away from the melee below. One disadvantage was its position by
a window. All the windows were fixed to leave a three inch gap at the
top for ventilation and the draught from this gap caught me in the
shoulders when sitting on my bed. Thankfully, the draught passed over
me when lying down. The technique I had adopted in folding my blankets
for sleeping in reduced the fear of rolling out onto the floor while I
was asleep. The three blankets were folded to form a tube, and to give
me two thicknesses of blanket under me and three thicknesses on top. I
slithered into this, rather like squirming into a sleeping bag. We
were not asked to lay out our kit in a `spit and polish' manner. As
long as it was clean and tidy, everyone appeared to be happy. The
toilets, ablutions, and showers were in the basement and we had a good
supply of hot water.

Well worn stone steps between floors were the cause of many
accidents. Nailed and steel-tipped boots skidded easily on these
steps and there was seldom a day when you didn't hear the clatter of
someone going down faster than he intended and calling the steps some
unprintable names. I had my share of bruises and abrasions from such
falls. 

The rooms were not centrally heated and fires were not allowed
before 18.00 hours. The coal ration, supplemented with scrounged
pieces of timber from the training fields gave us about four hours of
fire and after lights-out the benches were draped with wet clothes to
dry by the dying embers. I'm sure we had more than the normal amount
of wet days during our twelve weeks of training. During one of
these drying sessions, the benches were disturbed by someone coming
in after lights-out. A leg of my trousers landed on a hot cinder in
the hearth which burned a hole through the cloth. I was fortunate to
find a scrap of khaki cloth to sew on a reasonable patch which was
luckily covered by my gaiter, saving me from having to buy a new pair
of trousers from the quartermaster stores. 

Our dining hall was behind Depot Company block, on the opposite side
of the square to our quarters.  A.T.S. did the cooking and cleaned up
the dining halls. They had a terrible cookhouse to work in and
naturally the meals suffered. The food was roughly presented, but I
found it eatable and blessed those who had queasy tummies and refused
their rations which enabled hungry bods like me to have extras. Seeing
the conditions here one could understand why Fulwood and Maryhill
barracks were on the Army's condemned list.

The sapper course in peacetime was one of six months
but it was now condensed into a course of twelve weeks and at the
end of this course we were having our first `privilege leave'.


\Corporal Blackwell was an impatient instructor, and was also the
foulest mouthed, most evil minded N.C.O. one could wish to meet. The
section hated him. \Lcorporal Derbyshire was the exact opposite,
well-liked by everyone and we tried not to give him any trouble in the
room. 

There were no hogmanay celebrations at Fulwood. New Year's Day
was the same as any other day here. We began the course by having
F.F.I. inspection and lectures in the classroom about what we were
going to do here for the next twelve weeks. We also had an elementary
test paper to fill in. The questions were mainly about building and
drainage work. A day or two after this test, about a quarter of the
section had a more advanced test paper to fill in with questions of a
more advanced nature, covering a wider range of building, road
construction, drains, sewers, and reinforced concrete work. When
Holmes, Langley and myself were ordered to report to the Adjutant's
office, we wondered what it was all about. Having taken care with our
appearance we apprehensively approached his dreaded sanctions. 

A staff sergeants' Course was about to begin and our results from
these test papers had shown that we were likely candidates to go on
the course. When the Adjutant told me this we were being interviewed
separately. I was delighted and I could already feel that crown,
bomb, and three stripes sewn on my sleeves. He then went on to tell
me that having passed-out as a staff sergeant I would be posted to the
Sappers and Miners, the engineers of the Indian Army. This put a
damper on the project. I had every intention of getting home as often
as I could while in the army and thought it unfair to Nona to
volunteer for an overseas posting. If the course had been compulsory,
that would have been a different story, but to volunteer for for an
overseas posting, no, and so I turned the offer down. When I saw
Holmes and Langley after their interviews, they were both my age and
married, they also had declined, for the same reason. Other members
of the section thought we were barmy. Had I been a single man, there
would have been no hesitation in accepting such an opportunity. I
believe this refusal became a black mark on my record sheet which I
knew had been marked at the beginning, `a potential leader'.

Training was carried out, in all weathers, on a large training area
behind the barracks for five days of the week. The hours were 08.00
hours to 17.00 hours with an hour's break for dinner. On Saturday
morning we had rifle inspection and went through some square bashing.
On Sunday morning, it was either Church Parade, for which we
volunteered, or some awful fatigues. The afternoons of both days were
free periods.

I always put my name on the list for church parade. I like a bit of
ceremony and much preferred the extra `bull' required for these
parades, to being detailed to some dirty fatigue. I did miss out one
Sunday morning, how I don't know, my name wasn't on the Church
Parade list and I landed in a gang detailed to sort out the fuel store
in the basement. Coal stocks were getting low and we were required to
sort through the coal dust to find usable lumps of coal for our fires.
In no time the air was full of black dust, clogging nostrils and
throats and we soon looked like a troop of Kentucky Minstrels. How I
envied those smart, clean-looking sappers marching back from church
with the Garrison band in the lead.

Most of my evenings were spent in the NAAFI and on Saturday and Sunday
afternoons I went into Preston with anyone from the room who was
seeking my style of recreation. Three fairly regular men I joined
with were Lacy, Macdonald and Owens. We didn't want to pub-crawl but
enjoyed a roam round town, a cup of tea in a canteen and a visit to
the cinema.

The Sapper course covered a wide range of subjects which were
considered to be a sapper's basic knowledge. The drill on the correct
use of a pick and shovel amused many of us, but it is surprising how
many men use these tools to make hard work from them. Tying knots and
lashings and splicing ropes was practised over and over again and a
lot of this practice was done in our own time in the billets. Sorry
to say I have forgotten most of the knots and splices. I couldn't tie
a running bowline and make a crown splice to save my life. We rigged
blocks and tackles on trestles made from stout timbers to lift heavy
objects, built strong points from sandbags and barbed wire, learned a
little about explosives and mines and acquired some basic knowledge
about bridging. Instruction on explosives, mines and booby-traps was
given by a sergeant major, a very cool customer while handling
explosives. Using clay and dummy detonators, we made demolition
charges and were constantly watched for any mistakes. There could be
no room for errors when using the real thing. When we were considered
knowledgeable enough to handle the real stuff, we had to make up a
small charge of explosive and `blow it'. Making a live charge and
inserting the detonator for the first time was worse than throwing my
first hand-grenade. Each move was made slowly and, after lighting the
fuse, we had to walk away to a safe place. Running away was not
allowed because of the danger of tripping over and being in the danger
zone when the explosion took place. I think the worst part of the job
was inserting the fuse wire into the detonator and crimping it on. A
push too hard into the detonator could set it off and take away part
of your hand. My fingers were a bit trembly as I inserted my first
fuse into a live detonator and, after placing it into my charge and
lighting the fuse, my legs wanted to break into a gallop, fearing that
I might have miscalculated the length of the fuse. We received a
basic instruction on mines and mine laying. The skills of this kind
of work would be learned in specialized companies. I had a valuable
lesson in the part dealing with booby-traps. After a lecture on these
we were sent into a prepared area, where all kinds of traps had been
set up. We had to find these boobies, recognise them and neutralise them.
Seeing an officer's revolver and lanyard lying on the ground and
thinking one of the officer's acting as umpires might have dropped it,
I picked it up. There was a fine wire attached to the trigger guard
and the other end tied to a pull-type of detonator. An officer called
out, `Well Sapper, you thought you had found a nice souvenir on the
battlefield didn't you?  Now you are a dead or badly wounded soldier'.
From then on I was most distrustful about any object that could be
`boobied' when in suspected areas. 

On the morning of February 8, the section was standing in a circle
round the Explosives Instructor receiving a lecture and demonstration
on an anti-tank mine. We were joined by the company sergeant major.
He fished out a telegram from his pocket and making sure everyone
could hear, he read out: `\Sapper Lawrence's wife gave birth to a son
early this morning. Both doing well'. There was a loud cheer, many
congratulations and handshaking as I stood there unable to speak for
the lump in my throat. The company sergeant major then told me to
break off training, return to barracks and get ready for a forty-eight
hour leave. I was washed, changed, my gear stowed away in Company
stores in record time and waiting in Company office for my pass and
ration coupons. The company sergeant major also gave me permission to leave the
barracks immediately after dinner which tacked a few more hours onto
my leave.

I caught a bus outside the barracks gate to take me to the station and
boarded a train to take me to Nuneaton where I changed to one
traveling to Rugby via Coventry. I thought the train would be taking
the line from Coventry to Rugby via Wolston. I got off at Coventry
and asked the ticket collector at the barrier when the next train
would leave for Kenilworth. He pointed to the red light of the train
I had just left and said `That is the last one tonight. The next
train will be the milk train at 05.00 hours.'  Swearing to myself
about my stupidity in not checking the train I watched the red light
disappearing round the curve of the Kenilworth, Leamington, Rugby line.
It was too late to think of catching a bus so I set off to walk home.
When I reached Crackley Bridge I thought I might be able to shorten
the walk by walking along the Berkswell Loop line, the line that ran
in front of the houses in Red Lane. This turned out to be an unwise
decision.

It was a dark, moonless night and I found many obstacles along the
narrow path beside the railway lines, which I stumbled over and
sometimes slipped down the embankment. I tried to walk on the
sleepers, but found their spacing, too short for a normal walking
step. Arriving at a point where I reckoned I should leave the railway
and cross the fields, I had difficulty in finding a suitable gap in
the thorn hedge, these gaps were hard to find in the hedgerows that
divided the fields. 

It was very late when I reached home. I had my own latch key so I
gently opened the door and called out. Nona wasn't scared when she
heard me opening the door; she was more than half expecting me. It
was wonderful to see Nona again and to peep at the little scrap in the
cot; our son. Cynthia was most surprised to see her daddy when she
woke up that morning and was delighted with the bars of chocolate she
found in my pack. With so much activity in the house which
accompanies a new-born baby those few hours soon slipped away. 

The section had now moved on for exercises on Bailey Bridge building.
This was a new type of bridge invented by Mr. D. Bailey in 1940.
There were no complicated parts in its construction. It was quickly
assembled, was easy to manufacture, and it was now in use throughout
our battle areas. There were seventeen different parts for building
the bridge with nine others to build the supports at each end. We
were taught to be familiar with each part and its sequence in building
a bridge, but invariably I found myself in a `panel party'. These
lattice-framed panels, ten feet long and five feet high, fabricated
from flat and angle iron, were a six man load and they were carried
upright by means of staves through the lattice work. The staves
rested in the crook of the arms of three men on either side of the
panel. They were quite heavy and it was not always easy to engage the
male and female ends and line up the holes for the fixing pins. During
the training period Blackie met with an accident. 

Normally our mail was issued to us by Blackie when we lined up to
march to the dining hall for dinner, or sometimes he would toss it
round the tables while we were at dinner. This day Blackie was in one
of his foul moods and wouldn't issue the mail. Someone began to tap
the table with his mug and chant: `We want our mail'. This was taken
up by all of us, making a terrible racket. The Duty Officer came into
the hall, accompanied by the provost sergeant and some regimental
police armed with pickaxe handles who stationed themselves by all the
exits. When we quietened down and the Officer had listened to the
cause of the rumpus he sent for a copy of King's Regulations and read
to us that part which dealt with `riotous and mutinous behaviour'.
The punishment we could have received for making the scene was pretty
severe, but we got away with mail being stopped for three days and
having a shortened meal break. It was after this incident that Blackie
was injured. 

While building one of our bridges a panel was brought in the wrong way
round. Blackie went berserk. Mouthing all kinds of obscenities,
pushing and shoving the panel party until we went off balance and
dropped the panel just at the right moment to trap Blackie's foot.
No-one could prove whether this was by design or accident. There were
many unsympathetic smiles in the section when we saw Blackie moving
around the barracks on crutches, nursing his badly bruised foot. 

As well as being a unit under instruction we were also a unit that
would be called upon in an emergency to defend the area against
parachute infiltration and one night we had an anti-parachutist
exercise. The night chosen for this exercise was a foul one, bitterly
cold with icy rain falling. The first we knew about the exercise was
when N.C.O.s burst into our rooms about 02.00 hours and told us to be
outside in ten minutes, dressed in battle order. There was an awful
melee going on down below me as everyone tried to dress in a hurry,
while I calmly dressed on my bunk. In the freezing rain we assembled
outside. The roll was called and we were doubled away to man key
positions and patrol the area for about two hours before returning to
barracks.

There was little time to thaw out and get to sleep before reveille so
we lay in our blankets smoking and grousing about the army. Our
opinion was that the upper stream of officers consulted the
meteorological men to find out when the weather would be at its worst
for these exercises. 

Talking about the melee at floor-level due to our overcrowded
situation brings me to two lads in our room. These two youngsters who
had lower bunks side by side were forever coming to blows over the
jostling and pushing about in the narrow space between their bunks.
They were, however, chummy enough to spend a lot of their free time
together. They kept the room alive with their squabbles. The quicker
tempered of these two lads, they were only just old enough to be
called-up, a Bradford lad with gingerish hair was late coming in one
night and received a seven day C.B. punishment for it. Besides having
to report on Defaulters' Parade at 18.00 hours each night for an
hour's square bashing, he was also required to be outside the
Guard-room at 06.30 hours each morning dressed in battle order to be
inspected by the Duty Officer. To make sure that Ginger was always
smartly turned out and on time, there were always plenty of hands to
help him dress, tidy up his bed or loan him clean equipment. I found
this comradeship in the army most impressive. 

We had learned a bit about map reading and as a test we were taken
into the country one Saturday afternoon, split into small groups,
given Ordnance Survey maps and told to make our way back to barracks.
We had no idea where we were. All signposts and name boards had been
removed as soon as war broke out and we had no compasses. Looking
round I could see a tall chimney in the distance which I recognised as
the chimney of a Courtaulds factory in Fulwood. Some groups began to
try and orient their maps with the chimney and find a way through the
lanes back to barracks. There seemed to be a wide margin for error in
this so I said, `Hang the maps, I'm making a bee-line across country
for that chimney.' Others agreed with the idea and we set off across
the fields. Crossing open country, with our sights on the chimney,
sometimes fording streams that crossed our path wasn't too bad.
However, in the built-up areas it became more difficult and we needed
to ask for directions. Our group arrived back in time to have a
shower and change before tea, as did many other groups, but some got
lost in the lanes and missed their tea. One group who scrounged a
lift on a lorry were seen by the section officer from the top of a
bus, on which he was traveling and they had to do the exercise again
the following week. Although the officer wasn't too pleased with us
for not using our map properly he let us off from repeating the
exercise for our ingenuity in using a known prominent landmark for
direction and taking a direct route to it. 

Almost at the end of the training programme, we had to go through `The
Battle Course', and a Sunday morning was chosen for us to take it.
\Lcorporal Derbyshire had warned us not to wear too many clothes, since
we could expect to get very wet and muddy, a condition we were getting
used to during our field exercises. However I just wore my shorts and
a vest under my denims to parade in battle order at 08:00 hours. We
marched across the training fields until we came to some rough country
and halted at a low wooden fence. On the other side of the fence were
two ponds covered with a green slimey looking weed. Each pond was
fifteen to twenty feet across and surrounded by brambles and weeds.
An officer called us to gather round him while he instructed us about
what lay in front. This, he said, was the beginning of the course and
we were required to vault over the low fence, wade through both ponds,
and follow the trail from the other side. Along the way N.C.O.s would
be giving further instructions and making sure we kept to the
track. He also told us to keep low when ordered to do so, because live
ammunition would be fired over us and that tear gas and smoke bombs
would be tossed amongst us at some point along the trail. While he was
giving us these details, I eyed those horrible looking ponds and tried
to imagine their condition after they had been stirred up. I decided
that I ought to be among the leaders before they became too muddy. On
the word `Go' I vaulted the railings and headed for the ponds. The
water was icy cold and came about waist-high at their deepest points
and they stank like a pig-sty.  Struggling to keep my balance as the
thick muddy bottom sucked at my boots, I sloshed my way to the banks
and stood on dry ground oozing dirty muddy water from my lower
parts. We now had to break into a run for the remainder of the course,
always being urged on by N.C.O.s.  Someone had had a grand time
thinking out and building all the obstacles to make our passage more
difficult. There were all manner of barbed wire obstructions to
overcome, various climbing structures to clamber over, and ropes to
either crawl along, swing along hand over hand, or swing Tarzan-style
over gullies. The hazards were innumerable and ingenious. Puffing and
grunting we came to an open ditch with about six or seven inches of
water in the bottom and we were ordered to squirm up it, snake-like,
on our bellies. To make sure we did squirm, Bren guns on fixed
positions fired short bursts of live ammunition over our heads. We
charged up steep banks to bayonet straw dummies and slithered back
down to the trail. We came to the gas area, and how those N.C.O.s
enjoyed tossing smoke and tear gas bombs into us. I was a bit slow
getting my mask on and suffered a pair of tearful smarting eyes. To
add to my discomfort, the eye part of the mask steamed over and I
couldn't see clearly where I was going.  Another lesson I learned from
experience: apply anti-mist cream before these exercises. At last I
entered the all-clear area and was told to dismiss and make my way
back to barracks. After the ponds' icy water the exertions of the
course had kept me warm, but now, through tiredness I was easing off
and gradually I felt the cold of my wet clothes. Reaching my bunk I
stripped off, tossed the muddy clothes into a pile on the floor,
grabbed soap and towel and nakedly ran down to the showers. A nice hot
shower, a brisk rub down and dressed in dry clean clothes I felt a new
man and ready for any meal the A.T.S. had for dinner. There was the
unhappy thought of cleaning and drying my kit in the afternoon, ready
for Monday morning.  Dirty denims were exchanged for fresh ones, not
only were they wet and dirty but mine were also torn through, catching
on the barbed wire.  My body had not escaped the sharp barbs
either. There were numerous scratches on my arms, back and thighs. One
of the lads in Blackie's room had a bayonet wound in the calf of his
leg. He had slipped back onto the bayonet of a man behind
him. Otherwise no-one seemed to be harmed by our run through the
battle course.

For the last week of the course we did more square bashing than field
work and had tests to find out how much we had learned from our sapper
training. We all possessed a small booklet called The Royal Engineers'
Training Manual issued to us in the early days of our training, which
contained information on all the subjects we had taken on the
course. We were supposed to know by heart all the information it
contained and an officer from H.Q. asked us individually to answer
some questions taken from the manual. At the same time he was looking
for suitable candidates to form an N.C.O. cadre\footnote{A basic unit
of servicemen forming a nucleus for expansion when necessary}. My
questions on the manual were easy to answer but I'm afraid I was a bit
undiplomatic with my answer to his question of whether I liked being
at Fulwood.  Of course I didn't like Fulwood. I hated the place and
told him so and expressed my views about the cramped, primitive
conditions of our quarters, all quite respectfully, of course, so that
I couldn't be booked for being insubordinate. My refusal of the staff
sergeants' course and these remarks at this interview meant no place
for me on the N.C.O.s cadre and I was remustered as a sapper C\&J
grade III. I would have to take more courses to become sapper C\&J
grade I.

We may have been cramped in our quarters but it amazed me how well
organized, and how smoothly the training was carried out. Neither the
companies or the three sections of companies in training ever clashed
or got in each other's way during the training period. Church parade
on Sunday seemed to be the only time when a large number of men were
on parade at the same time. Company training was such that one
company `passed-out' each month and it was now our B Company's turn.


The company passing out parade went off quite well and the following
day we moved over to depot company on the other side of the square
ready for leave and posting. This same block of buildings were the
quarters of men on permanent duty in the barracks. Leave passes,
railway warrants and ration coupons were issued and away we went for
ten days privilege leave. Leave is not a right in the army but a
privilege. 

It is surprising how short those leave days were. There seemed no
time at all to visit the people you wanted to see or try to catch up
with the neglected jobs a man usually does around a home. There was
also something special about this leave: I had a new born son to get
used to. 

Our doctor was very surprised to see how fit I was. The cold, the
wet, wearing damp clothes and the unaccustomed exercises of the past
four months had certainly improved my physical condition. At the end
of my leave I reported back to depot company at Fulwood and was
detailed to a room with some of the men I knew in No.~2 section.
Unlike our training quarters there was ample room here to move about.
 

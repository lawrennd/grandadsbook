The company moved to Aachen, a key city in an area that had been
fiercely defended by the Germand trying to stem the American advance
into the Rhur. When the American forces had surrounded the city, terms
of surrender were offered to the trapped garrison, hopefully to spare
the city from futher destruction adn save civilian libes. The garrison
commander rejected the offer and the Americans began to bobard the
city wieht shell and morta fire trying hard to be selectice with theri
targerts and avoid known concentrations of civitlians. House to house
fighting gradually broke down the Germand resistance and they were
forced to rsurrender. Twenty thougsand prisoners were taken in teh
action and ten thousand civilians emarged from shelters and
cellars. All this had taken place almost a year aog a some services
such as gas, electricity, water and sewage were now functioning in
parts of the city and much repair work and cleaning up had taken
place.

Our billets were in the remains of a school in the suburbs of Aaxhen
and had formely been occupied by the Americans. Being on the
outskirts, the damage here was lighter than it was further in the
city. Blocks of flats although scared by shrapnel an ddamaged by blast
were stil lstanding and were overcrowded with citizens. All that
remained of the school was this one building, our billets, with a
narrow side street running down its side. Small terraced houses were
on the other side which had probably save this part of the school from
destruction. The other three blocks of buildings built on the sides of
a large square had been flattened and set on fire by the American
shelling.

The habitable block had been made into a cosy billet by the Americans,
having central heating and lashings of hot water for ablutions,
provided by a boiler in the basement.

Aachen wsa rathere an anti-climax after Calais. There was still a
relaced form or no fraternisation in existance and civilian and
service filk eyed each other in an unfriendly manner. We were warned
not to be out alone after dark because gangs of D.P.s were roaming
around comitting every crime in the book, although their victims were
usually Germans, lone servicemen were easy prey. There was a
D.P. centre in blocks o flates about a mile and a half form our
billets. The D.P.s here were mostly Russians or Slavs, some of
Germany's forced labour who were now being gradually repatriated or
moved to other D.P. centres. The flats were urgently needed to house
Aachen's civilian population.

A combined local government consisting of Allied Military personnel an
dGerman civiilans had been formed to try and bring about some
normality to the running of local affairs and they controlled a Labour
Force to which all men and women requireing work had to
register. Housing was a major problem. There was so much damaged
property that accomodataion was hard to get and the overvrowding was
aggravated by returning refugees who had fled from the city during
hostilities.

When the last of the D.P.s left the complec of flats, our platoon was
sent to supervise a gang of Germans sent to clean up the
building. What a nauseating job it proved to be. How humand beings
could live in such disgusting filth when there was no excuse for
uncleanliness is beyond my understanding. A Sanitation Unit was
attached to us and before entering the buildings we were liberally
dusted with anti-louse powder. A hearty puff of powder up each trouser
leg, a puff up each sleeve and for luch a puff down the top of the
trousers was a hopful barrier to all lice and fleas. At the end of the
day I had a long hot shoewer scrubbing well with Carbolic soap. I also
washed my clothes, getting them dry was no problem here. Clean denims
were provided every day.

Except for a few personal effects, the D.P.s had to leave everything
behind and the Germand workers were organised to mover throught the
buildings and throw every moveable article out of the windows and burn
it on bonfires. Inside the blocks the stench of dirty verminouse
bedding and old cloths, stale food and exvreta in all corners was
unbelievable. Two decomposing bodies were found in one basement hidden
under piles of rubbish, murder victims or dying from natural causes
was never disclosed. When the flats were cleared the sanitation men
sealed all doors and windows and lit fumigating canisters, locking the
doors behind them. After these canisters burnt out the sanitation men
in gas masks moved in to open all windows ans when they declared the
rooms were safe to work in dozens of buckets, brooms and scrubbing
brushes arrived and the Germans began scrubbing the flats out. In the
party I was supervising there was a young German who gave me cause to
complain about his work. As I spoke to him he made a lunge at me with
his broom and I was almost caught unawares, sice the youth had always
shown a willingness to work. As he lunged, my little knowledge of
unarmed cobat flashed through my head. I sidestepped, grabbed the
broom head and tried to wrest it from his graps. The handle snapped
and the brokedn end struck him in the face, cutting his lip. His short
burst of anger had now subsided and he was now in fear of punishment,
fear aggrevated by angry remarks from hsi fellow workeres, afraid they
might loose tehir jobs. I took the youth to out platoon first aid man
to get a dressing put on his gut lip and reported the incident as
slipping on a wet concrete floor, breaking the broom handle which had
caused the injury. The young Germand and his workmates were very
grateful for not being disciplined and from then in I had no problems
with them.

Arthur Neal was posted soon after we came to Aachen and I missed his
comapnionship. We had taken a few walks together but had quickly got
tired of viewing piles of rubble and the sullen faces of the Germans.

Falling tiles and masonry loosend by the weather made it dangerouse to
walk along the paths and if walingin in th road one needed to watch
for missing drain and manhole covers.

There was no NAAFI in Aachen. For the first few Saturday afternoons,
Liberty trucks took us into Maastricht. There we were allowed to visit
the American canteen where we could buy delicious doughtnuts, coffee,
chocolate and cigarettes. The people here were quite friendly and ever
ready to shake hands and try to converse with you. Some complaints by
the G.I.s about us buying up their supplioes put a stop to these
visits which were greatly missed.

A workshop next to the boiler house had a nice carpenters bench and
vice and I began to spend som time down theri carving pieces of oak. I
had acquired a couple of small gauges and a small flat chisel from a
German and I carved out some of the seven dwarves and a few sailing
boats. Others in the company tried their hand with crafts. \Corporal
Dawson embroidered the Royal Engineers cap badge on the back of his
pull over thieving colored wool from the ends of blankes for his
work. So often one heard a cry of `Bloody Dawsons's been at my
blankets.' after finding that coloured wool of the blanket stiching
had been unpicked. Another sapper using various weird media for his
work painted a large cap badge over the fire place in a room used as a
recreation roo,. It was furnished with chairs and tables, one or two
arm chairs with stuffing sticking out every where, and old piano and
some gymnastic equipment, all dsalvaged by the Americans form bombed
buildings. We held competitions to see who cold lift the heaviest
weights on the bars and tried some club swing exercises. Sing songs
round the out of tune piano an dcard playing became noormal
entertainment. \Corporal Dawson, \lcorporals. Munro, Bates and myself played
endless hours of Crib.

Cleaning redecorating and repairing the flats occupied the platoon for
several weeks. Occaisionally I had a break form the work and went with
segeant Bibby to assess the viability of repairing buildings for the
use fo troops who would be part of the army of occupation. These
reports went to the town Major's office for consideration.

German ex-servicemen with no homes or families to go to, were
organised into Labour units living in small hutted camps and were
emplyed on repairs to the city's roads and sevices. A request for help
from one of thes units came to the company. They were having problems
with building two Nissen huts and I was detailed to go and sort them
out. Collecting haversack rations from teh cook-house, a P.U. truck
took me to the unit and was calling back for me in the afternoon

missing 155

and, thinking the captain might have a liking to it I pushed it down
my gullet. I refused afters, I wasn't taking any chances in that, and
vanished to the hut building's to nibble at my sandwiches. Now the
building sequence was fully understood the huts quickly took shape and
I was able to leave them with the knowledge that they could finish
without further help.

Having had earache for several days I called in company office to have
my name put down on the following moring's sick parade. At 06:00 hrs
that morning instead of the Orderly shout of `wakey wakey' \sergeant
Webster, orderly sergeant for the day, poked his head in the room and
said: `Stay in you beds, today is V.J. day, you have the day off and
breakfast will be served at 08:00 hrs. Just my luck I thought, no lay
in for me. I had to leave my bed made up, kit laid out and with my small
pack report to company office at 08:00 hrs. Ear drops prescribed by
the M.O. cured the inflamation in a few days.

There were no celebrations here for V.J., none of the gaiety we had
enjoyed in Calais on V.E. day. Mooning about the billets and playing
cards occupied the day. Forty eight hour leaves to Brussels became
available and I took advantage of this opportunity to get a pass and
visit the city. My travelling companion was \sapper Horn from No.~3
platoon. We travelled by troop transporters and were taken not to a
nice hotel, as we were in Paris, but to a building with two large rooms
on the ground floor. One room, full of two tie bunks, was our sleeping
quarters and the other room, furnished with tables and chairs, our
dining hall. The food was similar to that served back in Aachen. There
were no sightseeing tours laid on and very little information about
places to visit. Manckin Pig was the only place we knew about. We rode
on the trolley buses free of charge but without knowledge of the
language we found it hard to find the right stopping places and often
had to jump from moving buses, a practice which proved to be
disastorous for me on our second day in Brussels. A little rain had
made the cobbled road a bit slippery and as I hopped off a moving bus
my studded boots skidded from under me and I slid along the wet road
getting my backside soaked in the process.

For me the Brussels visit was a bit disappointing, perhaps after Paris
I was hoping for too much. The Belgians didn't have the same spirit as
the Parisians.

While packing my gear for our return to Aachen the next leave intake
arrived and who should walk into the room but sapper King, `Kingy' of
training days in Preston. He was in the Welsh divisional engineers and
we had time to chat about `Blacky' and our days of training at Fulwood
barracks before my transport arrived.

A day trip along the Rhine was arranged and enough bods put their
names on the list to fill three trucks. Haversack rations were issued
and we left Aachen on a fine sunny Sunday morning to head for Cologne,
cross over the Rhine, travel down to Bonn to recross the river and
return to Aachen. On the way to Cologne we drove through the remnants
of Duran which had received saturation bombing. There was nothing to
see but bricks and rubble. The bumby bomb cratered roads were lined
with walls of broken masonry built six to eight feet high to contain
the rubble. Knowing that thousands had died in that one raid I
couldn't say `Serve you right for what happened at London, Coventry
and other towns and cities.' The punishment was severe.

Cologne, where we made our first stop was another terrible sight. Here
three-quarters of the largerst city in German lay smashed to
pieces. Due to some fine precision bombing the Cathedral was almost
unmarked and there it stood like a lighthouse amoung acres of
desolation. The railway station behind the cathedral was a heap of
broken concrete and twisted ironworks. A little way beyond the famous
Hohenzollen bridge across the Rhine blown up by the retreating German
Army lay rusting and blocking the river. It was rumoured that a German
officer was shot on the spot for failing to obey an order to detonate
the charges.

Cathedrals attract me, I love to wander round admiring the beautiful
craftsmanship put into the building of them. Cologne cathedral is a
fine building and some enjoyable time was spent looking round it. The
black market activity on the cathedral steps was now the big
attraction of the city. For cigarettes and tobacco I obtained a
antique bracelet studded with garnets, which Nona is afraid to wear in
case the garnets fall from their settings. Crossing the river here on
one of the floating Bailey bridges our convoy of three trucks drove
along the Rhine towards Bonn. Being a nice day we had removed the
canvas covers of the trucks and we had an uninterrupted view of the
lovely country side. Many sunken, rusting craft lay in the river and
there were battle scars along the banks, one day I hope to have the
peasure of seeing the river without the reminders of war. By another
floating Bailey bridge we recrossed the river at Bonn and returned to
Aachen. I enjoyed this one day out more than my leave in Brussels

In early October I had another priveledge leave to the U.K. and
travelled most of the way to Calais by rail. The channel was in an
ugly mood. An autumn gale was blowing and our sailing was delayed. The
weather appeared to be improving and permission to leave harbour was
given. The crossing was pretty rough, I believe our ferry boat was the
only one to cross the Channel that day. Most of us were sea sick, my
first experience of this malady and I times I could not have cared if
the boat had been sinking. How quickly one recovers on dry land,
before the train reached London I was feeling fine again, a bit empty
and ready for a smack at Euston. There were the same happy days at
home, helping with the chores and visiting relations.

Reporting to the R.T.O. office at Victoria on the return journey I was
informed that due to bad crossings on the Channel there were delays
and my leave group would not be leaving Victoria until the next
day. Local lads in the group went home for a few more hours, but men
like myself, too far from home to make the journey in time were
directed to one of the large halls requisitioned to provide
accomodation fro these situations. The hall was warm and the bunks,
although blanketless, were comfortable. A decent breakfast was
provided and haversack rations were given for the journey to
Calais. For this trip the Channel was reasonably calm. I had eaten my
sandwiches on the boat and the meat patty they contained wasn't
sitting too well. When the ferry began its manouver to reverse into
harbour the little swaying and slipping made me ill again.

On the train journey various steps were made to allow leave men to get
off and while stationary civilians ran up and down the coaches
offereing coffee, minerals which often proved to be clear water and
other items in exchange for a cigarette. One civilian who stopped by
my open window had some eggs. Fresh eggs were a rare commodity, I
considered myself lucky to be able to barter for two. I had two tins
of cigarettes in my pack. I emptied the cigarettes out , then
carefoully stowed my two eggs in the tins with bits of paper
packing. When we next had bacon for breakfast I took my two eggs to
the serving hatch and asked the corporal cook if he would drop them in
the hot bacon fat. The A.C.C. corporal came back with a sour look on
his face and asked if I was taking the micky out of him. He gave me my
two eggs back and said `these bloody eggs are hard-boiled.'. I was
disappointed not to have fried eggs for breakfast and I also had to
endure some leg puling about how I had carfully nursed these hard eggs
on the journey back to Aachen.

Children who lived in the flats round about were rather a problem, it
was impossible to keep them out of the billet area. As fast as one
hole in thick wire fence was repaired they made another one to crawl
through. They played in the ruined school building's lighting fires
that were difficult to put out. At meal times thwy crowded around the
mess room doors to grab for the scraps of food left on our plates and
search through the waste bins for anything eatable. I wonder they
didn't choke on fish days for batter and bones were stuffed into their
ever open mouts. Some began to bring cans with them to collect scraps,
taken away I supposed for younger sisters or brothers, gradually more
than scraps were left on plates and scraped into these cans. The
rations for the Germans were quite low but I believe it was higher
than rations recently allowed by the Nazi regime.

Afer a fairly large fire in the ruins while I was guard commander, the
orderly officer sent me to the nearest block of flats to warn the
parents about the consequences of the children trespassing. I found an
elderly man who understood some English and through him I explained to
the little gathering of women theat they should try harder to keep the
children away from the billets. I had such a tale of woe telling me
about the overcrowded conditions in the flats and the inadequate
crations they were receiving. When I said that they had themselves to
blame for the plight, they all claimed to be anti-Nazi. How then I
asked did you allow them to have such a strong grip on you if you were
not in agreement about their methods. Fear was the explanation. Stron
Party people were put in each block of flats and tennements who
reported anyone showing signs of being anti-Nazi and often the
reported person was arrested and vanished. It appeared that the fear
of being reported by a neighbour or a family member to these Nazi
spies ensured the Party's success.

The fraternisation ban was officially lifted so meeting with civilians
became easier. Our company, due to demobbed men and postings, was
gettting smaller (\lieutenant Bailey and \sergeant Wilcox had been among those
demobbed) and some civilians were employed in the billets which
relieved us of many fatigues. The civilians were washing up in the
cook-house, attending to the boileres and keeping the area clean. They
were not overworked on tasks and the two men who looked after the
boilers and sweeping up had time to spare while waiting for their
transport. I found them in the workshop making a trinket box and a
fruit bowl. The one making the trinket box was using several different
coloured hardwoods to make a mosaic lid and his friend, after roughly
chipping away unwanted wood, was scraping flutes in his fruit bowl
using pieces of broken glass. Their patient work resulted in two
respectable articles.

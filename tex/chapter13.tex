
With a detail of ten sappers, \Lsergeant Bibby and myself were sent on
detachment to Euskirchen, a town about thirt miles from Aachen where a
Wehrmacht barracks was being repaired and mad habitable for a
battalion of Guards and we wer billeted in a maisonette in th
barracks' married quarters. Leaving the men to sort out kit and settle
in Bibby and I went to the garrison egineers office two or three miles
away to report and get our instructions about the project. Some of
these garrison engineers and town Majors could be quite officious and
to give a good impression we entered the engineers office and gave a
smart salute. \Captain Watson casually acknowledged our salute and told
us to cut out bull. Work was his interest, not wearing his arm out in
saluting, but added that he thought we would know the occasions when
better discipline would be required. We were told to drag up chairs
and smoke if we wanted to. Bibby was a non-smoker but I lit up and
\captain Watson began to put us in the picture. German civilians
claiming to have various building skills and supervised by a general
foreman, Herr Bacht, were engaged to carry out ther restoration of the
barracks which was now far behind schedule. He said that although
Bacht appeared to show enthusiasm with the work when he was around he
suspected that he was in league with the men to drag out the work
using all the excuses he could think of to explain the delays. We had
been brought in to sort out the material problems and to organise and
get better productivity out of the workers. Having given us this short
briefing he said he had arranged a meeting at the barracks fo us to
meet the Guards colonel. He also said he had a few crates of champagne
under the stairs and we could have a few bottles cheap. We cheeked him
into selling us a couple of dozen bottles for a few marks, two or
three marks a bottle I believe we paid. We suspected it was black
market stuff.

The colonel and his H.Q. staff were accomodated in the undamaged part
of the barrack office block and a platoon of Guards on security duties
were billetted in one of the partly finished blocks of buildings.

This was a time for saluting which we did as we entered the colonel's
office to meet him and his senior W.O. II \SergeantM Macdonald. We
were all invited to sit down and \captain Watson explained what he
hoped we would achieve here. From the colonel's talk we could see how
disappointed he was about the slow progress in repairing the barracks
saying the ano blame could be attached to \captain Watson who had
recieved little help on the prokect. The German work force he
considered were not workiing efficiently and he was desperate to get
his first company of Guards, formed long ago in England, established
over here. Any help he could give by way of transport to fetch
supplies we had only to ask for it adn turning to S.M. Mac told him to
see we had priority with any requests. His parting words were `My
office is always open to you if you tinkg I can help with your
problems, now get my barracks habitable.' S.M. Mac, although a strict
disciplinarian with the Guards, and always bawling at them to get
smartened up, was not the S.M. Ogre with us. In fact he became a very
useful ally during our stay here.

We left the colonel's office and \captain Watson took us on a tour of
the barrach complex, explaineing as we went along where our priorities
lay. Herr Bacht had joined us and hinted that he thought he should
have been at the COlonels meeting. He looked very deflated when
\captain Watson told him from now on we were in charge of the work and
he would take orders from either \lsergeant Bibby or \lcorporal Lawrence. While
walking round I noticed groups of men who did not appear to be fully
occupied and when I asked why, Bach quickly said that they were
waiting fro materials. \Captain Watson raised his eyebrows at us and
said: `See what I'm up against, I hope that you can sort it out.'

Our billet was on the second floor of the building and two middle aged
civilian women had been engaged to cook our meals and keep the place
clean. The captain had a mug of tea with us and looked round to make
sure his requests had been carried out before leaving he said he would
look in later.

Our team of sappers were all tradesmen and while Bibby and I sorted out
our plan of campain they were sent out to find out what materials were
readily available in the barracsk. Bibby said he would look after our
rations and getting materials and detailed me to organise and
supervise the civilian workers.

Some buildings in one block in particular were severely damaged by
shelling and were being cleared of rubble. Unstable walls were pulled
down and usable materials salvaged for use elsewhere. Two blocks of
barrack rooms and ancillary buildings were our proirity and exept for
glazing and painting part of one bloxk was almost servicable. The
sappers search for materials revealed that glass was high on the list
of shortages and was most urgently required. \Sapper Horn, our
tradesman painter, travelled miles picking up crates of glass that had
been hidden. The Germans tried to conceal glass stocks for their own
use and it was highly valued on the black market. M.P.s were
constantly unearthing hidden stocks of glass, paint, plumbing and
heating filaments.

Grabbing Herr Bacht to interpret for me, I began a systematic tour of
the workforce and where I found groups of men without work becaus they
had no materials I supplied the m with shovels and barrows and sent
them to clear up severly damaged buildings. They protested about
tradesmen being given labouring work and Bacht tried to back them
up. `O.K.,' I said 'You will be dismissed because we are short of
materials. I'm sure the town mayor has other work for you and I will
get some unskilled men to replace you for this clearing up operation.'
Not wishing to loose what I'm sure had been a cushy job, they sullenly
went away with shovels and barrows. I then turned on Bacht and told
him in no way were tehse men going to drag on the work and from now on
I would expect a reasonable day's output. He whined that I couldn't
expect the men to do more than they were because they were weak from
low rationing. In reply I hasked how much food had the Germans given
to the forced labour units made to do heavy work in unfavourable
conditions and what about the rations given to the inmates of the
concentration camps. There was nor reply to that, only further shining
of dislike in his eyees. Afterwards Bacht made a show of spurring the
men on while we were around but I felt that I had not quite got a grip
on them.

With the advance party of Guards in the barracks, the civilians were
trading goods for cigarettes. Most of our cigarettes were marked
H.M. Forces only and they could be confiscated from civilians who
where in posession of them. I asked \sergeantM Mac if he would have the
workers searched once or twice a day as they left the barracks and
explained my reasoning for it. He readily agreed and when marked
cigarettes were confiscated a howl of protest went up and we were
called `thieving Tommies' Bacht and a couple of spokesmen approached
\sergeant Bibby to complain about being searched at the gate and Bibby
told him: `This is \corporal Lawrence's province, you must talk with
him. I told Bacht and his complainers that when I would see some
improvement from the workers I would have a word with Guards' sergeant
major about the gate searching.

In a couple of months the colonel was able to send for his first
company of Guards and young, newly appointed officers in the company
expecting an receiving from the Guards salutes at all times began
complaining about the sappers disregard of saluting. Royal Engineers
standing orders on saluting stated that when we were on works we
saluted officers on first parade and from then only saluted if
directly approaching an officer to make a request or if an officer
approached you with a directive. \Sapper Crounch looking after the
central heating part of the project was pulled up and berated by one
of thes new officers for not saluting him and Crouchy complained to
\sergeant Bibby. The other members of the team said that if saluting
was enforced contrary to our standing orders then they would begin to
make sure there were no meetings in the barracks between them and the
officers. Bibby asked to see the colonel and after he had stated the
reason for the interviews the colonel said `Yes, I have had complaints
about your sappers' lack of saluting.' Bibby quoted our standing orders
but the colonel said: `You are now in a Guards establishment and if
my men see your sappers not saluting I fear discipline might suffer.'
Bibby replied that if saluting by the sappers at all times was
insisted on despite standing orders then he feared they would begin to
hide away form the officers and work would suffer. Pondering a little
while the colonel said: `I have to give way to your standing orders
and I will speak to my officers but for heavens sake don't let work
slack up now it is begining to show signs of improvement.'

\SergeantM Macdonald laughed when Bibby told him the story of
saluting and said: `I have the young gentlemen on the square this
afternoon for a drill session. I'll warm them up for you. He did
too. I watched that parade from a discreet viewpoint and Mac certainly
gave the officers a hard session of drill. I was glad I wasn't part of
it.

When \sapper Crouch told us that the part of the officers mess where
the junior offices were quartered had to be isolated from the heating
sysem because of a faulty valve gasket we suspected but could not
prove that he might be getting his own back on the officers. It was a
few days before we were able to get a replacement of this particular
valve gasket and the officers shivered in their quarters. We were now
getting sharp frost sand flurries of snow. For those officers who had
been decent with us in the early days I rustled up a few oil stoves to
heat their rooms. I'm sure we were suspected of engineering the
breakdown for we had no more inteference from these new officers. The
women engaged to cook and clean for us were dismissed. They would
insist on mixing in carraway seeds with the cabbage and potatoes which
no one liked. They were suppling the carraway seeds which I think they
considered to be a fair swop for some of our rations that went
missing. Knowing what the hungry kids were like back in Aachen and
thinking they had young families to feed we were prepared to accept
the missing rations but carraway seeds every day were too much. With
the extra Guardsmen coming into the barracks we were able to get the
women engaged in the Guards cook-house. A cook's assistant, Norman
Higgs, was sent from Aachen to do our cooking. He was a confectioner
in civvy street and was a general dogbody in the cook-house - washing
up most of the time. He moaned about being here and spoiled so much
food that we soon sent him back. One time he put so many dried peas in
a vessel to soak overnight that the floor was covered with peas the
next morning and were ruined.

Taffy Lewis, the same Taffy Lewis that had been on leave with me in
Paris, and was here doing general duties, voluteered to try his hand
at cooking as well as being billet orderly. He was quite a success at
cooking.

Sometimes Bibby was invited to join the Guards senior N.C.O.s and
officers on hunting trips in the Aachen Forest and he would return
with either hares or pieces of wild boar meat. Both were delicious to
eat. the hares were the biggest I have ever seen. They were twice the
size of English hares.

To relieve my boredom of the barrack walls I sometimes went out to
collect materials and had two eventful rides while doing so. One of
our Guards drivers was an alchoholic Scotsman who drove recklessly
with the accelerator pedal hard on the floorboards. With him as my
driver we were travellin along a country road, which was just about
wide enought for two vehicles to pass in comfort. We caught up with a
horse and cart quietly plodding dow the centre of the road and no
ammount of horn blowing could get the driver to move over let us
pass. I imagined that he was sitting their and thinking something like
`Let them wait'. The vergers were neither wide enough or suitable to
mount and get by and Jock not known to have patients mumbled `I'll
shift the bugger' then quietly eased upto the rear of the cart and
accelerated. The horse had to wake up and gallop or be run over by the
cart and Jerry driver came to lifer waving and shouting at us. In a
few yards he was trying to steer his horse and cart into a side road
where we left him sining our praises. I bet he moved over the next
time a horn sounded behind him. The other eventful trip with Jock was
some days later. Taking a corner much to fast the truck went into a
slide and we ended up with the trucks nose in the sitting room of a
cottage. Without checking the damage. Jock engaged reverse gear and
backed out bringing down more wall and leaving the roof sagging. We
were unhurt but shaken and the study Austin truck was driveable, so
after leaving the college occupants particulars as to whom they should
report to we drove on. Our story at the court of enquirey that an
uncontrollable skid on loose gravel had caused the accident was
accepted and no further action was taken. We were lucky that there
were no witnesses to say we were travelling too fast.

Although every effort was made to keep drink away from Jock he did
occasionally get stoned and violent and was hidden away to sober
up. Jock was never disciplined for his drunken bouts, just carefully
kept out of sight until he was quiet again. From bits and pieces of
talk the full story was never divulged, we gathered that Jock had
distinguished himself during the fighting. He had been a tank driver
and, the colonel unable to get his courager rewarded, wanted to keep
Jock's army record clean. Jock wsa soon due for demob. and I believe
his addiction to alcohol began during the campain. When sober he was a
good companion and S.M. Mac said he was and ideal soldier, but oh
dear, when he craved for a drink he was quite embarassing.

Had it not veen for the pressure of work here, life would have been
even more boring thatn at Aachen. Our evening meal was a time when we
were all together and it was taken leisurely, discussing the work
programme, relating our day's adventures or talking about more general
things.

\Sapper Bill Heath told a good story about one of his trips. The owner
of a sand pit had invited Bill to have a drink and he was taken into
the house where Christmas decorations and the Nativity scene were
being prepared. In our mixture of English and German words Bill asked
what ehwy would have for Christmas dinner and the man pointed to a
plump dog lying on the hearth and said: `Gut essen, Gut essen.'
Whether the dog was eaten at Christmas I don't now but bill said that
he felt ill thinking about it and come to think of it I think we all
did. For an excited debate politics and Winston Churchill was all that
was needed to get Bill almost foaming at the mouth. He hated `Winnie'
his victory V sigh said was an abnormality to his fingers caused by
holding his big fat cigars. Except fot those political scenes we were
a harmonious group. \Sapper Godwin, a Cumbrian man, was a monumental
mason in civvy street and his talk about how ston was quarried,
selected and worked interested me a great deal and \sapper Jenkins, a
plaster caster, was another tradesman I enjoyed talking to. He made
some pictures to hang on th e wall using old photographs from
magazines then casting a plaster back to them using saucers for
moulds.

We returned to Aachen each weekend to exchange laundter and pick up
mail, leaving Euskirchen after work on Friday and returning at 08:00
hrs on Monday. Familiar faces were disapperaring each week as postings
and demobilisations took effect. For a few weeks a new sergeant major
was attached to the company and one Saturday morning he had a special
juniour N.C.O.s' parade. After drilling us on the square for a while
he said: `I will call out a name and ask that N.C.O. to give the squad
a specific drill order. My name was called with an order to change the
squads direction. I didn't hear him and the squad with grinning faces
carried on with the last order until the sergeant major called a
kalt. He gave me a sergeant major's kind of dressing down but later
apologised when someone told him I didn't hear very well.

The second company of Guards moved into the barracks in early December
and we were informed that our work here would endat Christmas and that
the company was being dispanded in the New Year. Some of the company
funds were spent to buy extra fare for our last Christmas as a
company. Five geese were brought and so afraid were we that they might
be stolen that a volunteer picket was organised to watch over them.

\Sergeant Bibby and the other sappers left Euskirchen early on
Christmas eave but because our rations from the Guards would not be
ready until after midday I was left behind to collect them. I now
discovered Bibby was drawing rations for two more men than we had in
our party and now understood why he wanted me to give him the rations
before handing them into the cook-house. The Guards tried hard to
persuade me into spending Christmas with them but although I'm sure I
would have had a marvelous time with the, I declined the invitation on
the grounds of wanting to be with my friends probably for the last
time, a feeling thwy readily understood. In the rations there was a
leg of pork, and before giving them to Bibby, \sergeant Whitehouse of 3
platoon offered me a bottle of whiskey for it which I accepted. Bibby
was mad with me but there was nothing he could do about it. I had
played him at his own game. The whiskey I shared with the rest of th
Euskirchen party. There was a lot of eating and drinking during the
Crishtmas break then larger posting took place. Many of the younger
men went to maintenance units in the Bailey bridges over the Rhine.

In January I was given another U.K. privelige leave. and had a cold,
long journety to the Hook of Holland. A hutted transit camp had been
built about a mile from the harbour. and looked as though it was a
temporary arrangement. It was small compared to Calais and had no
NAAFI facilities. Duck board paths between huts and Mess hall were
laid over muddy ground and the whole place was prison-like and
cheerless. Ferry boats from the Hook were larger than the ferries of
the Channel and it was arranged that we had one night at sea and
landed at Harwich in the early morning.

Garth was almost two years old now and was changed each time I saw
him. Cynthia was a second little mother to him but got quite cross
with him when he found and ate some of her hidden chocolate. Garth was
still unable to make out who I was. He had been told I was his Daddy
but to him I was just another mand and often asked his mum when I was
going home. On the journety back I spent several days in the transit
camp at The Hook and was away from the company for over three weeks.

The company had shrunk drastically in those three weeks. I was now the
only occupant of our room. It was a safe bet that your turn for a
nights guard or picketting came on the following night of your return
form leave. Whether it was thought that night duties gave you more
time to reflect on the good time you had enjoyed on your leve I never
understood. The duty usually made me feel more browned off. I was
guard commander the night after this last leave and about 23:00 hrs.
I went out to check the man on the gate and couldn't find hi,. All
kinds of thought went through my mind. Had he been kidnapped or
murdered and hidden away. I put another man on guard and sent others
in pairs to search the area. In a ahort time the missing man, \sapper
Joycey shoede up and I immediatedly put him under arrest. The stupid
nut playing on the lax conditions of the depleted company had gone
down the road with a `pro' hoping that he wouldn't be missed for a
short while. Konwing he was due for demobilisation in a couple of
weeks I decided to consult the C.O. before logging the incident in the
guard report. The C.O. was very lenient with Joycey who could have
been court marshalled and gave him a lecture and had him locked in a
room until his demob. release.

I was having a quiet read and a smoke one evening when an officer
orderly came to my room to tell me the C.O. wanted me in th
office. The C.O. handed me some papers and said `I have had a request
from H.W. to look for a suitable candidated to take a staff sergeants
course and I find you have the required qualifications. Take these
papers away, fill them in and return them to the office in the
morning.' In my room I carefully read tehm through trying to
understand the small print and after the episode at Preston looking
for unexpected conditons. The six months course was being held in
England which I found rather tempting but a heading onone of the
papers I had to sign puzzled me. There was a hint about the cost of
the course and suggested that unless I was a complete dolt there was a
reasonable chance of passing out as a staff sergeant and would be
placed in some unfamiliar form of army service. This sounded to me
that when my demob group release came through I would still be
retained for service. I popped back to the office and fortunately the
C.O. was still there with \sergeant Whitehouse who was acting sergeant
major. I explained my dilema and my reluctance to sigh the papers. The
C.O. felt sure I would be released when my group no. was announced by
\sergeant Whitehouse, after reading th e paragraph being questioned,
had his doubts. The C.O. rang the H.Q. for clarification and the
answer was yes, my signarute to accept a place on the course was also
a signature to signing on for another three years in the army. `Sorry
sir, I cannot accept those conditons.' I said and promotion passed me
by again.

A few more dreary days dragged on and then with \lcorporal Jack Thompson I
were posted to D.C.R.E in Soest.



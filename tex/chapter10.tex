
We moved much nearer to Bayeax and set up camp in another cider apple
orchard about two and a half miles form the town centre. The ground
here was very stoney and made hard work of digging out the tent
floors, not for blast protection which was now unnecessary for we were
something like three hundred miles behind enemy lines and to be bombed
by the Luftwaffer was now very unlikely, but to get extra headroom
inside the tent. A small, empty, single storey cottage with
out-buildings stood at the entrance of the orchard. Soors and window
frames had banished, vandalised for firewood I suppose. The largest
room of the cottage, when cleaned up ans with hessian tacked over
openings became the cook-house and ration stroes. Meals were severed
through a window and taken away to be eaten in our tents. \Lieutenant MacKay
and the sergeants used the out houses for their quarters.

The platoon had been sent her to start the operation of building a
hutted hospital in fields opposite to the orchard. Survwyors from
D.C.R.E. came to mark out the site and fix the building levels and for
a while I was attached to \corporal Wilkes of D.C.R.E. pegging out
Nissen hut bases and the connecting pathways between huts and Romsey
hut wards. Hedges alongside the road were pulled out, the ditch was
piped and filled in and huge mountains of sand and gravel were bought
to the road side by R.A.S.C. tipper trucks. A battery o fconcrete
mixers were lined up by this aggregate, with tons of cement staked
alongside and water was piped to the mixers form the town supply. On
cold mornings these diesel mixers were devils to start. The starting
handles, situated very close to the frame work of the mixers, required
some vigourous turning by relays of men to turn the enging over and
get them to start. Often back fires occured throwing the handle out of
the dog and causing one's knuckles to come into contact with the
framework. Skinned knuckes and much swearing often began the mornings
work.

The rest of the company moved into similar camps to ours round the
hospital site, which soon became a hive of activity. A company of
German P.O.W.s escorted by Pioneer corps guards were brought from
theri compound each day to help with the work. Many had served part of
their service on the Russian front and were without either fingers or
toes lost through being frost bitten. Some were still a bit pro Nazi
but the majority were only too pleased to be P.O.W.s and enjoy the
small comforts of their compounds. The hard cases were gradually
whittled out. One such case spat in Sgt. Websters face when being
detailed to work. With great control Webby overcame the urge to smash
the prisoner's face in, who was put under arrest and taken
away. Heaven knows what happed to him; he was not seen on the site
again. The Pioneer Corps Sgt told Webby that the prisoner would be
suitably punished.

Teams of P.O.W.s directed by sappers stripped off the turf from the
hut based and wheeld it away. Dozens of other P.O.W.s wheeled barrow
loads of ready mixed concrete to the shuttered bases to be tamped and
screeded by sappers. I was now in a detail with \lcorporal Munroe
shuttereing bases and erecting posts to carry the roofign over the
covered ways between thuts and wards. There was still road work to be
taken care of and one day I was with a gang working close to the
Mulberry Harbour. An endless flow of vehicles were coming and going at
the harbour where several ships were berthed unloading stores and
equipment and reloading with P.O.W.s and wounded.

It was difficult for me to believe that those large concrete
structures had been fabricated in England, towed across the channel,
sunk, and assembled to from this artificial sheltered harbour in such
a short time. The day was warm and having time to spare befor our
transport was due Sgt Webster who was in charge of the party suggested
we clambered down to the beach for a quick dip in the sea. An odd
character in our platoon known to us as the professor didn't want to
go into the sea and was left to look after our clothes. We called him
the professor because he had thin sandy heari fringing an almost bald
dome, wore steel rimmed glasses which were frequently on the end of
his nose and was always theorizing or arguing how work should be
done. While we were bathing the Professor made tea, using some
scrounged materials he had in his haversack such as tea cubes,
Methaldehyde tablests and a stove for heating the water. He made two
mess tins of this brew and he appeared to enjoy drinking it. I was
invited to have a drink. It was vile tasting and I quickly spat it
out. Webby who had also been offered a drink swallowed some of it
before getting the taste, he gagged and swore the stuff was
poisonous. The tea had been made with sea wate slightly tasting of
diesel oil and was sweetened by strawberry jam to camouflage the
taste. The Professor was quickly debagged and tossed into the sea.

There was a shower room in Bayeax and once a week we were marcched
there by sections for a shower in warm water. This weekly outing was
eagerely looked forward to and with clean laundry appearing more
regularly pesonal higene improved. We had a piped supply of water in
camp but fuel for our lazy man's boilere was difficult to get and
ablutions had to be made in cold water.

A cider press appeared in the orchard and a couple of elderly men
gathered the apples in large wickere baskets, even pickingup those
from the ground that we had thrown at each othere. They were put into
the press in alternateing layers of staw mats and apples, then by
screwing down the press slowly, every drop of juice was squeezed from
the apples to drain in a trough round the base of the press and then
be baled into barrels. Apparently this mobile press, owned by these
two men, travelled round the distric pressing apples for the orchards
respective owneres who then made their own special brand of cider from
the juice. They mad a potent brew, nice to drink, and when sitting
down it semed to have no effect on the body, but, as soon as you went
into the air, legs became very rubbery and difficult to control.

A small NAAFI hut was opened between our camp and Bayeax, so poorly
stocked and over-crowded that I seldom thought it worth going
in. Later when a large Romsey hut replaced the Nissen and a larger
selection of goos became acailavle it became a stopping off place when
taking a walk to Bayeax. Three sappers I frequently went out with were
Fred Harper, Arthur Neal and Ted Rawlins. When I wanted to be alone my
favourite place for a visit was the Cathedral where I spent many hours
admiring th beatiful work in there especially the work in the side
chapels. The famous Bayeax Tapestry was another piece of work I
enjoyed lookking at. This was a replica of the original which had been
placed in a safe place but even so the artistry of the needlework
could be appreciated. I had read or heard somewhere that the tapestry
could well have been a pictoral representaton of our invasion of
Normandy and how true that statement was. The illustrations of
William's preperations to invade England in 1066 could have been a
cartoon of our preperations in England to invade Normandy in 1944. and
so right through the tapestry I found the comparisons.

The town of Bayeax, hardly touched by the fighting during the landing
offensive was an interesting place. So many of the streets I
discovered still had that medieval look. They were narrow and cobbled
with opens and disgusting looking drains running down the middle of
the road and many of the houses were half timbered. A few stalls in
the market square sold vegetables and fish, some fo the weirders and
most revolting looking fish I have ever seen.

Autumn rains made camp life under canvas most uncomfortable. The
ground chruned into a muddy mess, especially on the site. Blankets
became very damp and must smelling, I seemed to be always ddressing in
cold, wet clotehes and my foot wear began to fall to pieces. I had
sent one pair of boots to be replaced at H.Q. and they should have
been returned to me in a week, but after several weeks there was no
sigh of them. The missing boots had been reported to Segeant Wilcox,
now in charge of the platoon, and he had been unable to trace
them. Eventually m remaining pair began to fall to pieces and the
flesh of my feet started to go white and soggy, the first signs of
trench foot. One night, when leacing the hospital site, Segeant Wilcox
picked me up and drove me to H.Q. to show our C.O. the condition of my
boots and feet and to explain about the lost boots. What happened
behind the scenes I don't know but I was issued with two new pairs of
boots and socks and my feet began to improve.

There was little we cound do in the longer dark hours of the evening except read or play cards in the dim light given by our crude diesel lamps. Reading material was a scarce comodity and a pocket bible given to me by Nona was oten read, especially old testament stories. \Sapper Dickson in our tent wrote home and sasked his wife if she could rake up some magazines to sent out, he nearly exploded when he received a bundle of Women's magazines. Women's magazines or not they were still well read before being discarded. It was fun chasing after numbers to continue reading the serials. 

Our platoon officer \lieutenant MacKay wasn't an inspiring leader. There were times when I wondered how some of the officers got their pips.

\Sergeant Wilcox ran the platoon most of the time and he was ably
supported by \lsergeants Webster and Bibby with \corporals Dawson and
Wimpey (a nickname he was given for having worked with the firm of
Wimpey) and \lcorporals Munroe and Bates. Many mornings as we paraded for
roll call at 08:00 hrs and were dismissed to works, \lieutenant MacKay would be
seen sneaking out of camp with a rattly old bike hie had found to
cycle into Bayeax for Mass. He was a Scottish Catholic and appeared to
be more concerned about his soul than the running of the platoon. When
he did take the 08:00 parade he invariably came out with `A wee word
about the job' meaning the hospital project. His talk was seldom
constructive to the job and his words were usually ignored. The
exception was when he told us that the R.A.M.C. colonel was getting a
bit worried about his patients having to spend winter under canvas and
that our C.R.E. said it would be a wonderful gift to the wounded if we
could get them into huts by Christmas. The hutted hospital was nowhere
near finished but we accepted the challeng. We were almost exhausted
every night when we dinished work. \Lsergeant Biddy had charge of all
carpentry work and part of the programmwe was to hang six doors a day
in the nursing staff Nissens. complete with all fittings, stops and
achitraves. I had a young P.O.W., a very willing and helpful worker,
to fetch all I needed from the dump. The P.O.W.s were now adopting
those they wished to work with and if detailed elsewhere went to great
lenghts to get to their original work. The system pleased us as for we
got to kow each others signs of communication. Although not completed,
by mid December the hospital became far enough advanced for the
R.A.M.C. to take over and bring in their staff, stores and
patients. It was a wonderfull reward to our frenzied effort to see the
wounded and sick patients in a warm draught proof Romsey hut wards and
the Nursing staff housed in Nissen huts. Covered concrete footpaths
between huts and wards were also nearing vompletion, an improvement on
the duckboard walks of the tented hospital.

The failure at Arnham to complete the `Corridor into the Ruhr' planned
by Feneral Montgomeery and the difficulties encountered to capture the
Port of Antwerp, the taking of which would have shortened our supply
lines, meant that the German armies could not be beaten into
surrendering before Winter, so, the high command were forced to
implement a winterisation programme. More huts became acailable to
replace tented camps and more buildings were requisitioned for billets
to house our troops. Our platoon moved into wooded huts farhther along
the road from the orchard. HTe huts had slow combustion stoves in them
and we had a reasonable supply of coke from the hospital stocks; now
we could dry wer clothes and keep blankets aired. The hospital had its
own generators for making electricity and we were connected to their
system enabling us to read and play cards much easier.

My third army Christmas was approaching which looked as though it
would be the most miserable of the three. True our accomodation had
now improved but moral was a bit low. Much speculation about how
successful Runstedt's breakthrough at Ardennes would be and whether we
could become involved in the fighting areas again. Also we wondered
what effect it would have on the rumour that leave to the U.K. was
begining in the New Year.

On Christmas Eve, feeling a bit unsociable and not wishing to go to
the NAAFI, I walked into Bayeax to a church army canteen. It was only
a small place where you seldom found more than a dozen bods having a
cup of tea and joining in the discussions led by the captain. The
captain come in late that evening and he said that he called in at the
NAAFI and found it unusually quiet. He though everyon appeared to be
homesick. He suggested theat we should go there and sing a few carols
to try and introduce some Christmas spirit. We all agreed to his
suggestion and soon the NAAFI was resounding with carols. A further
suggestion was made of carolling round Bayeax and a group of about
thirty made for the town. At each point where we stopped to sing, the
civilians gathered round and tried to join in. There was soon a crowd
of a hundred or more walking to various vantage points to sing
carols. The crowd, although merry, was not boistrous in any way and
the carols were well sung. A few bottles of cognac and cidere passed
round helped to keep throats lubricated. In a square surrounded by
large housed occupied by various Army H.Q.s Staff Officers came out to
join in the singing and one Brigadier was full of prais for the Church
army captain's brilliant idea. The people of Bayeax must have talked
about than night for a long time afterwards, for me it was one of my
more memorable Christmas Eves.

Leave rumours became a reality, Pay books were called in to find those
who had priority for leave. To be eligable one had to have served six
months overseas asn last leave home was taken into consideration. It
too many men of a company were eligable then ballots were drawn. I had
six months overseas service and had not officially had leave for
nearly twelve mont. My A.W.L. were not recordered so I was high up on
the list of eligible personnel. Most of the company had been on leave
shortly before D.Day. U.K. leave began on Jan 1st 1945 and I was
issued with an eight day leave pass for 11th-19th January.

The rough treatment my battle dress had received during the past six
months had not improved its appearance.

Hard brushing to remove mud had left it stained and threadbare. Tears
were not exeactly invisibly mended and it looked decidedly tatty. Full
of high spirits about going home, and ignoring my appearance I walked
into Bayeaz early on the 9th of Jan to be picked up by a troop
transporter that was touring round collecting leave personnel. A leave
camp had been set up in Calais and our journey there was a very cold
and uncomfortable one. Stops were made along the way for tea and
sandwiches and to stretch oiur legs, also to rendezvous with other
transporters. We were now in a convoy of leave transports and
arrived in Calais late in the afternoon. The welcome we received a
t Calais was very different from the reception I had experienced at the
other camps. We were quitedly ushered into our code numbered leave
groups without the usual bawling and shouting from N.C.O.s and led to
our sleeping quarters. The Nissen huts furnished with double bunks and
centrally heated were most comfortable. Ablutions and showeres with
hot water were plentiful and after allowing us time to freshen up we
were escorted to the Dining Hall for a piping hot meal. After the meal
I had a mooch around and found a NAAFI plentifully stocked with
goodies and my stock of chocolate and soap to take home was quickly
increases. ENSA had a show on in the Romsey hut and to while away the
evening I sat and listened to the near blue jokes and patter of the
ENSA artists and joined in with the sing-song. Vera Lynn's `Blue Birds
over the White Cliffs of Dover' and `Till we meet again' being two
very popular ones. After breakfast the following morning we were taken
to Q.M. stores and issued with new Battle dress. Unfortunately my size
of blouse being a popular size was not available and the one I
received had too big a collar awith a chest that was much too full. My
hair needed some expert attention so I paid a visit to a barber's shop
in the camp run by Frenchman. My French was rather week and their
English was just as poor and after nidding my head to their various
hand sighns I endedup minus on hundred francs (then equal to on pound
in English money) smelling like a perfumery and with my hair plastered
down with hair cream looking like a pimp. A souvenir photograph was a
must and I joined a queue to wait and have my photograph taken which
was promised to be ready at the end of the day. With a few pins my ill
fitting Battle dress blouse was mad to look a fit then the cameral
light flashed. The end result was not a `pin up' photograph and for
some time afterwards when Nona showed it to Garth he meowed like a
cat.

My leave boat was leaving Calais at 08:00 hts the next moring so I had
an early night. The ferry boat was crowded, there were so many men on
deck that you could hardly see the planks and below was just as
crowded. The crossing was calm which suited me for I lay no claim of
being a good a sailor. At Dover we passed through the Customs sheds
without being checked and boarded trains for London. Banners of welcom
and flags of all kinds fluttered from buildings along the way and at
Victoria station, also ablaze with banners, W.V.S. ladies served us
with tea and buns. At Euston I caught a train for Coventry and sat
back to relax. What a welcom when I reached home. The children had
altered and grown so much during those six months, especially
Garth. Soap and chocolate was soon emptied from my pack. Nona awas
looking rather tired and drawn having had quite a spell of nursing
Cythia an Garth through whooping cough and ear problems and was
grateful for the little help I could give het. Eight days went like
lightening and all too soon I was on the return journey to France. At
Victoria station I sorted out the code number of my leave group and
got onto the appropiate train which took us to Folkestone. It was
pitch dark when we arrived there and drizzling with rain. We were
taken to requisitioned hotels on the sea front, cold miserable billets
where the bare necessities were provided which didn't help to relieve
my after leave miseries. After a breakfast of bangers and beans we
embarked for Calais, where I spent the night before joingin my group
to return to Bayeax on Jan 21st.

Austrian P.O.W.s with very strong anti-Nazi views were in a special
compund close to our camp and parties of them worked in the
hospital. The German P.O.W.s who had earlier helped with the
construction work no longercame here. These Austrians were a well
disciplined group and secureity form us whas a formality. After thei
days woek one of us was detailed to escort them back to the compound,
a detail I had on one or two occasions. Odd bits of wood were
collected for their fires but they always asked permision before
taking it, their code of discipline amoung themselves was very
strict. One evening, while escorting a small party across the fields,
to their compound, a prisoner asked me if he could get a small piece
of wood the other side of the hedge. They all spoke pretty good
English. I said he could before I had seen a Frenchman in the field
who came running across yelling Dirty Boche Dirty Boche brandishing a
stick with which I thought he would strike the prisoner. I thought
`Blimey, I am going to have a situation here, but to my relief the
Austrian, twice as big as the Frenchman, ignored him and quietly
returned to the party leaving the wood on the ground. The Froggie
followed us along the hedge like a maniac and I felt like thumping him
myself. I was impressed by the Austrians' camp's clean liners and
orderlyness. Tents were in a perfect line and evenly spaced. Pathways
marked by short white painted stakes wiht rp=opes attached to them and
on the small patch of ground in front of each ten regimental badges
had been marked with a mosaic of colouted stons. Noting my curiosity I
was invited to look in some of the tents. The floors had been dug down
deep enough for these tall men, not one was undere six feet, could
stand comfortably upright. Ingenious stoves fasioned from ration tins
provided some warmth - the reasson they collected oddments of
wood. Smoke from the fires was conducted through the earth wall at the
rear end of the tent by pipts made with round ration tins joined
together. There was no guard on the camp. They had their own men at
the gate to check parties in and out.

On Sunday mornigs it was practice to give blankes a good ahek and if
weather permitted to air them ouside, also we attended to what we
called housekeeping. I no longer slept on the floor. From ration boxes
and a few loos boards I had the making of a bed, in fact most of us
had acquired a bed of sorts. On this Sunday morning I was sitting on
my bed sewing on a few buttons, and endless job for without elasticity
army braces buttons were forever pullin off. A voice called in from
the hut door, `Is Fred Lawrence at home?'. I looked around and their
was Bill Shirley, my next door neighbour from Kenilworth. His wife had
written to tell him I had been home on leave and that I was in the
Baywax area. Bill who was a despatch rider in the R.A.S.C. had brough
some dispatches to Bayeax and had called at the NAAFI for
refreshments. Seeing a corporal of the R.E.s standing at the counter
he asked by any chance if he knew a Fred Lawrence. To his surprise
\corporal `Wimpey' said yes I do, there is a Fred Lawrence in my
platoon of R.E.s and we are in huts a little way along the road. Bill
and I had a lot to talk about, we hadn't seen each other since call up
in 1942 and having been home on leave I was able to fil him in with
home news.

The winter of 1945 was quite a hard one, lots of heavy snow falls and
periods of severe frost. Thank goodness we were in huts and able to
have fires for warmth. During one particular hard spell of frost our
water supply froze up and we were unable to wash or shave for several
days. The cook-house and hostpital had water brought in by carrierd adn
our crafty sergeants scrounger water from the cook-house for their
shaves. Segeant Wicox all neatly shaved said to me one morning `You
haven't shaved Lawrence' and laughte dwhen I said `No sarge, I
couldn't find my three stripes to pass me into the cook-house for
shaving water.'

Many roads in Normandy were lowere than the surrounding fields and now
wiht so much of the land drainage system ruinded by war, these roads
became water logged and churned into deep, muddy tracks by military
vehicles. Aftern they were impassable for ordinary traffic. We were
sent out on various missions to try and improve drainage, work we
didn't enjoy. On of our impossible tasks was at an ordinance dump. In
getting out the stores, much of it in heavy packing cases and loaded
onto vehicles for transportation to dumps nearer the Fronts, the
ground was turned into a muddy morass, three feet deep in
places. Having listened to stories of World War I when men and horses
were drowned in mud I could now believe that those stories were
true. Segeant Webster and \corporal Wimpey with a party of fifteen
sappers were sent here on a ditech digging execise to try and ease the
water problem. The land was so evenly contoured that efforts to
improve drainage appeared to be hopless and eventually the remaining
stores were abandoned until conditions for their removal became more
favourable. I rather think that the amphibious vehicle drivers
regretted the decision for they seemed to enjoy driving at speed
through the muck, towing strong sledges loaded with stores and sending
out waves of mud behind them. Too bad if you happened to be in the
path of these waves.

Claso to the dump there was an abandoned farm house, partly surrounded
by a moat and we were using one of the buildings to store our tools
and eat our midday haversack rations. It was avout the nearest point
to the dump theat our truck could get in safely. On the moat a home
made punt floated and each midday break we had a play with it. The
punt seated two and because of the moats muddy banks, the second
passenger had to push it off. I was pushing with Webby seated in the
punt, my feet stuck in the mud, I lost my balance tipping Webby into
the moad and falling face down into the water myself. A fire built
from a few pieces of wood didn't produce enough heat to dry any
clothes or keep us warm and our transport to take us back to camp was
more than welcom that afternoon.

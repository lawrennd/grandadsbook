At the beginning of the war I was working for a Coventry builder
erecting houses on the Styvechal Estate. Materials for this
speculative house building had become harder to get during the 1938
European political crisis and when war was declared in 1939 this
trickle almost dried up so I began to look for other work. Carpenters
were in great demand by factories which now had to black-out all
windows and roof lights and take other air raid precautions and I
obtained a job with Alvis Limited in Holyhead Road.

With the introduction of conscription for men between the ages of
eighteen and forty-one years, I had my medical and due to varicose
veins in both legs was classified B.II. Although working in a
munitions factory I was not engaged in a reserved occupation and was
liable for call-up at any time or could have been directed to work on
Military Camps being built all over the country.

In February, 1941, I caught pneumonia, was very ill and away from work
for almost four months. Early in 1942 I had pneumonia again but less
severely this time. Some weeks after this illness I received a
directive ordering me to report for camp construction. With the
thought of approaching winter and working in all weathers on a muddy
campsite, I asked my doctor if he considered me fit for this work. He
told me to appeal against the move on medical grounds and gave me a
note to enclose with my appeal. I had to appear before a Ministry
medical adjudicator who upheld my doctor's view and recommended that I
continue in my present employment.

Nothing more was heard about working on campsites, but to our surprise
my calling-up papers arrived in November ordering me to report to
Maryhill Barracks in Glasgow for six weeks `Initial Training'. Nona, my
wife, was now expecting our second child and was having a difficult
pregnancy, needing constant attention from our doctor to prevent a
miscarriage. Because of Nona's condition I applied for a
compassionate deferment of call-up until after our baby was born; this
was refused. Our doctor, in an attempt to cheer us up, said he
thought I would be discharged as medically unfit in less than six
months. How wrong he proved to be.

The telegram informing me that my application for deferment had been
turned down arrived on the day I should have reported to Maryhill
Barracks. Having said our goodbyes I cycled down to my parents'
house, left my cycle with them and caught an early train for
Birmingham where I changed to one for Glasgow. Ordinary service trains
suffered long delays, priority being given to trains carrying
essential supplies or troops. There were no buffet coaches so I had
come prepared with sandwiches. Wearing my Home Guard uniform, now
stripped of flashes, I was able to queue with service personnel at the
W.V.S. trolleys for cups of tea whenever the train stopped at a
station. Some of the stops were long and the journey to Glasgow
seemed endless. It was after lights-out when I reported to the
barracks' guard room and was taken to a nearby hut to spend the night.
There were no beds in the hut and no black-out shutters at the windows
so I was unable to switch on a light. The hut was obviously a bedding
store for there were heaps of straw and palliasse covers on the floor.
With the aid of my torch I prepared a make-shift bed and tried to
sleep. I was tired after the long, boring train journey but sleep
wouldn't come. My brain was too active worrying about Nona and
wondering if I would be punished for reporting late.

The hours passed slowly away until about 05.00 hours when one of the
guards came for me. At the guard room I had a mug full of tea,
washed, shaved and made myself presentable to be taken to the orderly
officer who then detailed an orderly to show me to my hut.

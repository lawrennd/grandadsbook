Jack Thompson from No.~3 platoon was about two years older than
me. Although we had met in the company and knew each other by sight we
had never been on a project together. Like me he was married, had two
children, was a carpenter in civvy street and enjoyed a quiet
life. His home was in Hastings and we became great friends at
Soest. We were both a bit apprehensive about this posting to a
D.C.R.E. unit and wondered what our role would be in Soest.

D.C.R.W. H.Q. was in a quiet suburban type of rad on the outskirts of
the towm, lined on each side with Lime trees and a few hnouses. Our
billers were in one of these houses opposite to the offices. There
were fourteen sappers and N.C.O.s in the billet adn Jack and I shared
a small bedroom on the first floor. The other N.C.O.s in the billet
were \corporal Manton and \lcorporal Garrat who shared a room opposite to
ours and were both demobbed a few weeks after we joined the unit. The
other sappers shared rooms in the attic, on the first floor and in the
basement. The ground floor was our mess room.

Three women were employed to cook our meals and keep the billet clean,
working in shifts to cover the house between 06:30-18:00 hrs.

Our C.O. was \captain Scott, a short, tubby, little man who liked his
drink. He wasn't a strict disciplinarian but was full of weird and
exciting ideas. The Office staff were a mixture of civilians and Royal
Engineer personnel supervised by \sergeant Atkins. There was a transport
pool in a warehouse behind the offices and all the drivers were
civilians with a civilian mechanic to maintain the vehicles and act as
garage foreman.

Fifty yards or so along the road from the office there was a complex
of three storey buildings built round a large ashphalt square. These
barrack like buildings seemed to sprout up all over the place and
were, I believe military schools for Hitler youth. The buildings ere
partly occupied by Slavic D.P.s and Polish ex P.O.W.s. Other houses in
the road, all part of this barracks complex were over crowded with
German civilians.

Jack was detailed to assist \sergeant Feaney and tehy spent most of the
day away from the office A large room on the first floor, furnished
with several desks and bookcases was a general office were civilian
supervisors and staff sergeants gathered to make out their reports and
one desk was allocated to me. My first job was to ckeck why a small
Royal Artillery unit in our area had indented for a large number of
electric light bulbs. A P.U. from the pool with a civy driver drove me
to the uit and after counting the number of light fittings I reported
that I thought the number was excessive and recommended that the
quantity asked for should be reduced. One of D.C.R.E.s functions was
to sanction and control the materials indented for by military units
in our zone and I found that one needed to be vigilant about black
marketeering. Although I had done some estimating of work and
materials in the building trade I felt a bit overwhelmed when I was
detailed to get out an estimate to repair and redecorate the
D.P. centre.

There was a very good library of Army manuals and directives in the
book cases in our office and the knowledge I obtained from the carried
me through the estimating detail and many other unfamiliar projects.

To do my estimating I needed to look in all the rooms occupied by the
D.P.s and often found it difficult to explain my presence there. The
D.P.s viewed me with suspicion, suspecting I was spying on therm, in
the same manner that their former Nazi and Gestapo masters had
done. Some of the Polish P.O.W.s understood a bit of English and wer
quite useful as interpreters. All these people were afraid to return
to their now Russian occupied home lands for fear of reprisals. It is
sad to know that hundreds of these D.P.s were shot when they returned
home having been condemned as collaborators. There were some families
with young children living in these rooms and it was pathetic to see
how they all coped in one room. Cords tied to nails driven into the
walls had old sheets or blankets draped over them to get little areas
of privacy. The D.P.S had a kind of hopeless expression on their faces
and shoed no signs of relief of being freed freom their forced
labour. The Polish ex P.O.W.s were different, happy to ve free and
hopeful for the future and maintained a high standard of cleaniliness
in their quarters.

There didn't appear to be any urgencey avout this estimating for I was
often sent out on other middions and I had some peculiar sncks from
hospitable German material suppliers. The Schnapps they gave me was
quite drinkable after I got used to its bite. The Ertzat coffee was
quite unpallatable and the black bread with raw picked bacon or raw
pickled herring I found hard to chew and swallow.

Soest was a quiet, little, fortified, medieval town and parts of the
old thick walls were some feet above ground and were wide enough to be
made into walks. One of the gateways which arched over the footpath of
a road near our canteen was three storeys high and in a very in a very
good state of presevation.

Not having any major industry or being of military importance the town
had escaped war damage, but one nhad only to travel a short distance
to see the havov vaused by our bombing raids. Hamm, Dortmund,
Dusseldorf and Wuppertal wree all within a forty or fifty mile radius
and these places had been heavily bombed. There were not many
B.B.C. bulletins that hadn't refered to raids on these cities and
towns.

There were many half timbered buildings in the centre of Soest and one
of these, a restaurant called `Der Wilden Mann', was now a forces
canteen where Jack and I spent many tea drinking hours. No alcohol was
served here so it was a relaxing place to visit. The interior was full
of old black oak beams and panelling and the dark oak furniture of
tavles and chairs sparkled from years of polishing. A German violinist
was allowed to play here and he wandered through the tables playing
pieces of better known light classical music. For a few cigarettes he
would play requested tunes. If he wasn't familiar with the tune asked
for one only had to hum or whistle a few notes and he could play what
you had requested.

After the ruins of Calais, Aachen and Euskirchen it was nice to wander
round a town that had not been battered by bomb or shell and to
temporarily forget about the ravages of war. There was a football
ground on the edge of the town and not many days passed without a game
of some sort being played there. Teams of civilians and teams from the
various military units played against each other.

German beer gardens were out of bounds to sevice personnel exvept for
certain nights of the week, to give the civilian population a feeling
of not being always under surveilance and help them towards a more
normal way of life. Jack and I went to one beer garden near the
canteen a couple of times, but found it too rowdy for our liking. The
noise of the Jerry drinkers banging their tankars on the tables while
they sang was deafening. The beer they served tasted like soft water
and how they managed to get merry on it I couldn't understand. The
only effect it had on mye was to over work my bladder. We were talked
into these visits by two sappers in the billet. \Sapper Harrison had an
idea that he could sing and was forever torturing our ear drums in the
billets and he, after a few beers sang at the beer garden which seemed
to please the customers. His pal, and ex 8th Army sapper called
Hobbins, did a presentable sand dance act and he too performed at
these beery sessions.

Because there was no alcoholic drink in the canteen we had a NAAFI
ration of beer in the billets. Every Friday our ration truck brought
in a nine gallon barrel of this tasteless German beer and most Friday
nights turned into a sing song with Harrison and Hobbins doing their
acts. Harrison could get very bad tempered if he saw us sniggering
about his singing. By midday Saturday what was left in the barrel was
poured away, it became so flat it was undrinkable.

\Captain Scott was tipped off that a German architect was coming to the
labour bureau in Soest to seek work and Scotty wanted to get him in
the D.C.R.E. staff. He called me into his office, gave me a note with
the man's unpronounceable name written on it and told me to get him
before he was put onto other work. With help from the desk corporal at
the bureau I located Fritz, my name for him, waiting with others to be
interviewed by the officer in charge. Telling Fritz to follow me I
jumped the queue, took him into the office and showed the officer
\captain Scott's note. Fritz was given some papers to sign and a work
card and he became a member of D.C.R.E.s civilian staff. From then on
he became an embaresment to me for he thought it was I who had got him
this job.

The present plans for housing the occupation forces, which had changed
constantlky, were such that two small detachments were intended to
occupy the D.P. centre and each one was to have its own independant
kitchen. \Captain Scott detailed Fritz to make drawings for the
conversion of part of the ground flor in on of the blocks of buildings
at the D.P. centre into a kitchen with an estimated numver of men it
should cater for. If Fritz had any queries, he came to me, and I often
needed to refer to our library of manuals. I told him how many ranges
he would have to put in to meet the catering corps standard
requirements. Afterwards when Fritz told me he had planned for extra
ranges, I told him he would be in trouble and asked him why he did
it. He said: `My reason, corporal, is that soon Germany and England will
be joining forces to fight the Russians, more men will be billetted
there and the necessary ranges will be installed to accomodate them.'
What Scotty thought about that is anybody's guess. The work was not
started while I was in Soest.

This idea of Britain and Germany joining forces to fight the Russians
was very prevalent in the area. Bruer, an ex-school master now
employed as office interpreter felt quite strongly on the subject. He
also explained to me that although he was anti-Nazi he had to teach
the Nazi doctrine in school for his family's sake. Failing to obey
would most certainly have meant his disappearance into a concentration
camp.

A railway line passed in a big curve behind the offices and \captain
Scott thought it might be feasible to take a branch line from the main
track and make a siding by the large building we were using for a
garage. This he thought could be made into a stores warehouse for the
occupation troop who would be in the area. I almst panicked when he
asked me to look into the project. I had as much idea about railway
construction as I had of flying. Again the library of manuals was
consulted especially those of the Royal Engineer railway
companies. 0ne of the German drivers, seeing my interest in the
railways, proudly told me that before the war he had been a plate
layer on this section of the line. With Braer to interpret for me I
questioned this ex-plate layer who confirmed my opinion that because
of the gradient and the curve of the track a branch line was not
practicable. \Captain Scott was rather disappointed when I reported
back to him.

The only night duty we had was to man the telephone switch
board. During the day a German girl was on duty there, she spoke very
good English and was a keen Shakespearian. Knowing that I lived in
Shakespeare country she was amazed that I had little interest in his
plays. The night watch from 12:00-08:00 hrs at the switch board was a
boring duty. There was a couch in the room for us to lie on and a fair
amount of sleep was possible during the night. All calls came via the
town exchange operated by the Royal Signals and we arranged with the
duty operator there to ringh long and loud if any calls for the
D.C.R.E. came in and also to give us a rousing call at 06:00 hrs. One
of the operators became quite friendly over the phone and we had long
chats together between his needs to attend to other calls. sometimes
he used to say `Hang on, keep quiet and listen to this'. He knew his
callers and we would listen to officers dating their girl-friends in
the nursing, A.T.S. and NAAFI services. From his voice I formed an
idea of what their operator would look like if I met him and what a
surprise I had when we were able to meet in the Wilden Mann
canteen. He was much shorter and slimmer in real life than the image I
had built up round his deep strong voice.

When I was sent to a small Belgian unit to deliver a kitchen range
and explain how the parts fitted together and how to install it. I
felt quite ill on the return jouney to Soest. It was a warm day and I
thought my dizziness and sick feeling was due to a combination of
warmt and diesel fumes in the cab. I couldn't face my evening meal
when I got back to billets and went straight to bed. Jack fed me with
asprins and after a semi-delirious restless night I reported sick. I
was so dizzy that I could hardly walk straight and Jack rode with me
to the R.A.P. in town. The M.O. gave me a quick examination and
ordered me to bed in the sick bay. I couldn't eat any of the food
brought to me by the orderlies but drank large quantities of
water. The following morning I was put on a stretcher and loaded into
an ambulance to have one of teh most frightening rides of my
life. Walking sick sat on the lower stretcher racks, I was put on the
top on. The ambulance man seemed to be travelling too fast and when it
keeled over at corners I was afraid of rolling off the strechter and
scared of bouncing off it each time it hit a pot hole. That
twenty-five mile ride to No.~6 British General Hospital in Isolohn was
quite nerve racking.

Pneumonia was diagnosed at the hospital and I ended up in a small six
bed ward on the first floor. Why I had pneumonia when I was living in
the most comfortable conditions of my service career puzzled
me. Apparently there was an upsurge of pneumonia cased at this time
and the Medical people were assuming that it was a delayed action type
of illness from the earlier days of unfacourable living
conditions. For a few says I was well stuffed with drugs and not even
allowed to wash myself. When I stopped deeling sick and able to eat,
all sorts of luxurious dishes came my way such as fresh eggs, chicken
and jellies. For elevenses I had egg beaten in milk and sherry, a
bottle of Guiness with my dinner and a pint of champagne at night. The
champagne stopped when my temperature became normal. By this time I
was allowed to get up and dress in the awful ill fitting hostpital
blues and as my legs got stronger I helped in the ward kitchen where
the special light food were prepared. Besides our little annex there
were two single bedrooms and a ward extending the whole length of the
building. Walking patients helped to serve out the meals and wash up
in the kitchen. As a reward fo r this help there were many extra bits
of food to eat.

Except for Nobby Clark, I didn't get to know much about the occupants
of the other four beds in my room for they rarely stayed more than a
week. Poor old Nobby had some vomplaind which in the early days of his
stay here required injections in his buttocks. One of the sisters mush
have been a dart player. At the sight of the pan with the hyperdermic
laying in ti Nobby used to groan, roll over onto his stomach and bare
his bottom. This dart throwing sister would take the needle end and
jab it into Nobby's buttock. Each time I expected to see Nobby fly
through the window giving a yelp of pain. He told me that he never
delt the needle but was getting quite sore after the frequent jabs.

The R.S.M. of the medical unit in Soest came in for a few days to have
some check ups and was put in the single room next to our ward. We
knew each other from visits I had made to the uit when checking
material indents and he took me in to parts of the hospital grounds
that were out of bounds to ranks below W.O. and one of our favourite
visits was to the stables where some fine horses belonging to officers
of the hospital were stabled.

A few passes were issued to go beyond the hospital gates and having
discovered from on of the auxiliaries that there were some nice walks
round the lake I applied for one each day it was fine enough for a
stroll. The other reason for going out was to bys some lovely
donughnuts sold at a canteen outside teh hospital. It was against
hospital rules to buy food outside and bring into the hospital but I
always brought some to share round the room. We were getting good food
and a generous amount of it but these smuggled doughnuts had their own
particular appeal. My oversize badly fitting hospital blue coat made
easy cover for a bag of doughnuts. Rowing boats were available to the
hospital staff and one very robust auxilliary nurse on our ward was
often seen taking a patient for a trip on the lake. She was most
concerned about our welfare in the hospital. When I first began to get
around she roped me in to take communion in the chapel on the ground
floor and literally carried me up and down the steps.

A ward sister noting my deafness arranged for me to see the
E.N.T. specialist, a crusty old colonel. After probing in my earas and
asking questions he produced some lengths of rubber tubing and a
rubber bulb like those that were fitted to old car horns. The tube
with the bulb on one end was plugged into my ear, then a tube with an
ugly looking curved thing called I believe a catheter needle was
inserted up my nose and the loose end inserted in the colonel's
ear. He then told me that when he said swallow I was to swallow
hard. I heard him say swallow, I did so and at the same time of
swallowing he squeexed the bulb hard sending a stream of air through
my sinuses. I thought the topof my head was going off. This was
repeated on my other ear, then while my head was reeling he asked if I
could hear any better. I wasn't sure that I could hear at all after
that onslaught and he got very snappy with me. I then made a mistake
by saying `No sir, my hearing is no better.' and he repeated the
blowing out process. I had now had enough and told him my hearing was
better when in fact it seemed worse than before. Yards of narrow, oil
soaked fine gauze were then poked in each ear, so muxh of it that I
though my head must be hollow. Going down to the surgery the next
moring to have this gauze removed I met two others who had received
the same treatment and felt as I did that no benefit had been achieved
from this blowing out.

Taking a walk in the hospital grounds again I met \sapper King
again. Kingy, having had stomach problems had come in to see a
specialist and was only expecting to be here for the day. The
M.O. thought I was now fit to return to D.C.R.E. and when making out
my discarge papers he told me that he had been surprised about my slow
recovery and was recommending that I had immediate sick leave.

Once again dressed in khaki I set out to thumb my way to Soest. Two
M.P.s in a P.U. pulled up and asked to see my papers. After finding
they were in order they brightened up my day by saying Jump in, we are
going into Soest.

Sick leave was granted by D.C.R.E. and the day after my discarge form
hospital I was on my way to the Hook of Hollan, that long tiring
overland journey plus the sea voyage to Harwich found me less fit than
I thought I was. towards the end of my leave, not feeling too well I
went to see our family doctor who, after an examination gave me a note
to take to an Army Office in Leamington and I was given an extension
to my leave.

It was almost the middle of June when I arrived back at Soest to find
the members of the unit has changed considerably. Except for the
civilan staff, I was in strang company. Jack and many others had
either been demobbed or posted and now there were only six of us in
the billet. \Captain Scott and the office sergeant were amoung those
who had been demobbed and in their place we had a \captain Howard as
C.O. and regular soldier sergeant Brandon who was in charge of the
office staff. \Sergeant Feaney was under house arrest awaiting court
marshall for misuising military transport. He had gone out on a joy
ride in a truck with a civilian driver and they had been involved in
an accident n which the driver had been killed. At the court marshall
Feaney was demoted to corporal and posted from the unit.

The C.O. sent for me the day after my return home from leave to tell
me that I was demoted. This demotion usually happened to lance
corporals and lance-sergeants when they were away from their units for
long spells. The evening was spent cutting off my stripes and the bits
of cotton from my sleefes and the next day I had difficulty in
explaining to the civilians that my reversion to sapper was not for
misconduct. It was even more bewildering to them and I suppose to me
when I read on orders a day later that I had been promoted to lance
corporal again and had to sew my stripes back on my sleeves.

The D.P. centre was now empty and deserted and I was to spend the rest
of my long lonely days while waiting for demob wondering through
these buildings making what appeared to me an estimate of repairs and
redecorating that would never be used. I had no other jobs to break
the monotony. Fritz was engaged on other prokects, his prokect of the
kitchen at the centre had been shelved and his previous eagerness to
be friendly which had been an embarresment was now welcomed to liven
my days. I missed Jack quite a lot and was unable to accept others in
the billets as mates to go into town. Jack corresponded regularly and
invited me to visit himin Hastings when I returned to England. Bruer,
the interpreter, was cultivating part of the garden behind the billets
and I spent many hours taking with him getting to know something about
life under the Nazis. Seed were had to get and he was experimenting
with such things as dried peas in an attempt to grow food. Potatoes
were peeled a little thicker by the eyes and grown in containers in
the house befor planting outside.

Hundreds of fire flies attracted by the sweet sticky substance of the
lime trees fascinated me in the twilight as they swarmed around the
trees in luminous clouds.

At the centre I found that if I stood still in the attic mice would
emerge from their hiding places and play over my boots. Without
gaiters I'm sure I would not have done this because without a doubt
they would have climbed inside my trouser leg and I watched the
adventurous ones climbing on the outside of my trousers keeping them
below knee level. It only needed a small movement to send them
scuttling to safety.

The date for the release of group 37c was announced, I made a
calendar from a piece of card and like a prisoner ticking off the days
of his sentence I crossed off each day and what long days they
were. Most of my gear was handed into stores, leaving me with little
more than I had when I reported to Maryhill Barracks to begin training
four years ago.

The journey to The Hook of Holland was a happier one this time. Each
mile was one more mile to civilian life again. On the boat taking us
to Hawich I once more met Kingy. How odd that we should have bumped
nto each other so often and in differing circumstances. Kingy was
going on leaving this time and during the crossing we had many chats
together, our R.E. training at Fulwood being the main topic. How many
of that group we wondered had been as as lucky to survive as we had.
 
From Harwich, men of the group 37c, travelled by train to Dover. At
the demomobilisation centre here we were formed into alphabetical
groups and, after a hot meal, began to queue up the last rites of
demob. Passing in single file from one point to another we turned our
various remaining items of kit into segregated piles. Knives in one
pile, forks into another, mess tins on another heap and so on until
all we had were the clothes we were wearing, a kit bag, personal
cleaning item such as tooth brush, razor, towel and a change of
underwear. The pieces of kit were thrown onto the piles with much
emphasis and loving! We came to a gigantic menswear department to
choose our civvy clothes. The colour choice was small. I chose a pin
striped gray suit a gray mac and trilby and black shoes. The fit of
the suit was reasonably good so I poked all the khaki gear into my kit
bag. Papers signed, pay given with travel warrant and ration cards
collected. I walked out of the centre not Lawrence 14340452, but Mr
F.~H.~Lawrence ready to meet the challange of settling into civilian
life again which didn't came easy.


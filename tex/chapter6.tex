Shrugging my gear into a more comfortable position I set off along the
road towards Croft. There were cottages on either side of the road
but as far as I could see, and that was about three hundred yards
before the road curved, there were no houses large enough to accommodate
a platoon let alone a company. There was a surprise waiting for me.
After walking some two hundred yards I came to a fairly large
stone-built house, set back from the road with a boundary wall in front
of it. A Divisional flag was fluttering from a pole outside the house
and standing at a gateway leading to some outbuildings and the grounds
of the house, was a sergeant of the Royal Engineers. Hopefully I
asked him if this was H.Q. of 278 Field Company and to my delight he
replied that it was. After explaining that I had been posted to the
company he introduced himself as \sergeant Greenway, took the large
manila envelope containing my documents, grabbed my kit bag and took
me to company office. Here I met \lieutenant Taylor who scanned my
documents and asked me a few questions such as when I had last been on
leave, what my family life was like and if I had any problems with
which he could help me. He then told me that I would be in No.~2
Section of No.~2 Platoon which he commanded and that \sergeant
Greenway was my platoon sergeant. When I was dismissed \sergeant
Greenway helped me with my kit and took me to No.~2 Section hut
calling at the cookhouse on the way for a welcome mug full of tea.
There were a few empty two tiered bunks in the hut and I dropped my
kit onto an unoccupied top one. The sergeant told me where to find
company stores for my blankets and after sorting and laying out my kit
to change into denims and report to him. Company Stores was in a
small chapel building a bit further down the road and from here I drew
my summertime issue of two blankets. Until the section came in for
dinner \sergeant Greenway gave me a few chores to do round the
cookhouse and dining hut and after dinner I was to join the section.

\Lieutenant Taylor gave me the impression of being a decent officer and
\sergeant Greenway seemed OK. Now I wondered what the rest of the
platoon was like. \Corporal Murphy came in for dinner with the section
and I was quickly made to feel that I belonged with them. First
impressions are often a good sign as to what people are like. Murphy,
as his name suggests, was an Irishman. He was section leader and a
good one he proved to be, tough as leather and a demon for work.

Company office, platoon office, officers' and sergeants' messes were
in the house and on what had been the lawns and gardens of the house,
wooden huts had been erected for Other ranks' quarters, dining room
and a NAAFI canteen, run by ladies from the village. The cookhouse
was in what had been the coach-house and stables and our latrines and
ablutions were in other outbuildings of the house. From the huts, a
long winding footpath through a spinney and shrubbery led to open
fields, the path crossing the river Tees which cut across the grounds,
by means of a rustic wooden bridge. One of the fields at the side of
the river which had now curled back on itself was our bridging hard,
where we practised bridge building. Two huts were built on the hard.
One contained a miscellany of stores and in the other one explosives
and ammunition was stored.

The company was split into two parts. H.Q. platoon and No.~2 platoon
were billeted here at the house and Nos.~1 and 3 platoons and our
transport were in buildings nearer to Croft. At this time \captain
White was acting C.O. and our company sergeant major was \sergeantM
Duncan a strict disciplinarian; short in stature with a bark much
louder than his height gave reason to believe he had. The other
N.C.O.s of the platoon were \lsergeant Cawfield, \corporals Wilkinson
and Jarvis and \lcorporals Diaplo, Jamieson and Whittaker. \Lsergeant
Cawfield, for some unknown reason, tended to irritate me and often
brought me close to a situation of insubordination when addressing me.
I had a feeling that he was a `creeper'. \Corporal Wilkinson, a
regular, lived in the London area. He lowered his popularity when he
brought his saxophone back with him after one of his leaves home.
Wilkie was a beginner on the instrument and his scale practice wasn't
appreciated. \Corporal Jack Jarvis was a gaunt West Country individual
full of chatter but had a terrible habit of biting his nails, he
gnawed them until he drew blood. \Lcorporal Diaplo was a short, lively
fellow, well liked by the section and full of good ideas.  Jamieson
was a Scotsman, most difficult to follow when he was talking for he
not only had a broad Scottish accent, he also spoke through clenched
teeth. Horace Whittaker was a Brummie like myself.  He lived at Water
Orton near Coleshill. He was our platoon storekeeper and was quite
handy with comb and scissors for haircuts which cost sixpence. 278
Field Company was one of three companies forming the Divisional
Engineers of the 15th Scottish Division and our C.R.E. was \lcolonel
Miller. The 15th was a Territorial Division formed in the early days
of the war, still understaffed and under-equipped and until recently
had been part of the southern defense system. The shoulder flash of
the division was a red Scottish rampart lion inside a white circle
which represented the letter `O', the fifteenth letter of the
alphabet, all on a blue background.

A wide piece of ground between the house boundary wall and the road
was our parade ground and our parades, especially defaulters'
parades, were a source of entertainment for the young villagers.
They enjoyed watching the orderly sergeant of the day putting the
defaulters through an hour of lively drill.

There was a small workshop in the basement of the house and I was
often detailed to work down there with `Dippy', \lcorporal Diaplo,
who was also a carpenter. Very secretly a baby's cot was being made
down there. \Lieutenant Taylor's wife was expecting her first baby and
cots were not easy to come by. In fact it had been heard on the
grapevine that this was a major problem for \lieutenant Taylor. Unknown to him
the platoon carpenters were making a cot from materials scrounged or
bought out of funds contributed by members of the platoon. The
construction had been well thought out and it was being built so that
\lieutenant Taylor would not have a transport problem but would find it easy
to assemble. When it was finished \sergeant Greenway brought the
\lieutenant into the basement. He was astounded when he saw the cot and
mystified as to how we had been able to make it under his nose without
him knowing. When he had the platoon together he thanked everyone on
behalf of his wife for the gift and added that he would be more
vigilant about our movements in the future. He was a very popular
officer and we were all sorry to lose him when he was later promoted
and posted to an airfield construction company.

To my surprise I was granted another privilege leave in April and on
this leave Garth was christened at St. Nicholas' Church. We
had a family photograph taken at a studio in Leamington and I carried
one of these photographs with me for the rest of my service in the
Army. It finished up looking a bit tatty from the rough treatment it
received in my pocket. I think it was on this leave that I did a
perambulator conversion job. The wartime model we had bought for
Garth had a rather flimsy wheel and springing construction and it was
quite hard for Nona to push. In fact it had developed a lean to one
side and each time it was eased down the kerbs one expected
it to collapse. I had a chassis from a pre-war perambulator onto
which I was able to fix the body of our newer perambulator. After a
bit of paintwork Nona had a sturdier and easier perambulator to push
on her walks to Kenilworth.

On the hard there was a mixture of Folding Boat and Bailey bridge
parts and besides practising the drills on the conventional methods of
bridge building and rafting from these parts, \lieutenant Taylor tried
a few ideas of his own. Rafting as we then knew it was normally done
with the F.B. equipment which was capable of taking loads such as a
Jeep towing an anti-tank gun. \Lieutenant Taylor thought a raft built
from Bailey bridge parts and floated on Folding Boats could take
heavier loads. We built the raft for him and it appeared to float
nicely on the water. A loaded three ton truck was then driven onto it
and again everything seem all right until we began to pull it across
the river. Then calamity struck. The weight of the Bailey bridge parts
together with the loaded truck were too much for the canvas and wooden
structure of the Folding Boats. In midstream the boats collapsed and
our raft with the truck toppled into the river.  Fortunately for us
the river here was only a few feet deep and not in flood so salvaging
was not too difficult.  After heavy storms in the hills and moors of
Durham and North Yorkshire the Tees rose very quickly and developed an
ugly mood. This was one experiment that went wrong but there were
other improvisations that were a success. Rafts using Bailey parts
built on pontoons made of Marine plywood were later used to ferry
tanks across rivers.

A room above the coach-house was used for a lecture room and, having
assembled in there one morning for what we thought was going to be
another lecture about some new mine or piece of equipment, we had
quite an awakening. We were told that 15th Scottish Division had been
chosen to take an important part in an assault against the enemy. When
or where was not revealed but from now on there was going to be a
vigorous period of training beginning at section level and progressing
up to Divisional level. He also told us that this information was to
be treated with the greatest secrecy and in future not to discuss
outside the camp where we were going on our exercises. Punishment for
careless talk was severe.

A check on our physical fitness was the first exercise in the training
programme. For this the platoons were required to make a ten mile
forced march, have a limited number of men falling out on the way and
at the end of the march the platoon was to be strong enough to take up
a defensive position which would be satisfactory to the umpires. Two
hours was the time allowed for this exercise and it was to be repeated
until platoons achieved this standard. \Lieutenant Taylor said he hoped
we would do the test on our first attempt and not have to waste time
and energy on repeats. We were dismissed to our daily tasks and went
away making wild guesses as to where we were going. There were plenty
of rumours floating around about the opening of a second front.

Dressed in Battle Order with fifty rounds of ammunition in our
pouches, weighing altogether about fifty pounds, our transport took us
to the starting line. Sections were formed up in staggered formation
on either side of the road ready to begin section leapfrogging. On
the word `Go' from an umpire we moved off with No.~1 section in the
lead marching at quick time while Nos.~2 and 3 doubled up until No.~2
section were in the lead who dropped to quick time. Now Nos.~3 and 1
were doubling up, bringing No.~3 into the lead when they dropped to
quick time. This procedure of leapfrogging over sections was carried
on for the rest of the march.  We had one command of `Enemy aircraft,
take cover!' when we dived into the nearest ditch. Turns were taken to
carry the section Bren gun and its two boxes of ammunition. The Bren's
weight of twenty pounds on top of the fifty pounds I already had felt
like a ton when it was my turn to handle it, especially towards the
end of the march. It wasn't an easy weapon to carry If you carried it
by the handle it swivelled and bumped on your thighs. If you put it on
your shoulder it bounced up and down as you jogged along. The
ammunition boxes were easier to carry. Having two of these, one in
each hand, they were a more balanced load and easier to manage than
the Bren. The man next to you carried your rifle during this
period. The pace began to tell after five or six miles and the odd
sapper was forced to give in. \Sapper Harris was the only one of our
section who didn't finish the course. My legs were now moving
mechanically, my chest was beginning to hurt and my breathing became
more difficult. I was gulping in air and gasping it out like on old
pair of bellows being pumped at a fast rate but somehow I managed to
keep up with the section. The periods of quick marching seemed to get
shorter as the march progressed but they were a welcome respite to try
and get breathing back to normal. We finished between Croft and
Hurworth where we set up our defensive position which satisfied the
umpires. Not too many men of the platoon had dropped out of the march
so we were declared strong enough to hold our position. when we were
told that we had completed the exercise within the time limit we found
enough wind to give a rousing cheer and the news put life into our
legs to march back to camp. I gave Darlington a miss that night. Tea
and waddies in our NAAFI hut and a smoke and read on my bed were
enough for me and I wasn't alone in this idea of spending the night.

After the forced march we had exercises to polish up our drills on
mine laying, mine clearing and the setting and neutralizing of booby
traps. We also had two exercises on the River Skerne, a tributary of
the River Tees, to practise the use of kapok footbridging and how to
get into a two-man rubber dinghy without upsetting it. Kapok
footbridging was designed for the use of assaulting infantry units
really but it was thought to be useful for us to get some idea of how
to use it. The equipment was a series of kapok pillows strung across
the river rather like floating stepping stones. While practising to
cross over and not fall into the water we must have looked like some
comic ballet dancers as with arms outstretched for balance we tried to
get a rhythm into our feet and step from pillow to pillow and reach
the opposite bank without getting wet. The inflatable two-man rubber
dinghy was part of the platoon's equipment used on bridging schemes.
There was a correct way to get into the dinghy without upsetting it
and we all went through the drill. From both these exercises we got
soaked but we had many laughs to compensate for our discomfort.

We had route marches both by road and across country. On one of the
cross country marches, \lieutenant Taylor brought the platoon to the
bank of the Tees and told us it was time we tried the correct way to
wade across a river. We crossed in single file with the tallest man of
the platoon in the lead, and each man clutching the belt of the man in
front. \Sapper Caswell and I were two of the tallest men and Caswell
took the lead with me behind him. We found a couple of sticks to
probe the river in front of us searching for hidden holes or sudden
shelving in the river bed. At its deepest the river was over waist
high on us and chest high on the shorter men who were in the middle of
the file. the flow of the river wasn't too strong and the river bed
was firm to our feet so we managed to make a safe crossing. The
remainder of the march was done at the double to fight against the
chill of our wet clothes.

In May platoon exercises on bridging began which took us away from
Hurworth for several days at a time. Bridge building using the folding
boat equipment was carried out on the River Blythe, a short distance
inland from the town of Blythe where the river was still tidal. This
type of bridge had a maximum load capacity of nine tons and was soon
superceded by the Bailey bridge equipment. At first we built the
bridges in the day time to get used to the equipment, then we built
them at night. There were ramps at each end of the bridge which
adjusted themselves to the rise and fall of the tide on trestles. On
one of the night operations, the party fixing the ramps and trestles
on the enemy side of the bridge were having difficulty in getting the
trestles erected. Some pins were slightly bent and were not aligning
themselves to holes in the trestle legs. \Sappers Peters, Morris and
myself were detailed to dash over and give the party additional
lifting strength. We didn't get very far. The decking party had
fallen behind with their work and to catch up in the sequence of
building had left a piece of decking unfinished. In the dark, and in
our hurry we failed to see this hole and fell through it. As I went
through the gap I caught my chin on a transom. My jaws were snapped
together so hard that my top dentures broke. Three wet, angry sappers
struggled to the bank muttering all kinds of threats to the decking
party. A rum ration was issued for these night exercises and tonight
my ration was more than welcome. Building bridges on the river Blythe
was a dirty, messy job. At low tide boats and equipment had to be
manoeuvred over a wide expanse of tidal mud. In places it was ten to
twelve inches deep and the glutinous stuff sucking at your feet almost
brought you to a standstill. To fall over was fatal for it was
impossible to push yourself upright again without assistance. The
folding canvas boats were unreliable. From long storage in a folded
position the canvas weakened at the folds and often split along this
crease when built into a bridge. Changing damaged boats could be more
difficult than building the bridge. My teeth were repaired at an army
dental unit in Darlington I spent the rest of the day in town.

In between schemes a percentage of forty-eight hour leave passes were
issued. The number depended on the C.O.'s discretion ad to how many
men he could allow away from the company at any one time. Whenever
possible I applied for one of these passes and since a large number of
men either didn't wish to spend money on railway fares or were too
far from home for the pass to be worthwhile I was often successful in
getting one. Railway warrants were not issued for these short leaves
and to supplement my money for rail fares I teamed up with \sapper
`Ginger' Nichols to go hay-making during the short periods between
schemes. The local farmers were glad of a little extra help to gather
in the hay and paid us two and sixpence an hour for the hand blistering
work.

Ginger was lucky not to be badly injured on one of our trips to the
Butts for firing practice. After firing live rounds from our rifles
they had to be boiled out to remove any carbon deposits left in the
barrel. Boiling water was poured down the barrel to wash it out and
was followed with pulling oiled cloths through the barrel to prevent
rusting. On this particular shoot Ginger saw a spare rifle standing
against the wall of the Butts and when it was his turn to fire at the
targets he decided to use it and so save himself the trouble of
cleaning his own rifle. Ginger was always on the lookout for these
sort of dodges. Unknown to him this rifle was waiting for an
armourer's attention to remove a blockage in the barrel. While the
sapper who owned the rifle was pulling a wad of cloth through the
barrel his pull through cord broke leaving the cloth stuck in the
barrel. Ginger took his position on the firing line and squeezed the
trigger. There was a louder than usual explosion with a recoil from
the rifle that almost broke Ginger's shoulder. The explosion also
split the barrel of the rifle. For firing another man's rifle without
permission and damaging it by neglect he received a detention
sentence.

No guards or gate pickets were mounted at Hurworth during the day, but
at night a fire and roaming picket was mounted for the hours between
18.00 hours and 06.00 hours. After the picket mounting parade and the
detailing of duty periods for each man, the off-duty or rather those
not on patrol returned to their hut or went into the NAAFI and later
lay on their beds fully dressed until it was their turn to patrol.
Prince, a stray crossbred Border Collie dog had adopted the platoon
and lived in the camp. When I was on fire picket I loved to get
Prince to accompany me. If the night was cold I would squat in the
dark with my back to a tree on the spinney path and hug Prince to me
for warmth.  His low throat rumble if there was any unusual noise or
movement was sufficient to have me on the alert. One night a sapper
patrolling round the hard saw smoke coming from the hut where the
ropes, netting and other stores were kept. He rattled the camp alarm
to rouse everyone and a chain of men were soon organized between the
hut and the river passing along buckets and utensils of water to try
and douse the flames. Another party of men began clearing the
explosives hut, its sides were beginning to scorch from the burning
hut, now well alight. By the time a fire engine arrived from
Darlington there was little left of the hut and stores to damp
down. Careless smoking was ruled to be the cause of the fire and a `No
Smoking' notice went up on the hard. The incident illustrated how
vulnerable our wooden huts were to catching fire.

Our company was detailed to organize a series of demonstrations and
lectures on mines and booby traps to be give to the 6th Guards Tank
Brigade who were training on the Yorkshire Moors. As training schemes
became larger the Guards and the 15th Scottish Division worked
together on operations and tactics which proved to be valuable when
they fought together in Normandy. \Lieutenant Taylor and no. 2 platoon
were chosen to give these lectures and demonstration to be held at
Askrigg Masham, Bedale and Leyburn. For two weeks we gathered
together models and exhibits for the lectures and collected gear
required for the demonstrations and for our camp. At the same time
\lieutenant Taylor introduced more spit and polish parades to try and
get our standard of turnout as good as the guards.

It is customary for Royal Engineers to have their jack knives on
lanyards worn round the waist with the knife dangling over the right
buttock. Too often the knife was in a top pocket of our B.D. blouse or
not even carried and \lieutenant Taylor had a purge on this one
morning. When we paraded at 08.00 hours he told the orderly sergeant
to take the names of all men not wearing their jack knives round the
waist. As \sergeant Cawfield moved to take our names, and mine was on
of them. \Lieutenant Taylor called out: `Put your name down sergeant
you haven't got your knife on'. \Sergeant Cawfield answered `Excuse
me, sir, I cannot see yours.'  \Lieutenant Taylor fumbled at his waist
and burst out laughing. `OK' he said, `I'll join you for spud bashing
tonight.'  Sharing defaulters with an officer and a sergeant was quite
a joke. the quota of potatoes for the cookhouses were soon peeled and
we were dismissed early for a spell in the NAAFI hut which closed at
21.00 hours.

We were away from Hurworth for about three weeks on this
demonstration scheme and camped at the four chosen centres. The
campsites had been well chosen by \lieutenant Taylor. Each one was
beside a lovely Yorkshire stream with clear running water and firm
rocky or gravelly beds, often with a waterfall nearby for ablutions.
The July temperature was warm enough for us to splash about in the
evenings, playing a rough form of water polo. There was always a ball
of some kind amongst our equipment and we were never without packs of
cards. I enjoyed exploring these streams which seemed to abound with
waterfalls of some kind, some were quite spectacular. This was my
first real exploration of the Yorkshire Dales and I became quite
attached to them; so much so that I brought my family here for some
lovely holidays after the war.

\Corporal Jarvis and myself were made responsible for the models and
exhibits used by \lieutenant Taylor to illustrate his lectures which
were held in the mornings. In the afternoon after our sandwich snack
we gave the lecture room a clean-up and rearranged the exhibits for
the next lecture and were back in camp early enough for me to go
exploring before the demonstration party arrived back for the evening
meal at 18.00 hours. Their booby traps and mines were prepared during
the morning while the Guards were at lectures and the demonstrations
by \sergeant Cawfield's and \corporal Murphy's squads were given in the
afternoon.

Two sappers, Lauri and Stevens who were still serving a C.B. sentence
for being absent from camp after 24.00 hours were on permanent camp
duties which included digging latrines, washing up for the cooks and
going the night picket. To ease their plight the potato peeling chore
was overcome by each man picking up two potatoes, peeling them and
dropping them into a bath of water as he left the dining tent. Lauri
and Stevens were unlucky to have been picked up by M.P.s in Darlington
in the early hours of the morning. They had been out with some girls
for the night and when answering their A.W.L. charge they had told the
C.O. that they had lost their way. The C.O. nearly exploded when he
heard their excuse. He told them that if they had said they had
overstayed their time with girlfriends he would perhaps have been more
sympathetic but for two of his sapper to say they were lost in
Darlington after the length of time they had been in the area was
inexcusable. He gave them fourteen days C.B. and the camp chores on
the demonstration course, which amounted to an extension of C.B.

At Leyburn with only a few days to go before we returned to Hurworth I
was thinking about how this course had been like a camping holiday and
that soon we would be on those uncomfortable training schemes again.
However, I was to have quite a pleasant surprise before returning to
Hurworth. The D.R. who came daily with our mail brought a privilege
leave pass, with railway warrant and food ration coupons for me to go
on ten days' leave. \Lieutenant Taylor gave me permission to leave
camp at mid-day after his morning lecture which gave me ample time to
catch a bus for York and catch the Birmingham train, the same train I
used to travel on from Darlington when I had leave passes. I walked
the few miles into Leyburn to catch a bus for York. While waiting at
the bus stop I was joined by a guards sergeant. He was going to
Birmingham on a short leave pass to visit his sick father. He told me
that he had frequently hitch-hiked from York to Birmingham which he
intended to do tonight rather than wait for the train. Because of the
convenient times of trains for my leave passes I hadn't bothered to
hitch hike, but having been convinced that I could save hours on this
journey I decided to accompany him. We had a quick tea in a canteen
at York and set out to thumb our way to Birmingham. A private car
driver took us as far as his local pub, a few more miles along our
way, where we had a couple of beers. Traffic seemed to be quiet that
night and we walked miles before we had another pick up. By now I was
beginning to regret my decision to try hitching a ride, even the
sergeant was getting pessimistic about our luck. Eventually a seedy
looking lorry pulled in in answer to our thumb signs. He was going to
West Brom and agreed to give us a lift into Birmingham. The sergeant
got in the cab. There wasn't room enough for three so I climbed into
the back of the lorry and tried to make a comfortable seat from some
tarpaulins. The lorry had a huge casting of a naval gun turret roped
in the back. It was going to West Brom for machining. The casting was
too heavy for this type of lorry and frequent stops had to be
made for cooling off. For me it was a most uncomfortable ride and how
I wished that I had caught the train at York which would have got me
to New Street station hours before this hitched ride. I was cold,
dirty, tired and hungry and I vowed that never again would I hitch
hike while public transport was available. After leave it was back to
Hurworth and more training exercises on bridge building, mine laying
and mine clearing.

Four months had passed since I was posted to 278 Field Company
R.E. and I now had many friends to go out with during my off-duty
hours. It is difficult to name any one of them as being a special
friend for I was always prepared to join in with anybody who had my
idea of entertainment. Drinking parties were not one of them, my
money was better spent on rail fares. There was a wonderful team
spirit in the platoon, with friendly rivalry among sections trying to
prove which was the best section in the platoon. \Corporal Murphy
drove us hard to be the best. This spirit I later discovered
existed throughout the company. Company discipline, although strict
was not harsh and very fair and our C.O. was more concerned about our
training for efficiency than he was about bull. This attitude
towards bull cost the company a month's loss of all privileges while
we were at Otley. An inspecting brass hat complained about the
appearance of the camp. We had been out on a company scheme and the
white washing of kerb stones, and boundary markers were below
standard. When the brass hat complained to our C.O. about this
the C.O. asked him which he preferred, training or unnecessary bull?
The company couldn't do both. For this our privileges were cancelled
for a month. Although we grumbled about losing our leave passes we
were behind the C.O. for his attitude about whitewashing.

\Lieutenant Taylor was promoted and posted to an Airfield
Construction Company, R.E. and we now had an energetic and keen
young officer, \lieutenant Baron as platoon commander who soon
became as popular with the platoon as \lieutenant Taylor had been.

About August we left our camp at Hurworth and moved to Otley, midway
between Bradford and Leeds. We were not allowed to take Prince with
us. He was left behind with the NAAFI ladies who promised to take
care of him. This camp at Otley built on the town show ground was
large enough for the company to be together in one place and, round
about, the 44th Lowland Brigade of the Divisions were in similar camps
to ours.

One of the air raid shelters built in our camp was found to contain a
large quantity of explosives and ammunition left behind by companies
who had been in this camp before us. Due to unsatisfactory storage
conditions it was beginning to sweat and corrode and looked unstable.
An inspector of explosives called in to examine the sweating
explosives ordered all of it, together with the rusting and corroding
ammunition, to be destroyed. The task was given to \lieutenant Baron
who detailed No.~2 section for the job. While \lieutenant Baron and
\sergeant Cawfield went on to Ilkely Moor to select a spot suitable for
blowing up the explosives, \corporal Murphy organised us to pack the
sensitive materials into boxes and load it onto our section truck.
there was not the usual light-hearted banter while we were doing this
work. Altogether there must have been a quarter of a ton of
mixed explosives and ammunition to destroy. When loaded, half of us
sat on the boxes in the back of the truck to stabilise them while our
little Taffy driver carefully drove us to Ilkely Moor and onto the
spot chosen by \lieutenant Baron for the blow up. The other part of
the section were in a pick-up with red flags flying to denote a
dangerous load. They drove in front of our truck. I didn't enjoy
that ride, sitting on a box of unstable explosives and I'm sure the
others like me travelled with crossed fingers. the boxes were stacked
into a small depression in the ground and well sandbagged to contain
the shrapnel of the hand grenades and the bullets of the small arms
ammunition. While \corporal Murphy made up his detonating charges,
inserted electrical detonators and wired them back to an exploder box
situated well away from the dump, the section radiated outwards to
ensure that no-none was in the danger area and took post to keep out
intruders. When \lieutenant Baron received all clear signals from the
sentries he signalled to \corporal Murphy who was safely tucked behind
a small hillock, to detonate. There was a terrific bang when the
explosives blew up and Ilkely Moor now had a new hole in its surface
which was afterwards inspected to make sure everything was destroyed.
The area round about was also searched for anything that could be
dangerous.

\Lieutenant Thomas, commander of No.~3 platoon, who was about six feet
two inches tall and weighed something like ten stone was naturally
nicknamed Jumbo. He was a Rugby league fanatic and he began
organizing rugger games which we played on the showground arena, a
grassed area in front of a grandstand, all out of bounds except for
these organized games. At first these games were enjoyable but
gradually without proper refereeing and the introduction of rough play
they became a free for all and it became harder to get players. The
games were also a bit one sided for when Jumbo gathered the ball it
was almost impossible to floor him. Like a runaway bulldozer he
charged towards the goal line with half the opposing side hanging onto
his shirt. His handing off was a thump in the face leaving you with a
bloody nose or a split lip and in tackles he introduced the pulling of
under arm hair. He was a dirty player. \Lieutenant Shaw of No.~1
platoon and our \lieutenant Baron soon had excuses to opt out of the
games and in the evenings or on Saturday afternoons it only required
someone to mention that Jumbo had been seen leaving the officers' mess
with a rugger ball under his arm for everyone in camp to go into
hiding or make hurried exits through the barrier and into Otley.

I was on gate picket one night doing the 24.00 hours to 02.00 hours
turn of duty and it was well past midnight when everyone without a
special pass should have been in camp. I saw Jumbo approaching. He had
both hands in his pockets, cap tilted back on his head and was quietly
whistling to himself, obviously a little merry from drink.  Although I
recognised him I thought it wise to challenge him in the correct
manner, and not take a chance of being put on a charge for neglect of
duty. Shining my torch in his face I called him to halt for
recognition. This annoyed Jumbo and he shouted `Put that bloody light
out. You know who I am and let me through'. He stomped into the guard
room, stormed at the guard commander and faulted all he saw.  The
incident was recorded in the guard book and we heard on the grape vine
that Jumbo had a ticking off from the Major for his ungentlemanly
behaviour. After that I kept clear of Jumbo as much as I could.

Our old Bedford trucks and Bren carriers were called in and we now had
new Austin four wheel drive trucks with winches built on the front,
and white armoured cars to replace the Bren carriers. The Austin
trucks had more body room and by careful arrangement of our tool boxes
and equipment and a few improvised hammocks, our section was able to
sleep in the truck when we were on exercises, that is until some nosey
M.O. discovered what we were doing. He put a stop to this practice
from a hygienic point of view. I must admit it did get a bit fuggy in
there. Building a bivy from groundsheets and gas capes was all right
while the weather was reasonably warm and dry, but not so comfortable
as the year rolled on when the streams in which we washed and shaved
were covered with ice. Ablutions then were distinctly chilly and our
food, brought to us in hay boxes soon became cold in our mess tins.

On one of our exercises the platoon was laagered\footnote{A camp or
encampment.} in a farmyard when our evening meal was being
prepared. It had been a quiet afternoon and we were looking forward to
a quiet night. The stew and tea, plus a couple of hard biscuits was
dished out and before we had time to consume it we were called upon to
move out and build a a class 40 Bailey bridge across a river where the
road bridge was supposed to have been hit by shellfire. Two umpires
stood on the bridge to make sure we didn't cheat and use it. No.~2
section was detailed to prepare the river bank on the enemy side for
the bridge and to our horror we discovered that our rubber dinghy was
punctured and useless.  We now had to wade across with our tools and
equipment which included the metal parts of the bridge support, all of
which we normally floated over in the dinghy. The heavy timbers used
to strengthen the bank were pulled over by rope. It was another night
of cold and wet when my rum ration was gratefully sipped.

We were exercising with the 6th Guards Tank Brigade on one occasion,
clearing a gap through a minefield for the tanks to begin an assault.
This particular field was supposed to have machine gun covering fire
and we had to clear the gap in the dark. Crawling on our bellies
through the wet heather and spongy ground we probed for the mines with
bayonets and removed them. For some realism the trip-wired booby
traps were attached to small explosive devices which prevented any
haphazard searching. The devices were harmless unless your face
happened to be over one when it detonated. In the darkness they were
quite startling when their flash revealed a mistake in searching and a
grim reminder about what could happen with the real stuff. The
umpires did make some of our searchers casualties. On all our
exercises I never made the casualty list, an omen or not I cannot say.
To help the tank drivers with their night drive through the gap, small
shaded lights were attached to short stakes, battery operated cycle
lamps were used for this. The gap was cleared and marked on one night
and the tanks drove through the following night. Having cleared the
gap and returned to the start line for further orders, a group of us
sheltering behind a piece of broken wall were having a moan about why
brass hats chose such situations for these exercises. Out of the
darkness from the other side of the wall a voice said, `I couldn't
help hearing your grumbles sappers', and out of the darkness the
brigadier appeared. He gave us a lecture as to why the brass hats
deemed it necessary to have exercises in bad weather. He told us
that working in adverse weather and living on half-rations was to
condition us so that when we were in battle we could carry out our
duties efficiently and ignore the weather conditions. Sometimes bad
weather could be an ally in war and he concluded by saying goodnight
before he strode away.

These big schemes took place on the Northumbrian and Cumbrian moors
and our journey from Otley to these moors took us through Newcastle.
When our convoys approached the bridge over the River Tees the D.R.s
rode up and down the line of vehicles calling to the truck drivers to
close up so that the column could get through the city without being
split up. On this particular trip the convoy was well closed up and
doing about 30 m.p.h. over the bridge when a pedestrian stepped from
the pavement in front of the leading P.U. truck, causing the driver to
brake sharply forcing the following drivers to do likewise. There had
been a light shower which had made the road surface a bit greasy and
the trucks began sliding and skidding in all directions. Luckily due
to petrol rationing oncoming traffic was virtually non-existent and
casualties were confined to our convoy. There were a few dents and
scratched paint on the truck and the occupants received a few bruises.
Our Taffy driver fought his skid well and regained control with the
truck facing in the opposite direction. There were some shunt
collisions and a truck of No.~1 platoon had a lucky escape. It had
mounted the footpath, crashed through the side railings of the bridge
and had one wheel hanging over the river. The winches built on the
front of the trucks protected the radiators from damage and came in
useful to pull the truck that had nearly gone into the river back onto
the road. Skids became very commonplace during the late autumn and
winter manoeuvres.

Ferrying was another exercise in our training programme. No.~2
section once had a ferrying job to do while the rest of the company
built a Class 40 Bailey bridge higher up the river. Infantry, with
the aid of light assault craft, had crossed the river and were now in
need of their anti-tank guns to consolidate and hold the bridgehead
they had gained, until heavier support could get over the bridge and
push on with the advance.

The ferry was constructed from Folding Boat Equipment. Four folding
boats with a platform built on them and ramps at each end which
raised or lowered formed the raft. Strong ropes were stretched from
bank to bank and crews of sappers in the boats of the raft pulled
along these ropes to get the raft across the river. The Jeep drivers
had to be good. There was just room on the platform both in width and
length to take a Jeep towing an anti-tank gun or a small trailer. Each
time one accelerated up the ramp and onto the platform my heart stood
still. The raft rocked uncomfortably and looked top-heavy with its
load. I quite expected the whole thing to topple over. After a while
the boat I was in began to ship water. The canvas was cracking along
the fold line and each time the raft was loaded the crack went under water. After an examination and knowing that the bridge building was going well \lieutenant Baron decided to carry on ferrying with the damaged boat, rather than change the boat and hold up the infantry support vehicles. What water we shipped while crossing with a load we baled out with our helmets on the return journey when the split was above water. It was arm and back aching work pulling on the ropes and Murphy was all the time yelling at us to go faster. There was a sigh of relief when the news came for us to cease ferrying and begin to dismantle the raft. The bridge was now open for traffic.  


Having obtained those good marks at the firing range at Maryhill and
being classed as an A-1 shot in my records, I was chosen to be in a
squad of marksmen under \sergeantM Duncan to zero the company weapons
at the Butts and since we would be there for a day or two we camped on
the Butts. It was here that \sergeantM Duncan almost lost his foot. We
were checking the Sten guns, a most unpredictable weapon for jamming,
and beginning to fire again after a heavy jolt if the safety catch was
not on. We were firing the Sten in short bursts at the targets when
\sapper Brownley's gun jammed. Instead of staying facing the target
and holding his gun in the air he turned round with the gun pointing
to the ground while trying to release the jam. The gun cleared itself
and began firing a stream of bullets which went close to the sergeant
major's feet. When the sergeant major got his breath back and looking
a bit pale in the face he roared at Brownley using a beautiful
selection of profanities. Brownley was then forbidden to take part in
any more weapon firing and given all the menial camp tasks that Duncan
could find for him.

Having had no practical experience of floating Bailey bridge
construction the company moved to Lancaster for a month's intensive
training to build this type of bridge. Our billets were in an old
mill about four miles outside Lancaster and about two miles from the
hard on the River Lune which was a permanent training establishment.
The Lune was a fairly fast flowing river, especially after heavy rain
and it became quite dangerous for bridge building. Some Canadian
sappers had been drowned here while on the same course we were now
going to take. One of their pontoons with sappers on board, got our
of control and was swept away by the rain-swollen river to crash into
the stone pillars of a road bridge. The pontoon broke up and the
sappers were thrown into the river and drowned; their bodies were
later recovered from Morecambe Bay. In its quieter moods the river was
so clear that you could see down to the river bed an watch the large
salmon gliding in the water. We tried to gaff them with boat hooks,
but were never successful. There was also a large flock of seagulls
here who were ever ready to snatch at the bits of food we tossed into
the air. Their aerobatics were so fascinating that they had quite a
share of our haversack rations.

It was a hard course. The journey between billets and hard was always
on the double and we were dressed in battle order. The morning run
wasn't too bad but the return trip after a days heavy work on the
bridge wasn't so welcome. Building the bridge section on the pontoons
was awkward and difficult and getting the sections into position and
holding them against the current was a nightmare. As fit as we were,
our muscles groaned at the end of the day. Large rowing boats called
whalers were used to control and tow these sections of the bridge
into place. and they were crewed by six to eight oarsmen. The oars
were like young oak trees and the boat needed a lot of muscle power to
get them moving. At the end of the course it was customary to have a
whaler race with crews from each section rowing against each other in
a knock out competition. We had trials during our midday breaks to
select crews. I didn't make the grade to be a member of our crew and
wasn't sorry about that. Our section crew didn't get very far in the
competition; a section crew from No.~1 platoon had the honour to be the
winners.

\LColonel Miller, our C.R.E., insisted on a regimental guard being
mounted at the billet, the only time I ever took part in regimental
guard mounting. Prior to being on guard we had the afternoon off from
bridging to get our webbing blancoed and brasses and boots
polished. In a guard of this kind there was always a spare man known
as the stick man who was chosen by the officer of the day during the
guard mounting inspection for being the best turned out man on the
parade. The stick man did no sentry duties but was available to fill
in if a sentry was taken ill and he also collected the guard's rations
from the cookhouse and ran other errands. It was an honour worth
trying for. Our guard fell in for inspection and who should be the
officer of the day: Jumbo with \sergeant Greenway as orderly
sergeant. The choice for stick man narrowed down to \sapper Petronugio
and myself and the stick finally went to Petro. The next best thing
was to be first sentry which went to the second smartest turn out and
today that was me. Having taken part I began to cover my beat in true
Buckingham Palace style, I thought. Apparently Jumbo didn't think so
and he ordered \sergeant Greenway over to tell me to get straightened
up. He said I looked like a bag of shit marching about and I was to
keep marching through the first hour of my duty period I'm sure Jumbo,
like his elephant namesake had a memory and had not forgotten the
Otley incident. The C.R.E. paid a visit to the camp while I was on
duty and I had to call out the guard and give him the `Present arms!'
as he passed through the gate.

If I wasn't too tired after my day on the hard I walked into Lancaster
where there was a good canteen and one of my many friends running
short of cigarettes would always be ready to join me, but most nights I
was content to stay in billets. Saturday morning was a morning of
cleaning up billets, shaking blankets and cleaning kit and in the
afternoon liberty trucks were laid on for visits to Morecambe. I only
went into Morecambe once and found it so uninteresting that I
preferred the four mile walk into Lancaster for my entertainment,
where, besides the decent canteen, there were cinemas to visit. The
liberty trucks didn't leave the billets until after ten so that their
arrival in Morecambe was always too late for a cinema show. Most of
those who went were pub crawlers and the return journey from Morecambe
was far from pleasant due to drunken brawls and sickness.

Morecambe Bay with wire entanglements everywhere and its sea-front
buildings, other than those requisitioned by the Defence Ministry,
were closed and shuttered which gave it a very depressing atmosphere,
especially in the blackout.

On Sunday we had church parade and were marched to a nearby church for
the service, other denominations were accommodated in Lancaster and
taken there by truck. Our parades must have delighted the vicar for he
not only had a large congregation, his collection plates were much
heavier. The afternoon was usually spent on my bed with a book or a
letter to write and in the evening a walk with some pals into
Lancaster.

Tommy Drummond in the bunk next to me always wrote very regularly to
his wife in London and he received regular replies from her. For
several days while we were on the course Tommy had no mail and he
became very worried, Eventually the C.O. sent for Tommy and told him
that his wife had written asking what kind of soldier he
was. Apparently one night Tommy had written two letters one to his
wife and one to a girl he was playing around with in Otley and had put
the letters in the wrong envelopes. Tommy was given a forty-eight
hours compassionate leave home to try and sort out his matrimonial
affairs which ended in a separation order. His girlfriend also
discarded him when she found out that he was a married man.

The course came to an end and I think we were all pleased to return to
Otley.

Otley was a good town to be billeted in. There was a cinema and a well
stocked NAAFI where we could spend our off duty hours and a regular
bus service between Bradford and Leeds passed near the camp. Bradford
was a bit dull and wasn't visited very often, Leeds on the other hand,
had plenty of entertainment to offer. The market again being one of
the attractions.

Freddy Frisk, a company clerk, and I frequently went to Sunday
Evensong at Otley Parish Church, Here the people's warden, Mr. Watkins
and his wife befriended us and sometimes invited us to their home for
supper. Mrs. Watkins had one big grumble and that was the disruption
caused in the town centre when we moved away for manoeuvres. She
thought it was most inconsiderate of us to have traffic right of way
and so delay the townspeople getting to their work. Well she provided
such nice suppers we agreed with her about the disruption of civilians
getting on with their war effort They had one son who was in the
forces stationed somewhere in the Middle East. 

A privilege I had in Otley was to be able to borrow and take away
books from the library. It began with my visits there, selecting a
book to read in the library and then asking the librarian to hold the
unfinished book for me until my next visit. Eventually, on promising
to return any library book by post if I was suddenly moved away from
Otley, I was allowed to take books back to camp.

George Banwell, one of the sappers I went out with, was made platoon
clerk when Jones went on an officer's selection course. George
discovered a group of elderly ladies who met one night a week in a
chapel room in town. They provided a nice cup of tea and a home made
cookie for any serviceman who wished to call in and also darned socks
for those unable to do their own repairs. I went with him fairly often
to this chapel room for a quiet, homely and friendly chat with the
old dears.

George was a poor manager of his money; it was not unknown for him to
draw his pay, go out by himself and squander it on drinks for anyone
in the pub. After these drinking bouts he was broke and miserable and
I either lent or gave him some cash to see him through to next pay day
and made sure he had some cigarettes. In return for these favours
George often left my name off picket duty lists which he made out for
\lieutenant Baron's signature and he also endeavoured to get my
applications for forty-eight hours leave passes signed. Yes, I was
becoming corrupt, even sinking to bribery.

The bus service into Leeds was very convenient for catching my trains
to Birmingham when going home but the return journeys were not so
good. Leave passes expired at 24.00 hrs. on the day of your return to
duty and one always caught the last available train to get you into
camp at that time. Trains could never be relied upon for punctuality
and a blind eye was turned on late arrival in camp for those who had
been on leave. Provided you were in before reveille and had not been
stopped on the way by an M.P. The train I used to catch at Birmingham
should have arrived at Leeds in time for me to catch the last train to
Gaisley and then have plenty of time to walk the three miles from
there to camp and be in before 24.00 hrs.

The Birmingham train was nearly always late and missed the Gaisley
train, also there were no buses running at this late hour. Normally
there was plenty of time for the twelve mile hike from Leeds back to
camp but if the train was extra late then the journey was a hurried
one. There were usually several of us from the camps around Otley to
make the walk.

The corridor coaches of those days had toilets at one end of the
corridor and at the other end their were tiered racks for large pieces
of luggage. By curling into a foetal position it was possible to get
in these racks and have a sleep; there were never any empty seats at
Birmingham. I had slept a few muscle cramping hours on these racks
when returning from a short leave pass and done the twelve mile hike
back to camp which I had reached just before reveille and stopped to
read duty orders' in the dawn light. That best B.D.s were to be worn
on first parade wasn't pleasant reading for me for mine which I was
wearing looked pretty awful after it's night of ill-use on the
train. There I stood on parade looking like something out of a rag
bag. Creases in my uniform anywhere but the right places. \Lieutenant Baron
looked a little shocked when he saw my appearance and asked if I was
really wearing my best battle dress. Aster I answered in the
affirmative he gave me two days C.B., time he said to get something
done to improve my appearance. Chalky White a Londoner in No.~3 Platoon
who was returning from a privilege leave and was on the same train I
had caught at Birmingham had forgotten to change from his brown shoes
into Army boots and was put on charge for being improperly dressed on
parade.

This inspection and check on our best battle dress was in preparation
for General Montgomery's visit to the brigade. He was touring around
the troops who, unbeknownst to us, were preparing to invade France. It
was a drizzly morning on the day of his inspection and we were marched
to a large field on the Leeds side of Otley for the parade wearing gas
capes over our uniforms and steel helmets instead of our caps which we
had tucked into our belts. The brigade was formed up company by
company into three sides of a square and just before Monty arrived the
sun came out, something we were all hoping for. Gas masks and helmets
were discarded and quickly gathered up to be taken back to camp by
P.U.s laid on for that purpose. Monty arrived in his special Jeep and
after a few barked orders we were in open order for his
inspection. Momentarily he stood and looked each man in the eye and I
had never had such a pair of piercing eyes looking at me. They seemed
to penetrate right through my head. After his tour through the ranks
which took quite a long time he mounted his Jeep, called us to break
ranks and close round to hear his short speech, one which he seemed to
use on all these inspection parades. After reforming into companies,
the pipes and drums of 44th Lowland Brigade headed the march past for
Monty to take the salute; a rostrum for this had been erected on the
road back to Otley.

Coming back to short leaves, I scalded my foot while I was getting
ready to return to Otley after one of these leaves. Boiling water was
spilt all over my foot and I was unable to get my boot on, the local
police were contacted and asked if they could inform my unit about the
accident. They replied that I wasn't to worry, they would phone a
regimental aid post who were in a large house in High Street and they
would take care of me. An M.O. and a sergeant of the R.A.M.C. arrived in a
jeep, looked at my foot which had now blistered, gathered my gear and
took me away, telling Nona that if she wished she could visit me the
following afternoon at the R.A.P.

Here the small blister was drained and dressed and for a couple of
days I spent most of my time lying on my bed. Nona came to see me each
afternoon. By Thursday I could get a shoe over the dressing and by
promising to return to the R.A.P. by 20:00 hours the M.O. let me go
home. On Friday morning the M.O. said I was fit to return to my company
and began making out my discharge paper, but again putting me on trust
not to leave the house, he said I could go home for the weekend and
dated my discharge for Sunday. By now the soreness had gone from my
foot and I could get my boot on.

The camp was empty when I arrived back for reveille on Monday
Morning. The company was away on a scheme and for security a small
rear party of sappers with \sergeant Dickens in charge were left
behind. At 08:00 hours I reported to \sergeant Dickens in Company Office
who immediately wanted to know where I had been, I was reported
missing. I told him that I had been in hospital and produced my
discharge papers to prove it. I had been under the impression that the
company had been informed by the R.A.P. about my accident, but apparently
I was wrong and a warrant had been issued for my arrest for
desertion. The company wasn't expected back at Otley until Friday and
\sergeant Dickens didn't seem too happy about having another man in his
party and said he wished the hospital had kept me for a few more
days. Cheekily I said `How about a short leave pass serge., that would
solve your problems.' He called me a cheeky bugger, but made out a
pass for me and I was out of the camp and on a bus for Leeds like
greased lightning in case he had a change of mind. Our neighbour Mrs
Shirley, whose husband Bill was in the army, said it was amazing how
many times I came home, her Bill didn't seem to get the same
leaves. Perhaps Bill didn't try hard enough. I seldom managed to get a
seat on the train at Leeds. when going home. It was usually filled to
capacity at Darlington where its journey began. There was a chance at
York or Sheffield that some people would get off and if you happened
to be by that compartment you quickly claimed their seat. On the trip
home I managed to get as seat at Sheffield and after undoing my boot
laces to ease my feet and unbuttoning my blouse, I fell asleep. When I
woke up it was quite dark and the train was stationary. When I
enquired where we were I was told that the train was now in Birmingham
and had been there some time. This train always made a long stop at
Birmingham to load on mail for London, its next stopping place. Panic,
I grabbed my and with buttons undone and laces straggling, I pushed my
way along the crowded corridor to a door. I had just dropped to the
platform when the guard blew his whistle, waved his green flag and the
train began to move away for London. After that close encounter for an
unwanted journey, I asked fellow passengers to wake me up at
Birmingham when I had a seat and felt sleepy.

Frequent F.F.I. inspections were made to make sure no one was hiding V.D.
complaints and although we had been told that catching V.D. from
lavatory seats was extremely unlikely I had a problem about it and was
very careful when using toilet seats. The platoon was erecting three
Nissen huts on some waste ground near the C.R.E.'s H.Q., and our M.O. stopped
by to give us an F.F.I. With one hand holing my slacks at half mast and
holding my shirt high with the other hand I stood before the M.O. After
his quick inspection I walked away tucking my shirt into slacks and
preparing to button up when I heard him call to me. `One moment
sapper, I would like another check.' Immediately I thought, `It is true
that you can get V.D. from a toilet seat' and was quite relieved when the
M.O. said, `You have a slight sweat rash, sapper. Report to your company
treatment room and get painted with iodine for three days.' There
where many ribald remarks made about my treatment. The treatment room
was open for one hour each evening and one evening while Tompkins, the
first aid man in charge of the hut, was closing up for the night,
\sapper Middleton, a bit of a bully tried to push past. Middleton who
was much bigger that Tompkins, soon found himself on the floor locked
in a judo hold needing one more twist to break his arm. That was the
first inkling we had that besides first aid, Tompkins also knew quite
a lot about judo.

Most of our schemes were to attack enemy positions and advance, but
some were exercises on the defense of territory and we accidently blew
up a bridge on one of these defence schemes. To hinder attacking
forces our section was detailed to blow up a small bridge in a side
road. While \corporal Murphy and a few sappers prepared and placed the
explosives in position and wired them back to an explosives box, the
remainder of the section with \lcorporal Dippy took up covering fire
positions. We were using real explosives and detonators and after
umpires had inspected our preparations the detonators should have been
removed. The Umpires, during their inspection, chose \corporal Murphy
and two sappers as casualties and they were taken away by a medical
team to have their imaginary injuries attended to. When a runner came
to us with orders to destroy the bridge and retire to the platoon,
Dippy, with a theatrical gesture, pushed down the handle of the
exploder box and to everyone's amazement away went the bridge. We all
thought Murphy had removed the detonators before being taken away as a
casualty. Luckily no one was near the bridge at the time of the
explosion but the smoke and dust had hardly settled before an number
of officers, including one \major Broome and the C.R.E. \lcolonel
Miller were at the scene. There was also a very angry farmer there
complaining that the only direct access to his farm was now destroyed,
and how the hell did we think that he could get his milk
away. \Lcolonel Miller told us that we had done a very good job of
demolition and now the platoon must erect a temporary bridge for the
farmer's convenience. This kind of happening must have cost the
country enourmous amounts of money in compensation payments.

The planned period of training schemes which had begun in a small way
had now reached its last big scheme in which the 6th Guardo Tank
Brigade joined in with the 15th Scottish Division for a large
assault exercise. The battle began with a big barrage from the
divisional artillery. The flashes from the blank shells lit up
the night sky along a wide front. It was described a being like a
miniature Alamein. The barrage at the beginning of that battle was the
largest known up to that time.

We were on reduced rations for this scheme. Two hard biscuits with a
little margarine and a mug of tea was breakfast and then nothing more
until the evening meal of stew and tea often eaten in the field. I
became so ravenously hungry that I tried to eat a piece of mangold
thrown out for some sheep. It tasted awful and was too much for my
rumbling stomach to face. We built and demonstrated several Class 40
Bailey Bridges, ferried for the 44th Lowland Brigade and cleared gaps
in minefields for the tanks. It was a very exhausting exercise carried
out in mixed weather conditions. Sleep was snatched at any odd
moment. While on one of the ferrying jobs \sergeant Greenway lost his
crash helmet. \Sergeant Greenway moved around to ensure that platoon
sections received their rations. Having made contact with our section
on this ferry he was conferring with \corporal Murphy, leaving his crash
helmet lying beside his motor bike. Minutes after it was seen floating
down the river. How did it roll down the bank? Although he never
admitted it we reckon \sapper Richardson helped it on its way. \Sergeant
Greenway had booked him for insubordination the day before while we
were in the laguar with the platoon and we were sure that this was
Richardson's way of getting his own back on the sergeant.

At the end of the exercise the 15th Scottish Division assembled round
Rothbury race course; the course being used as an arena for
fete-at-arms where the special skills of the differing corps within
the division were displayed. Our company put on a display of Bailey
bridge building to represent the skills of the Royal Engineers. There
was a small stream running by the race course and over this we built a
100 ft class 40 Bailey bridge and at the same time announced that we were
attempting to beat a record of forty minutes to build this bridge made
by a team of Royal Engineers at Chatham.

The Royal Corps of Signals had erected a loud speaker system round the
arena and over this \captain White was going to give a running
commentary on our the progress of the bridge building and since he
would be in a tent well out of sight of the bridge he hoped that we
would be within the forty minute record on which he was going to base
his progress report. The bank preparations for the bridge supports and
the rollers on which the bridge was built were prepared in
advance. Caswell and I were in one of the panel parties of the chosen
team, a task in bridging that I liked best. From the word go a furious
pace was set with plenty of encouraging bawling from the N.C.O.'s. \Lieutenant
Baron was bridge commander, and he did his share of the
shouting. Panel parties more or less controlled the pace of the
building. Until a panel was in position other parties couldn't do
their jobs and the faster we went, the faster they had to go to keep up
with their work. With footwalks completed and a R.A.S.C lorry driving
over the bridge our time was given as forty-two minutes and our
exhaustion was forgotten as we heard the spectators' cheer after the
time was announced. The C.O. and \captain were so delighted with our
performance that they gave each man in the team vouchers to spend in
the NAAFI paid from their own pockets. It was only two shilling
per man but a medal could not have been more appreciated.

The finale of the fete-at-arms was a display given by the divisions
massed bands of pipe and drums. The sun shone for the occasion picking
out and brightening the many colours of the tartans of the ceremonial
dress worn by the bands and a wonderful change from the khaki battle
dress. Watching these guardsmen countermarching in the sunshine
and listening to the music of the bagpipes was one of the most
spectacular events of my life.

Another privilege leave followed this scheme, the last major exercise I
took part in. There were company schemes to keep us well trained and
physically fit.

All our vehicles had to be modified and sealed so that they could be
driven through several feet of water. Exhaust pipes were lengthened
and turned upwards to clear the water and, using a special sealing
compound, vulnerable parts of the engine were encased in the material
to keep away the water. Infantry regiment drivers also had many hours
practicing the loading and off-loading of invasion barges and for
this purpose the Engineers built special training areas. These were
raised platforms of earth with ramps to represent the barges and a
track circuit with deep water troughs built into it to test the
waterproofing of the vehicles. These tracks, from constant use and
wet weather, plus the water brought onto the track after the vehicles
had been through the troughs, often became far to boggy for training
and we needed to carry out maintenance work on the areas to keep the
usable. A concrete track would have been ideal but that would have
destroyed the need to be able to keep going over rough terrain.

No.~2 platoon was sent to one of these areas to improve conditions
there and I was with a team of sappers fetching hard core from a
nearby quarry to drop into the boggy parts. The broken stone was far
too big for shovelling and we loaded and threw it off the truck by
hand. Our C.R.E. was making himself a nuisance at this site which, not
being particularly well drained was giving us some trouble. He was not
only harassing \lieutenant Baron and the N.C.O.'s but he was also
stopping sappers to ask them questions on the R.E. training manual and
if the right answer wasn't given you received a reprimand and ordered
to swot up the Manual. I was one of his victims. I didn't give the
correct answer to an explosive question. Having come back from the
quarry with a load of stone we saw the C.R.E. standing next to a
puddle of water and Diamond said ``Let's shift the old so and
so''. With our backs to the end of the truck Diamond, Caswell and
myself, began tossing the stone between our legs and dropped some into
the pool of water. The result was rather more than we had anticipated
for the C.R.E. was drenched with muddy water. He stormed up to our
truck, rattled its side with his cane and ordered us down. From his
looks I'm sure he would have loved to rattle us with his cane. Trying
hard to keep straight faces we stood stiffly to attention while he
raved at us for careless and dangerous behaviour. By now \lieutenant
Baron and \sergeant Cawfield had joined him and the C.R.E. told
\lieutenant Baron that we three were confined to camp for seven
days. The C.R.E. went away to get changed and we were all able to have
a jolly good laugh and that included \lieutenant Baron and the
infantry officers. We didn't see the colonel again on that job.

In November \lcorporal Diaplo, myself and a few more C\&J sappers from the
divisional R.E.s were sent to a Ministry of Supply depot at
Selby. Here there were mountains of broken bunks, chairs, benches
and Office desks and our job was to show some Italian P.O.W.s how to
cannibalize materials from the more serious damaged articles to repair
those that were least damages. They were P.O.W.s from the North
African campaigns and told us that they were pleased to be out of the
war and away from the bullying Nazis. By volunteering for this kind of
work they received extra privileges. Many had some knowledge of
English so communication wasn't too difficult and their keenness to
please enabled them to quickly grasp what they were expected to do
here. The Pioneer Corps, they were not Royal in those days, were
responsible for their security and marched the prisoners from their
quarters to this small workshop set up for the job and generally kept
an eye on them. For us it was a miserable detail. We were given a
Nissen hut by the perimeter fence for our quarters and using broken
unusable wood from the pile of damaged articles we kept the hut nice
and warm, but the lighting was so dim we could hardly see to read. It
was like prison for us since we were more or less confined to this
hut, even our meals were brought to the hut. If we went out at night,
and that wasn't often for there was no entertainment in Selby, you
were searched by the guard on the gate and searched again when you
returned. I was glad when, after about two weeks on this detail. I
returned to Otley.

It was going to be another Christmas away from house and family and
without the joy of giving and receiving those little presents
associated with the season. Sergeants bringing round the buckets of
tea well laced with whisky began another army Christmas Day and, after
a long morning, we had a dinner of tinned turkey and tinned Christmas
pudding washed down with free beer from the NAAFI. The afternoon was
spent quietly in camp and after tea it was beer and Housey-housey in
the so-called recreation hut. It was normal duties again on Boxing Day
but this Christmas was certainly better than the one spent in
Maryhill last year. In January I had eight days of privilege leave and
little did I realise that this would be my last official leave home for
twelve months. As a small compensation for not having presents to send
home at Christmas I had collected a sizeable amount of chocolate and
best toilet soap. It was one of the joys of leave to watch Cynthia
delve into my pack and unearth the chocolate and by now Garth had
developed a strong liking for the sweet meat.

After this leave I was posted to Chatham to take the army carpentry
and joinery course which, if I passed, would entitle me to a bit more
pay.

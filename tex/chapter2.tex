I was taken to a hut, one of several built on the side of a huge
parade ground, which became my Home for the next six weeks. The hut
was full of men in various stages of dress, tidying up their beds and
blankets. Seeing an empty bunk obviously waiting for me I dumped my
greatcoat, gas-mask and helmet onto it. The bunks were two-tiered and
I had a top one. There was hardly time to say hello and introduce
myself to those nearest to me before an N.C.O. assembled us outside
and marched us away for breakfast. Army breakfasts I found began with
porridge which I'm very fond of, except when it has a burnt flavour;
and the rest of the meal varied from day to day. Scrambled egg made
with dried egg powder and looking like a piece of chamois leather
served on a piece of fried bread or bacon and tinned tomatoes or
sausage and beans were some of the other dishes we had for breakfast.
The tea was generally good. It was hot, strong and sweet but was
sometimes spoiled by over-chlorinated water. I have a fairly good
appetite and am able to eat most things and this morning I was so
hungry I could have eaten the table. Some of the more finicky eaters
refused their rations and I was only too pleased to have extras. When
breakfast was eaten I returned to the hut and began to get better
acquainted with the men around me and asked them what had happened the
day before. At 08.00 hours with other sections of the new intake we
mustered on a small parade ground for roll call and after the parade a
lance corporal was detailed to rush me round the barracks, drawing
kit, having inoculations and injections. By tea-time my lost day was
made up and I was on schedule with the rest of the section. From now
on I was a surname with the number 14340952 behind it, a long number
which I thought I would never remember. It was to become so much a
part of my life and repeated so often that it is for ever fixed in my
mind. I now had an awful amount of gear and wondered where to store
it. At the moment it was just making its way from bunk to floor or
vice versa; we had no lockers in the hut. A row of double bunks were
set along each wall of the hut with a space about six feet wide
between bunks and in the centre of the hut there was a slow combustion
stove which proved to be more of an ornament than a means of getting
warmth. Fires were not allowed before tea-time and our fuel ration of
one large bucket of coke per day was hardly enough to warm the stove
let alone warm the hut. An annex off one end of the hut contained the
ablutions and toilets and the hot water taps in there should have been
marked `T' for tepid since the water from them just had the chill off;
showers were a very hurried affair. 

Our section N.C.O., a corporal of the Glasgow Highland Light Infantry,
was a married man who had permission to sleep out of barracks when his
duties permitted. He was a decent N.C.O., known to the Section as
Corp or Jock and we rarely saw him after training sessions. The
fellow on the bunk below me became known as Pinkie because he had a
very fresh complexion and also blushed quite readily when spoken to.
He was several years younger than me, quite a likeable lad and we
often chummed up to go out of the barracks in our free time for a
change of atmosphere. The only other member of the section that I can
recall was one we named `The Camel'. He was an awkward recruit, tall
and gangly and was the cause of the section having many extra drills.
He could not keep time when marching or drilling with a rifle. He was
never sure which was his right side or which was his left. He loped
along with a camel-like gait, usually out of step, sending the section
around him into disarray.

After those first two days of kit drawing and having inoculations we
were instructed by \corporal Jock on how to assemble our packs and
webbing into Marching Order and Battle Order and told what each pack
of that order should contain. We were shown how to fold our blankets
and overcoats correctly and how these and our kit should be laid out
on our bunks for morning inspection. A chart pinned on the door could
be referred to for this lay-out and after these instructions my gear
seemed more manageable. Most of that first weekend was spent
blancoing our webbing, polishing brasses and cleaning rifles ready for
Monday morning when training was to begin. The rifles handed out from
the Armoury were the greasiest things imaginable. It took ages to
remove the thick brown grease that had been liberally plastered on
them.

Reveille\footnote{A military waking-signal sounded in the morning on a
bugle or drums.} at 06.00 hours couldn't be missed. The bugler who
blew the calls had a post outside our hut and his calls were loud and
clear.  At 06.30 hours after a wash and shave we paraded outside the
hut in P.T. shorts, vest and shoes and were taken by a P.T. instructor
for a run round the streets of Maryhill which I didn't think was an
ideal way to begin a November morning. Returning from the run blue
with cold and often wet, we had ten minutes to change and parade for
breakfast. After breakfast we swept out our bed area, sweeping the
bits to the stove for the permanent billet orderlies to clean up while
we were out training. These orderlies, usually low grade in physical
fitness, also cleaned the ablutions and left our fuel ration by the
stove.

Our bunks, which had to be in perfect line, were juggled until all the
ends touched a string stretched through the hut. Kit, now laid out as
shown on the chart, was also lined up by a string. Huts were randomly
chosen for inspection and extra drills given if they were not up to
standard.

First parade at 08.00 hours was taken on a small square where the roll
was called and we were inspected for smartness.

Having been in the Home Guard I was familiar with some of the drills
and was elected Section Marker. This involved being on parade ten
minutes before main parades, to be put through the drill of `Marking
off" where sections were to fall in. On being dismissed, each section
was marched away for its specific session of training which often
meant changing our clothes. For First Parade, classroom lectures,
I.Q. tests and drilling on the square we wore battle dress. Denims
were worn for field exercises and assault courses and P.T. gear for
the early morning run and gym exercises.

In a few days we all began to feel ill. The effects of our
inoculation, vaccinations, and having caught colds were making us
feel miserable. I now began to worry about my chest but my cold
appeared to by a heady one. Some of the men were in agony from their
smallpox jabs. Their arms were swollen and sore and drill, especially
Small Arms drill, was torture to them. I was a lucky one for my
vaccination developed into what looked, and felt, like a large gnat
bite. Although Corp said our present training was light, to
accommodate our miserable condition we didn't believe him. The Camel
was the only one to need medical treatment. He became delirious one
night and was put in sick bay for a couple of days.

The training periods were continuous each day from 08.00 hours until
17.30 hours with an hour's break for dinner except for Saturdays and
Sundays when the afternoons were free. My bones and muscles ached too
much after the unaccustomed exercises for me to bother about going out
of barracks. After my bunk, the NAAFI was my favourite place. It was
warm in there, I could read or write my letters and feed my ravenous
appetite on tea and waddies. In all my misery I never lost my
appetite.

Finding that close to the barracks there was a municipal bath-house
where a nice hot bath could be had for sixpence, I decided to try it.
I also thought that while I was there I could freshen up a few
handkerchiefs which were being soiled quickly from my head cold.
Having paid my sixpence I was directed by the attendant to a cubicle
with a bath in it; there were about six of these cubicles along the
corridor. Hot water was already flowing out of a pipe protruding from
the wall at the end of the bath which stopped when the water reached
the regulation five inches in depth. Ten minutes was the time allowed
for this luxury of soaking in nice hot water and when ten minutes had
elapsed the attendant called out that my time was up. Clever like, I
thought an extra minute wouldn't matter and carried on bathing.
Suddenly a gush of icy water hit me in the back. Gasping for breath I
was out of that bath in a flash. I had forgotten that the attendant
had control of the water and he had turned on the cold water valve to
get me out of the bath. He came in to clean the bath while I was
dressing and jokingly I called him a lousy so and so. He grinned and
replied `other people want a bath you know,' and he carried on
preparing for the next customer. On other visits I made to the
bath-house I made sure I was promptly out when my time was ended.

The food here was pretty good considering it was bulk cooked.
A.T.S. did the cooking and cleaning in the mess halls and an A.T.S
officer always stood by the waste bins lecturing those who wasted
food. She said, `If you didn't like the look of it you shouldn't have
taken it.'  To complain about a meal when the Orderly Officer entered
the mess and asked for `Complaints' required a lot of courage.
Invariably when a complaint was made he either sniffed the food or
tasted it. He then told the man making the complaint that the food
was all right and was the same as that being served in the officers'
mess. I wondered!

Gradually my body aches disappeared. My muscles were becoming more
used to the unaccustomed physical exertions. My cold also began to
improve and, generally, I was feeling fitter. One arduous exercise we
had fairly frequently was the Assault Course. I found this rather
exhausting and was thankful that we usually did this for the last
period of the day. The course had all kinds of obstacles built into
it to make it difficult. We had to crawl through tunnels constructed
from large diameter pipes with a few inches of water lying in the
bottom, swing on ropes Tarzan style to get across wide ditches of
muddy water, climb over frameworks of rope netting, overcome barbed
wire entanglements and scale over high brick walls. All done `On the
double' and wearing Battle Order. There was often a laugh to be had
from the exercise, usually at someone else's expense. The most common
laugh came when some guy missed his landing on the opposite bank while
swinging on the rope and fell into the ditch of muddy water. On one
of these runs over the assault course I damaged a ligament in my right
foot when I landed awkwardly onto the concrete pad beneath one of the
walls. I limped to the medical aid centre in the barracks where our
M.O. strapped up my foot with Elastoplast, told me to keep it on and
report back in seven days time and carry on with training. The
stiffness of the Elastoplast dressing made marching and drilling
uncomfortable and stamping my foot on the hard parade ground during
drills was done with caution. Sometimes I received some
uncomplimentary remarks from the drill sergeant about my slovenly
drill movement.

There was a nasty accident in the barracks while a section following
us in the training schedule was receiving instruction on some of the
uses of mines and booby traps. A small mine in use at the time was
called a `talc mine' because it was in shape and size similar to some
tins of talc being sold in the shops. The mine was activated when the
case was distorted by the weight of any light vehicle passing over it.
The weight of a man was considered to be insufficient to distort the
case and the instructor would stand on the mine in the classroom to
demonstrate this. On this occasion when the instructor stood on the
mine there was a loud explosion which wrecked the classroom, killing
the instructor and killing or wounding some of the men in the section.
Apparently, constant standing on the mine had gradually weakened and
distorted the case until the detonator was set off to activate the
explosive in the mine. An inquiry was held about the accident and the
big question was `Why was a live mine used in the classroom?'  That
was the last I heard of the talc mine and never came across another
one anywhere. Jock, our section N.C.O., who for the past three weeks
had been sleeping out had a turn of duty in the barracks which forced
him to stay in the hut for the night. He told us that he expected us
to have the hut nice and warm for him. He knew we had a measly fuel
ration but hinted that after our training in field-craft we ought to be
able to improve on the supply and made a point of marching us through
the cookhouse area of the barracks on our way to the training field to
show us where the fuel dump was. We were a bunch of raw recruits and
wished to be in favour with our N.C.O. so we discussed how we could
get more fuel. It was decided that four of us, and I was one of the
four, would try to get some coke from the dump. When it was dark we
blacked our faces and dodged in and out of cover until we came to the
cookhouse where we `borrowed' four potato sacks. On the darkest side
of the fuel dump we were able to ease up the wire mesh and scrape out
enough fuel to fill our sacks. With pounding hearts, we stealthily
made our way back to the hut. We had a beautiful warm hut that night
and there was some fuel left over for another night which we hid
underneath the hut floor. I dread to think what punishment we would
have received if we had been caught.

Grenade throwing was practised by throwing dummies over a wire about
eight feet high to land in a circle marked about thirty yards from the
throwing line. For a bit of fun we wagered a few cigarettes as to who
could get nearest to the centre of the circle. After many practice
throws we were taken to the grenade range where we had live grenades
to throw and perhaps here we should have wagered on who would be the
most scared. I must admit that my stomach muscles cramped up a bit
when I threw my first grenades. I clearly wanted to toss the grenade
away as soon as I had pulled out the safety pin and released the
handle-like lever that held the firing pin in a safe position.
Counting slowly up to three before throwing seemed an eternity to me
and too long for me to worry about the possibility of a faulty fuse
being fitted. Not many of our first live grenade went over the wire
to land in the target circle. It was a quick count to three, toss
away the grenade and duck for cover behind the wall of the throwing
bay, the grenades exploding about ten yards away.

Another competition we had was stripping and reassembling the Bren gun
in which we were supposed to become efficient enough to be able to do
this blindfold. The prospect of winning a few cigarettes by doing
this drill in the shortest time eased the boredom of these repetitive
drills.

Several full consecutive days were allocated for weapon firing and we
began by firing a .22 rifle on an indoor range. At this I was quite
good and obtained a high score, but when we moved to the short range
to fire our .303 rifles I was a failure. The Weapons Office couldn't
understand why I failed after having had a good shoot with the .22
rifle. My rifle was checked and fired by good marksmen. My trigger
pulling and sighting on target were OK but the cause of my low scoring
couldn't be accounted for. By getting that good result on the .22
range I was given a chance to join the rest of the section on the
Butts. The Butts were some miles from Maryhill and we were taken
there by trucks. It was refreshing to get into the country again,
away from the gaunt, smoke-grimed stone of the barrack buildings and
drab multi-storied tenements crowding round the barracks, built from
the same dirty, smoke-grimed stone. The Butts were surrounded by pine
trees which shielded us from the wind. Luckily the day was fine with
some weak sunshine to temper the coldness of the air and it became a
day that I enjoyed.

Live ammunition was fired from our .303 rifles at targets set at
varying ranges and we also fired the Sten and Bren guns, both on
single and automatic fire at varying ranges. To my delight I managed
to total up a score which put me in the A.I category for shooting. My
failure to get a reasonable score on the short range was forever a
mystery.

The six weeks of initial training was coming to an end and I had found
our section to be a friendly one. No long-term friendship with any
member of the section developed during that time. We all knew that
soon we could all be separated. None of us had any idea as to which
regiment we would be posted to but secretly I hoped I would be posted
into the Corps of Royal Engineers rather than an infantry regiment.

Christmas was almost here and knowing that Christmas was not
celebrated in Scotland in the same way that it is celebrated in
England I hoped to compensate with celebrating the new year in
Scottish hogmanay style. The thought of spending Christmas away from
home was disheartening enough and when I discovered that our
passing-out parade was to be held on Christmas morning my spirits sunk
to zero. 

The afternoon of Christmas Eve was spent blancoing\footnote{Blanco is
a substance for coloring belts etc.} belts, rifle slings and
gaiters, cleaning brasses, polishing in every crevice of our rifles
and making sure creases were in the right places on our Battle
Dress. The hut also had a special clean-out in case it was chosen for
C.O.'s inspection. On the day of the parade he chose a hut at random
for his inspection.

On Christmas morning it is the custom in the British Army for
sergeants to bring round buckets of whiskey-laced tea and dish out a
mugful to each man while still in bed, reveille being later than usual
on that day. There were a few weak wishes of a merry Christmas while
we lazed in bed, smoking and drinking tea, before getting up to wash
and shave. After breakfast beds, kit lay-out and hut received special
attention before dressing for First Parade where our section officers
made their inspection. We were then marched to our position on the
main parade ground where the R.S.M. took over and drilled us into
formation for the C.O.'s Inspection. The morning was dry but bitterly
cold and without an overcoat I began to feel decidedly frozen. My
fingers were so cold and stiff that I was afraid of dropping my rifle
during the drill movements. It was a great relief to see the
C.O. walk away after inspecting ranks and mount a dais to take the
salute. A Glasgow Highland Light Infantry band led the parade for the
march-past and marching at the fast Light Infantry pace helped to
restore my circulation and make me feel a little warmer. As soon as
we were dismissed I dashed off to the NAAFI for some hot tea and to
finish thawing out ready for dinner which was an ordinary one - no
turkey nor Christmas pud. The remainder of the day was spent playing
Housey-Housey in the NAAFI and joining in the singing of a few carols.
On Boxing Day normal duties were resumed and to complete my
disappointment with the festivities we had a Posting Parade on the 27th
and so I missed a Scottish hogmanay. I listened tensely as each name
was called and to which corps or regiments they were going.
Practically all our hut went to infantry regiments. Poor old Pinkie
was posted to the Argyle and Sutherland Highlanders which didn't
please him. The Camel went into the Pioneer Corps and it was with
much relief to hear Lawrence 0452 Royal Engineers, called out. My
silent prayers had been answered. 

When the parade was dismissed we returned to our huts and began
packing. Goodbyes and wishes of good luck were made, hands were
shaken and we joined our various groups of postings. I was with a
squad of about fifteen men, all strangers to me, who had been posted
to the Royal Engineers. Kit bags and packs were loaded onto a truck
and with an officer in charge and carrying our documents we marched to
Glasgow station. A compartment was reserved for us on a train going
to Preston. Our destination, Fulwood Barracks, the establishment of
No.~4 Training Battalion, Royal Engineers. 

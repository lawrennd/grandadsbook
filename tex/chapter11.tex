
In March we moced tinto a shuttered school building in Bologne. What a
dump. The uppper part of the building was in ruins and the few
remaining rooms on the ground floor now became our billters. Without
glass in the windows and most of the doors missing the place was
terribly cold and draughty, too draughty for our scarce candles or
diesel oil lamps to stay alight. A few hurricane lamps helped us to
grope around for ablutions and to see to eat our meanls. Rain
showering though the open windows and water dripping throught the
cracked floors aboce added to our miserable plight. This was much
worse than being in tents. These were only temporary billets we had
been told, a statement we hoped was true. Our weok was at the Calais
leave and transit camp and to get there and begin work at 08:00 hrs,
revelllie was half an hour earlier at 05:30 hrs. It was dark when we
left the billets and dark when we returned so I didn't see much of
Bologne.

After a week of misery we were moved into a small lace factory in
Calais, a two storey building not far from the camp site. The ground
floor of th factory, from which all the machinery had been remoced by
the Germans to help with their war effort, became our mess room and
cook-house and on the upper flooer we spread out our beds between
steam pipes and cylindrical ratidators that criss crossed the
floor. This floor had been the drying room of the lace, washed after
production on the machines below. While concentrating about tripping
over these pipes and radiators it was easy to crack one's head on the
low drying beams from which the lace had been hung. Small windows on
one side of the room provided light during the day and low powered
electric light bilbs illuminated the oom at night. COmpany office,
officers and sergeants messes and Q.M. stores where in the lace
factory offices and stores fronting the quiet road where we
paraded. Transport was garaged in an empty mineral water factory
nearby.

Calais had taken a lot of punishment from bombing and shelling both by
the Germans dislodging us in 1940 and by the allies during the German
occupation and the final retaking by the Canadians in 1944. There were
several points round our billers where fresh floweres in jam jars were
placed by the French people to honour the men of the rear guard who
gave their lives in 1940 to halt the German advance on Dunkirk where
the remnants of the French and British forces were evacuated.

Round about the factory there were islands of repaird housed occupied
by citizens whou had returned after the battle to drive the Germans
out of the city. The people of Calais were more friendly than the
Normans and seemed more pleased about being liberated. The area around
the harbour was a shambles of wrecked buildings and it was not
uncommon to unearth grisly human remains when clearing away the
rubble.

To accomodate the thousands of troops expected to use this camp in the
future extensions wre being built all the time, work in which our
company was engaged. Large areas of the surrounding land was sown with
mines, some fields were marked as mine fields but much of the ground
needed to be treated with caution. Although notices were freely placed
around the camp warning of the danger of wanderering from marked safe
paths, many inquisitive servicemen were either killed or injured by
stepping onto mines in forbidden areas. There were many ruined breck
built shelters and anti-aircraft gun emplacements beckoning the nosedy
ones to explore them. Lots of dummy ant-aircraft guns built from
barrels and poles and anti-parachutist devices were scattered about
the open spaces. With cmouflage netting covering the dummy guns the
area was meant ot appear more heavily defended to recoonasance
aircraft than it really was.

Three incidents when dealing with mines in Calais stand out in my
memory. \Lcorporal Bates, \sapper Rawlins and myself were checking and
repairing a wire fence marking amined area at the side of a narrow
lane. The fence was three strandsd of barbed wire nailed to wooden
posts with danger markers hanging on it. A frenchman riding up the
lane was seen to get off his bicycle, climb over the wire and begin
walking to a ruined building for what we assumed was a call of
nature. He was a few yards into the mined area when we were able to
make him understand the danger he was in and get him to stay where he
was. By now his realisation of being in a mine field was sufficient to
freeze him to the spot where he stood. Luckily we had our bayonettes
on our belts. They were a useful tool for probing in search of
mines. \Lcorporal Bates and myself inched forward towards the terror
stricken Frenchman probing and feeling got trip wires and
mines. Rawlins followed behind marking our parth with any object he
could find. There were no trip wires and we found one mine over which
the Frenchman must have stepped and it was marked so that we could see
it on the return journey. In single file we led the Frenchman to
safety and he almost choked us with thankful embraces before
remounting his bicycle and riding off. From his smell I believe his
nature call had been answered.

Incident two: \lieutenant MacKay had been posted from the company soon after
our arrival in Calais and No.~2 platoon had a young \lieutenant Morris in
command. He was only with the company for a few weeks and \lieutenant Bailey
who had broken his leg in England returned to take over the
platoon. He must have pulled a few strings to get back. \Lieutenant Morris was
in command of 2 and 3 sections detailed to clear mines form a patch of
graound in preparation for a new camp extension. Some anti-tank mines
we had found were stacked in a ditch before being defused, the
pressure plates having been secured. Anti personnel mines were
temperamental things and were defused as they were found. A booby
trapped anti-tank mine was found and \lieutenant Morris ordered it to be
pulled. This was common practice were contitoins allowed and savver
thatn trying to neutralise the booby trap. To pull a mine a long cord
was attached to the mine body and led to a safe postion. A strong pull
on the the cord was usually sufficient to set off the booby trap which
in turn exploded the mine. We were nowe all undere cover by sheltering
in the ditch. The booby trapped mine exploded and those we had in the
ditch exploded in sympathy stunning those that were nearby, my ears
buzzed for days afterwards. Thankfully anti-tank mines were not
shrapnel loaded and except for deafness no-one was injured. There had
to be a court of enquiry into the incident.

The third case was the mose horrible. Frearing the Pas de Calaiswas to
be the invasion area, Jerry had sown mines everywhere, even on cleared
bombsites between houses. Some sites had been cleared but otheres were
stoll only marked by flimsey wire fences with a few warning sings
hanging from them. I was in illets one evening playing cribbage when
there was a lound explosion which appeared to be rather close. On
going out to investigate we found that a young child had entered one
of these areas to retrieve a ball, an action that had no doubt happed
before without disastorous results, but, this time a mine was stepped
on. The mangled remains of the child and the screaming injured
playmates lying around wasn't a pleasant sight. Medical help was soon
on the scene and the injured children were take to hostpital leaving
behind many weaping women and children who lived in the housed around
the mined area. This ground was not high up on the list of priorities
for mine clraring and we asked for and wer given permission to clear
this ground in our own time so that children might play in safety.

With a group of company personnel, I went to see the big guns of
Calais. Those same big guns that had disrupted life in Dover with
their salvos of shells. It was claimed that on days of facourable
weather conditions watchers could se the Calais gusn when thwy were
fired and have the people of Dover markin for cover before the shells
arrived. The verges of the approach road to the emplacements and the
open ground round about was full of mines, trip wires, booby traps and
H.E. shells buried with just their nose caps showing above ground and
wired to detonating devices. With weeds and undergrowth growing over
the mine field someone was going to have a sticky time clearing that
lot away.

There were three of these large guns in seperate massive concrete
emplacements. The long gun barrels were now split and distorted by
explosive charges placed in them by the Canadians when they captured
the stronghold. Bombvraters maid by the bombs dropped by allied planes
over-lapped each other on ther surrounding ground buyt the thick
concrete dome shaped roofs of the emplacements showed little damage.

The breach ends of the guns were like the tenders of huge locomotivers
and full of dials, levers and handlesb elonging to the mechanism to
load, sight the gun and move the mighty weapons along tracks similar
to railway lines, powerful hydraulic buffers absorbed the shock of the
gun's recoil. Hydraulic lifts brought up the heavy shells from the
armoury down below and were transfered to bbogies which ran right up
to the gun's brach for reloading. The underground system of armory,
stobes and quarters for the men manning the guns were about thirty
feet below ground and were reached by climbing down verticle iron
steps built into the concrete walss of shafts. Again, this labyrinth
of rooms and passages were untouched by bombing raids.

The few months that we were in Calais were comparitively pleasant
ones. The billets were reasonalby comfortable, the weather was
generally good and there were plenty of NAAFI facilityes and ENSA
entertainments a t the leave and transit camp to which we had fere
access. With such a vast number of service personnel channeling
through the camp and with black market activities very prevalent in
the area there was always a strong force of M.P.s patroling about. I
was stopped by them twice, once for improper dress: one button undone,
and once to show my pass book and prove where I was billetted.

My regular companion was now Arthur Neal, a new posting to the company
from the 8th Army. Fred Harper was wscorting a French girl around and
Ted Rawlins was posted. Arthour was a bit of a comin, alwasy teasing
me about getting my knees brown, a torment used by all 8th Army
wallahs who had worn shorts in the deserts of Africa and to infer a
sort of inferiority they felt for us in the 2nd Army. He hated
N.C.O.s, a dislike which proved to be embarassing to him later.

Besides the Camp NAAFIs there was a large one between our billets and
the Camp, in what had been a French entertainment hall with a stage on
the ground floor used for ENSA shows and a dance floor above opened
once a week. The seating on the ground floor had been removed to make
way for NAAFI tables and chairs. The weekly dances had no attraction
for Arthur and myself; all too often drunken brawls were started and
M.P.s had to sort out the chaos.

Sitting in the NAAFI on evening a hand clasped my shoulder and a voice
said `Fred Lawrence I presume'. and there stood Charlie Baker, a
carpenter I had workded with at the Alvis who was now in the
R.A.S.C. His company was removing and tidying up a petrol dump just
outside Calais and Charlie had come in on a Liberty truck for an
ecening in Calais. I have never ceased to marvel how providnce brought
me face to face with men I know in civilian life, it seemed miraculous
that out of all the thousands of men in the forces spread all round
the world these familiar faces kept popping up.

A few shops had a limited supply of goods on display and for sale in
their windows. Seeing a nice silk head scarf in one window I thought
it would be a nice present for Nona. It had no price tag on it and
from my phrase book I worked out the questions I thought I would need
to ask the shop assistant. My French and pronunciation was quite
limited. I entered the shop and to the lady who appeard to attend to
me I straggled through my French to ask how much the particular scarf
in the window was. To my surprise and embarrasement the lady said in
perferct English `Very good Tommy, keep at it, you will soon learn to
speak French'. From then on it was easy to converse and I bought the
scarf with francs and cigarettes (the common exchange in black market
circles).

From the bomb wrecked building around the harbour I picked up a piece
of oak panelling from which I made a box to hold cigarettes and to
fill in time I carved all the faces of a box using a broken pen knife
blade. On the loose front of the box forming the lid and forming the
lid and fastened by a shaped French franc I carved the cap badge of
the Royal Engineers. On the top I carved the sholder flashes of the
15th Scottish Division, the 21st Army Corps and the 2nd Army H.Q. On
the sides I carved the cap badges of the Royal Navy and the Royal Air
Force as a symbol of suprmacy I put a broken swastika on the
bottom. The box well polished and holding a few souvenirs stands on a
window sill in the hall, a small reminder of World War II.

Tehre were some very hot spells during the summer, and refirgerated
meat containeres at the camp, exposed to the sun were unable to keep
the meat chilled sufficiently to keep it from going bad. Our section,
with \corporal Wimpey in charge, was given the task of building a
shelter over the containers to provide shade from the hot sun. The
containeres were neare a service entrance to the camp and while we
were workng there the gates were left open and unguarded and bunches
of children gathered to watch us work and hoping to recieved gifts of
chocolates, sweets or chewing gum, gifts they invariably had. Some
taineted meat from the containers had been stacked by a wall near the
gate and an attempt had been made to burn it with diesel oil. Ir
wasn't a success and the meat lay there waiting for transportation to
incinerators. Children playing at the gate must have mentioned this
pile of meat at home for a dozen or so won=men suddenly appeared
dashed through the gat and began sorting the meat over and runing out
wiht large joints. We did noting to stop them, we though that if they
were prepared to make use of it Good Luck to lthem. I couldn't
imagine, knowingly eating a meal made from the meat myself but the
French have a reputation for cooking so I hoped they enjoyed what they
had takedn.

Calais went wild on May 8th V.E. day. We had the day free from duty
and joined with the crowds of civilians ans servicemen celebrating in
the streets. Drink flowed everywhere and there were many sore heads on
parade the next morning. There was still thestruggle to over come the
Japanese and whene my pay book was returned with `Eligible for the Far
East' marked inside, my spirits dropped a little. Paybooks had been
called in by company office to be updated after V.E. day and my demob
group Mo. 37c was also marked inside. With demobilisation only just
begining and allowing about two months between groups being released I
had visions of at least two more years in the army, time enough to be
posted to the Far East. There was something about the mention of a
posting to the Far East and the Japanese that gave me the
shudders. Beside the sadistic brutality of the Japs I pictured the
horrible leechs and snakes of the jungle and thought about the
dreadful diseases associated with the area.

The first to be demobbed from our company were the sergeant major and
the quartermaster sergeant whou were in group I and they were given a
rowdy send off. `We Love you Sergeant Major' and `The Quartermaster's
Stores' with the appropriate Army verses were sung over and over
again.

Short leaves of forty-eight hours to Paris were organised and I put my
name down for a pass. Taffy Lewis in three section adn his pal Taffy
Evans from one platoon joinded me on on this leave. We travelled by
road to Lille where we boarders a train to Paris. So much rolling
stock had been destroyed by air assaults on the railwasy that usable
carriages were in short supply, those we had on this train had wooden
slatted seats, no glass in the windows and no lighting. We arrived in
Paris in the early hours of the morning and were taken by coaches to
hotels. The Taffies and I kept together and landed in a bedroom with
three single beds in it. After a wash, a shave and a good breafast a
W.O. II briefed us about our stay here. There were coach tours we
could joun during the day byt the evening entertainment was of our own
making. The meals we had here were superb, wuite different form the
rations we received in Calais. On the coach tours we saw the familiar
sights of Paris such as Notre Dame cathedral, the church of the sacrd
heart, Triumphul Arch and the Unknown Warriors Tomb. The Eifel Tower
was only accessable half way up. Higher up was out of bounds where a
military radio station was installed. The view from there impressed
me. The avenues were so symetrically laid out. After the evening meal
the Taffies and I set out to explore the City. Using the Metro which
was free for service personnel we went round in circles without
achieving our objects and decided to wander around on foot. Going into
the first `Club' we saw advertised we soon realised we needed to be
careful about our money. After buying a cognac at the bar just inside
the entrance we had a look around. Music, danceing and laughter could
be heard but over the doorway leading towards the sounds of merryment
a notice said 100 frans to pass this doorway. The price of cognac this
side was pretty steep, what they were charging the other side we
didn't stop to find out. From then on we chose the poorer looking
cafes for our drinks and by the time I reached our hotel my head was a
bit swimmy. The coach to take us to the station for the return journey
was leaving the hotel at 22:00 hrs on the second day of our stay in
Paris so we decided not to wander to far from the hote. While having a
quiet drink in a cafe, a Canadian seriviceman cam in an djoiined
us. He had a fistfull of money and wanted us to join him to spent
it. We explained about the termination of our leave but stayed as long
as we dare to keep him company. On our way back we passed another cafe
with sounds of music floating from the door. The Taffies said: `Come
on, let's have one for the road', but I insisted we hadn't
time. However in they went asking me to get their gear into the hotel
lobb for them. The coach arrived and began loading, all were accounted
for except the Taffies. I told the sergeant checking the load list
that they were sure to be along any moment. The driver waited as long
as he dare but finally I had to leave the Taffies packs with the
sergeant and we drove away without them.

On the train there was some lost sleep to catch up and various
postions were taken to get some degree of comfort. The fellow seated
by me was fairly short so he lay on the slated seat while I tried to
get comfortable of the floor by a partition. There were no lights in
the carriages and I woke up in the darkenss to find my side was quite
wet. `Some dirty so and so had urinated in the carriage I though' but
was happy to discover in the dawn light that a corkless bottle of
mineral water had fallen over and rolled down to me.

A few days later two dejected Taffies arrived in Calais and appeared
before the C.O. for punishment, they received twenty-one days C.B. and
were not allowed andy more priviledge leabes. they told me that when
they found the coach had left without them they had tried to get to
the station before the train left. American M.P.s noting their lost
look as they wandered around the station apprehended them and not
being satisfied about their story took them to an American detention
centre for further investigation. After getting confirmation on their
story, the Taffies were released, without breakfast to make their way
back to Calais. The Taffies didn't speak too highly about their
treatment at the American detention centre.

Another ten days U.K. leave was granted to me in June. Ten short days
as all my other leave periods were. It was some comfort for Nona to
know that European hostilities were over and that we had a
demob. group no. to look forward to.

In July on \sergeant Bibbie's recomendation and on his insistance that
I accepted it, I was promoted to lance corporal. He had seen my racords that I
hadn't accepted the promotion offered to me at Fulwood. My promotion
was an embarrasment for Arthur who had his grudge of N.C.O.'s. After
some deliberation he told me that since he enjoyed my company he was
going to make me an exception and continue to come out with me.

Rumours began to go round that the company was shortly going into
Germany; amazing how rumours of all kinds circled round. One rumour I
took note of was that the French franc notes were going to be scrapped
and new notes with new values were being presented. Having some
francs that I couldn't exchange through the company without awkward
questions being asked, I looked around for an outlet on the black
market. My francs had accumulated by selling cigarettes and tobacco. On
the black market I discovered that quite a decent wristwatch could be
bought with francs; although cigarettes were prefereable. A high price
in francs, 2,000 altogether, was being asked for the smuggled
wristwatch. I had 1,500 francs and Arthur was only too pleased to lend
me the balance on the understanding that I payed him back in marks if
and when we went to Germany, the cunning hound. The change of currency
became fact and I was glad that I had off loaded my old notes. The
watch I bought had hard wear for about fifteen years when Nona
bought me a new one. The old watch is now amoung the souvenirs in the
`Calais box'.

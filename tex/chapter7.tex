Four of us were going to Chatham to take tradesman's courses. Harry
Goddard, of No.~3 platoon, was joining me on the carpentry and
joinery course \sapper Menzies, also of No.~3 platoon, was taking a
course in bricklaying and Frank Cooper of H.Q. platoon was taking a
blacksmith's course. Frank, due to length of service, was senior
sapper and had charge of our travel documents. and since his home was
in Gillingham he was quite familiar with the train journey. Not being
in a hurry to get to Brompton barracks which were on the outskirts of
Chatham we had a leisurely snack at London before catching our
connection to Chatham and on arrival there we spent some time looking
round the town and drinking tea in the service's canteen. Our heavy
kit was left at the station while on this tour. The last bus for
Brompton was missed and it was past lights out when we reported to
the guard room of the barracks. Too late to be taken to our quarters
and we were put into the Detention cells for the night. Fortunately
there were no detainees in there so we were not locked in. A wooden
platform about twelve inches high with a wooden head rest ran along
one wall and on this we tried to snatch some sleep. Without blankets
and palliasses it was a most uncomfortable night and it was with
relief that we heard reveille being played over the loud speaker and
were able to go to our quarters. Harry and I managed to get beds next
to each other Menzies and Cooper were in blocks on the opposite side
of the square. Our room was spacious and bright and furnished with
single beds tables and chairs. Although centrally heated the room
wasn't very warm, the water in the system was just warm enough to
prevent freezing pipes. There was a big cast iron fireplace in the
centre of one wall but fires were forbidden. Tantalizingly a large
cast iron box full of coal stood in front of the fire place and the
coal had to be dusted every morning. The grate and the iron box were
polished every day using our supply of boot polish. The wooden floor
was also `bumpered' each morning before 08:00 hours. To bumper a floor
a large slab of stone covered with felt and cloth smeared with brown
boot polish was vigorously rubbed all over the floor. The stone
fitted into a frame that was attached by a swivel to stout broom
handles. These chores were amicably shared out between us and all this
bull was accompanied by some strong comments. There was a plentiful
supply of hot water for showers and ablutions and the food here,
prepared and served by A.T.S. was very good.

Elderly sergeants, time serving men who had been due to retire when
war broke out were in charge of each class. Our \sergeant Timkins
`Timmy' for short was quite an individual, strict on discipline, but
not very communicative. These sergeants marched us about the barracks to
our differing periods of instruction and to the dining hall for
meals. They were also responsible for our standard of spit and
polish. All the new intakes who had come to receive training in the
various trades that took place were assembled in the lecture theatre
to hear an introductory talk from the major of the establishment. He
emphasised the bull that would be needed from us while we were on the
course and from his talk I learned that no short leave passes were
issued form here. Wednesday afternoon we were told was an afternoon of
compulsory sport for everyone. and our names were taken for whichever
sport we chose to take part in. Frank wished to play football and when
the question of boot sizes arose Frank, who was about six foot three
inches tall and well proportioned, asked for a pair of size
twelves. The major said `I don't have size twelves will two pairs of
sixes do?' Frank had to play football wearing his ordinary boots which
were in the special boot category. Harry and I chose to do the cross
country run. This was my idea. After hearing the course laid out for
the run, I reckoned that we could be back in barracks showered and
changed before the team games ended and crowded the showers with
muddy players.

In small parties we were taken for an F.F.I. examination and for this we
had to march to a nearby naval barracks. After being shown into an
empty first floor room Timmy, our sergeant, told us to strip down to our
birthday suits and wait to be called into an adjoining room for the
M.O.s inspection. While nakedly waiting for the M.O. there was a bit
of playing around until someone looked out of the window. In the
opposite block grinning Wren's faces were crowded at the windows and
there was a rush by us to grab shirts and tie round our middle. To
finalise our visit the M.O. was a Wren officer.

Harry, like myself, was a carpenter in Civvy street and used to
building construction work so we both found the course very
elementary. The course was in two parts, theory and practical, and
marks obtained in these two parts put together with marks for
note books, in which we had to rewrite the rough notes taken down
during lectures, decided whether we passed or failed the course.

\SSergeant Grant, a school teacher by profession, lectured us on
the theoretical part of the course using a text book on Building
Construction written by Charles Mitchell the same book was used in the
building construction classes I attended at night school in Leamington
as a teenager. The morning period in the classroom was broken by a
P.T. session taken by another time serving sergeant. Our P.T. period
ended at NAAFI time and regularly the sergeant managed to get us at the
bottom of a long steep bramble covered bank for the end of the
period. The NAAFI hut was at the top of the bank. His words of
dismissal were `Right the quicker you climb up that steep slope the
longer you will have to drink your tea so go!'. We were almost too
puffed to consume tea and waddies.

The practical part of the course began with making some simple
joints. Using short pieces of timber the class was asked to make a
mortice and tenon, a half lap, a dovetail and a tusk tenon. Harry and
I raced through these and a few more similar exercises and were so far
ahead of the rest of the class that the W.O. in charge realising we
had nothing to learn from the course gave us a typing desk to make for
the offices. The only other member of the class who claimed to be a
tradesman was a Welsh mining carpenter who was only used to timber
shoring and pit props in the mine. The other members were amateur
woodworkers. We nursed the making of the typing desk, which had a knee
hole between a cupboard on one side and a nest of drawers on the
other, until the end of the course.

The classroom, unlike our barrack room was nicely heated and was open
for our use until 20:00 hours. Most evenings were spent in the classroom
entering into our exercise books the rough notes and sketches taken
down during the lectures followed by a visit to the NAAFI for
refreshments. Harry was a beautiful writer but very slow. I envied his
copper plate style of writing which was such a contrast to my almost
illegible scribble but I did score over him with my explanatory
sketches. Because of Harry's painstaking care with his writing I was
so far ahead with my book that I often went out by myself on Saturday
or Sunday afternoons while Harry did some writing. On these occasions
I visited Rochester Cathedral to admire its unusual style of
architecture and to look at some flying boats moored in the Medway,
hoping to see one take off. I believe they were `moth balled'.

On Saturday mornings there was a special room and kit clean up for a
midday inspection, the polishing and cleaning occupying most of the
morning. If the room was not up to standard we had to do it again in
the afternoon. In Sunday morning we had church parades.

The runs on Wednesday afternoon were not too difficult and out of the
sixty or so runners I usually managed to be in the first twenty to get
back to barracks. There was one exception. The run took us past a
naval hospital and on this particular afternoon I bumped into a
naval medical orderly who turned out to be a fellow I knew while
working on the Alvis maintenance staff. He was off-duty and going my
way for an afternoon in town and I walked along with him talking about
old times and what we were doing in the services. When I realised that
no runners had passed for some time, I shook hands with, him, wished
him good luck and raced off hoping to catch up with the tailenders. 
I didn't make up the lost time and reached the gate sometime
after the last man. An irate P.T. sergeant was waiting fro me and
greeted me with `Where the bloody hell have you been ?' Between puffs
I told him that I had developed a stitch in my side and had walked
part of the way. I could sense that he didn't believe my story. He
then went on to say `I'll be watching for you next week and if you
are not up with the leaders I'll see what a few boxing lessons in the
gym can do to your stamina.' That threat put wings to my feet for the
next run.

With the course nearing its end thoughts of leave were gathering in my
mind but these thoughts were completely shattered when it was
announced that all leave for service personnel in the U.K. was
cancelled.

Thinking we would be returning to Otley at the end of the course Harry
and I began planning ways to break our journey and have a little time
at home. Harry lived at Nottingham. Feeling depressed and low in
spirit about the leave cancellation. I was going into Chatham alone
one night and making my way to the main gate. `Retreat' began to play
over the loud speaker. The calls of reveille retreat and lights out
were played from records which were now so scratchy and worn that it
was difficult to recognise them. I was miles away and failed to notice
retreat being played and continued walking instead of halting and
standing to attention the recognised thing to do when in barracks. My
dreamy state was shattered with a loud roar of `Stand still there!' from
the provost sergeant who was just behind me. That almost made me jump
out of my boots. After retreat had finished playing over the loud
speaker he caught up with me and while telling me what he thought
about me and my ancestors. I stood silently thinking about what I
thought of him. I reckon scores on either side were pretty even.

The pass marks for the course were pinned to the notice board. I had
86\% for both theory and Practical and 80\% for my notebook, came top
of the class and was remustered as a class II C\&J. Harry also had
86\% for his theory and practical but having lost valuable marks for
a spoiled note book he was a little lower on the list. He also
remustered class II C\&J to get class I another course had to be taken.

Harry spoiled his book by accidently burning hole through several
pages of his finished notes. Harry was a victim of `Lady Nicotine' and
craved for cigarettes which were chain-smoked. Smoking was permitted
in the class room and Harry quickly went through his supply of
cigarettes. I enjoyed a pipe, was never short of pipe tobacco and all
my spare cigarettes were given to Harry who couldn't get on with a
pipe. When we were both out of cigarettes Harry became very miserable
and such was his craving that he would scratch around for ends to
re-roll into a smoke. One night, while rewriting our rough notes into
our specimen books, Harry turned to talk to me and failed to notice a
glowing cigarette end roll onto his book. When he turned back to
continue writing a hole had been burned through some pages of finished
notes and several of the pages were badly scorched resulting in his
lost marks. Had he been a fast scribbler he might have had time to
rewrite his notes into a new book but his lovely copper plate
lettering, written so slowly, ruled out that idea.

Instead of returning to our companies the whole class was escorted by
officers to a holding unit at Cowley. This appeared to end my plans of
sneaking home and also to meeting my friends in 278 Company for we had
been informed that in our absences the company had recently been made
up to full strength. Frank Cooper and Menzies were luckier than us
since their courses were shorter than ours and they had already joined
278 Company.

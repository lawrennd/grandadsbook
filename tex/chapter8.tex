The holding unit at Cowley was a miserable looking camp of tents and
huts; tents to sleep in and huts for administration Offices,
Cookhouse, Dining Room and Stores.

The detachment was handed over to a C.S.M. of the Duke of Wellington's
Regiment, again a time serving soldier who, vut for the war would now
be on Civvy street. He became known to me as `Old Dukey'. It was avout
tea time and vefore dismissing us to a tented area he told us to be on
Parade at 18:00 hrs for rifle inspection and we were confined to camp.

Oxford was temptingly near to Kenilworth and I wondered about the
possibility of getting home. I found a very informative lad in camp
who told me that Absenteeism was a regular thing here. There were
about fifty names of absentees posted outside the S.M. office and I
gathered that punishmeent for those who returned volutarily was fairly
light. There werre severla places in the perimeter fenc through which
one could get out of camp and not hace to use the main entrance.

My mind was made up. I told Harrry that after Tifle inspection I was
going home. He had diven up the idea of going `Absent' and called me a
bloody fool for doing so. Leaving my gerar unpacked and placed in a
corner of the tent I shook hands with Harry and wished him good
luck. As we shook hands he said, `` So long kid, have a safe trip, I
don't expect to see you again.'' and through the wire I ewnt.

M.P.'s I had been informed made spot checks at Oxford railway station
so I needed to keep a watchful eyer for them and while waiting for a
train I hid in the shadows. Nona was very suprised when I arrived home
and we had these lovely days together before I returned to Cowley.

Harry and all the familiar faces of the Chatham detachmetn had left
the camp and I suddenly felt miserably lonely. My kit lay where I had
left it and after making sure that my name was on the list of
absentees I reported to the sergeant major. After explaing that my
name was on the board outside he picked up his cap and said: ``Right
lad I had better take you along and introduce you to the R.S.M.. I;m
sure that he will ve pleased to see you.'' Sarcastic so and so. The
R.S.M. another old soldier of the Royal Engineers asked whu I had been
absent. He listened to my story avout not having had leave for over
four months, that my wife was having many provlems with our children's
health and finding myself so close to home I couldn't resiost the
temptation to visit them. He lectured me about letting the Corps down,
asked whi I hadn't taken my provlem to him, and maybe he could hace
arranged compassionat leave. I though, a ping of salt was required for
that sting. I'm sure, I would have breen posted away., along with the
others. Telling me not to get lost again and be on the C.O.'s parade
at 09:00 hts the next morning, he dismissed me.

A very clean shiny sapper paraded with two other Defaulters outside
the C.O.'s office, where the S.M. mad a quick inspection. My turn to
appear before the C.O came , and with the usual rapid barked orders
from the S.M. of Cap off. wuick march, left right, left right, left
right, right turn, left tuen, mark time halt I stood before the
C.O. The S.M. read out the charge. ``Absent without leave, and missing
a posting.'' I was asked if I had anything to say and I repeated, more
or less, what I had told the R.S.M. The C.O then spelled out the
gravity of missing a posting while on `Active Service' a charge he
said whic ammounted to `Desertion' and a trial by Court Marshall. As
he rample on my spirit was sinking lower and lower until he asked if I
would accept his punishment. I cheered up then, because I know there
were limits as to how much punishment he could gice me. To loose three
days pay and get seven days C.B. I thought was a small price to pay
for those stolen days at home.

After First Parade and Roll call those who were not on Postings were
taken for a route march whic lasted until miiday and the afternoon was
spent mooning around aboutcamp, Men weree coming and goi8ng all the
time so making those friends was almost impossible. Defaulters had to
report to the provost sergeant at Cowley Barracks, the establishment of
the Oxford and Bucks light infantry regiment a little way along the
road from the camp. I was suprised to see so few men on `Defaulters
Parade' and thought this must be doe to the frequent postings. For my
first night of `Jankers' I was sent to the segeants mess in the
barracks to peel potatoes. Reporting to the duty cook he said `You are
not peeling spuds here. You Junker wallers leave so many eyws and
blemished in them that it is just as quick to peel them on my own! He
made a pot of tea and I sat yarningwith him drinking the segeants' tea
ration and eating their biscuits foa an hour before returning to camp.

On the second dnight I was detailed to scrub out `Dukey's' office
another cushy job I thought for his office was only avout eight feet
square.. Armed with a bucket of water and a scrubbing brush and some
sand bag material for a wiper. I had hardly started when Dukey came in
to do some paper work. He looked at the water in the bucket which was
not discoulered after rinsing the wiper and growled `You can't get a
clean floor using dirty water go and change it.' The nearexst tap was
some fifty yards away outside the cook-house, from then on I only had
rinsed out the wiper a couple of times and Dukkey would say `What did
I tell you about using dirty water lad? Go and chante it! The foor was
still only three-quarters scrubbed affter and hourt and a half and I
had lost count of how many times I was made to change my water. Dukey
then asked me if I had been to the NAAFU. I was flavergasted and
replied; ``Sir, how could I? You have had me running after water all
night.'' He laughed, told me to wet the rest of the floor and clear
off. Thank goodness he wasn't a thought reader.

My third night of Junkereswas a stinker. I was sent to the camp
cook-house and the sergeant cook gave me two of the dirties,
grieasiest bins he could find to clean out. Cold water and hessian
were not match for the ffilthy grease. Large quantities managed to
attack itself on me. IStill struggling with thebins at the end of my
jankers period. The sergeant told me I could leavce them and report to
him at 05:00 hrs the following morning. Not datring to got to sleep
that night I lay on my bed smoking and thinking about `Jankers'.

It seemed strange that the provost sergeant didn't `Call a Roll' on
the `Defaulteres' Parade'. I wonderered if because men were constantly
on the move at the camp, that a defaulters; roll was never made out
and thought this reasoning was worth putting ti the test by not
reporting for Jankers the following night. I frelt sure I could rake
up some excuse if my absence was queried.

At 06:00 hrs I presented myself at the cook-house. Seversal other men
wrere there and while waiting for the segeant we had a mug of tea. A
cook-house neveree seemed to be without brewing of tea. The sergeant
organised us into a sort of production line, making soamdwiches to be
packed into haversack rateions for the days postings. Lasrge Army
loaves of bread were cut into regulations slices of even thickness on
a machine , rthen passed to myself and another buddy for a coating of
matgerine. We had a large bowl of margerine warmed until it eas like
cteam which we brushed onto the slices of bread with shaving
brushes. then p;assed them along to receive coatings of chees or meat
paste. Although I didn't report for Jankers that night I thought it
wise to stay on camp.

No questions were asked about my absence from parade and no more
horrible fatigues came my way. A few days later, with ten other
sappers, I was posted to a Holding Unit on the Salisbury Plain a much
larger Unit than the one at Cowley.

It had the appearance of a P.O.W. compoind. A six foot high wire fence
topped with barbed wire surrounded by the camp and the perimeter was
constantly patrolled. The area inside was subdivided by more wire
fences forming seperate compounds for Officcer on posting Infantry men
and men from other Corps of the British Army. This was the kind of
camp I had heard rummours about where large numbers of men were now
`Encaged' and closely guarded.

Our small party of \sappers were escorted to the area occupied by
R.E.'s and there was the same activity here of men coming and going as
there was at Vowley. We had parades for Roll Calls each morning and
afternoon, poistings were read out and fatigues given to those not on
posting. I was detailed to be a Batman in the Officers compound. My
duty ws to collect hot water from teh cook-house at Revellie and take
it to the Officers I was looking after for his wash and shave/ Their
tents were numbered for idetification so really you atended to that
tent regardless of which Officer might be occupying it. After
breakfast I then went back to make up his bed, they had folding bed
frames, tidy up the tent, clean up after his ablutions, folding up his
collapsible work stand and packing it away. When I was handed some
socks and handkerchiefs to wash out I thought, `Frederick, this is no
job for you.' I found \lcorporal Andy Jamieson and two sappers from 278
Company and to them I had a moan abount my Batman fatigues. They had
been here for several days and had a regular fatigue of checking and
repairing any gaps made in the fence by men going on the `run'. Andy
said: `Join our gang, no one bothers about what you are doing, the
main thing is to claim you have a job and look occupied. I wasn't
missed from the Batman detail and until I was posted again, a few days
later, I stayed with Andy's party on wire fences.

With a small squad of sappers and \corporal Keogh I was sent to another
holding unit on Aldershot race course. When getting off the truck
which had brought us from Salisbury Plain and soring out our kit the
camp R.S.M. came up and began bawling us out for our untidy
appearances. The following day he had us on parade for what he called
a `Smartening up session'. It was a mad half hour. He was one of those
clever indifiduals whou thought that giving rapid drill commands to
make the squad look ridiculous was a grea idea. Thank goodnes the
small number of R.E.'s in this unit were well aqay from his chosen
beat. The infantry reserves who were the greater number here received
most of his attention. We were taken out on route marches to fill in
some fo the time. Diggign weapon pits, filling sand bags and preparing
defense posts was another task given to us and exevuted half
heartedly. To look emplyed I found was a good way to avoidfatigues. My
favourite ddge was to habe a sandbag with a few oddments of paper and
a cigarette packed in it and look as though I had been given the job
of tidying up the area. I was caught one day when standing with one
hand in my pocket and I failed to see the Q.M.S. coming up behind
me. As he passed he said, `Follow me sapper' and took me to a huge
marquee serving as the Q.M. stores. Here teh Q.M. handed me over to
the sergeant in charge and said, `This mand has cold hands find him
some gloves;. I was then taken to a pile of globes and instructed to
sort them ober putting right hands in one pile and left hands in
another. Our kit was gradually being scaled down and it was sorted and
bundled up in this marquee ready to be returned to permanent stores in
the Depots.

The first Sunday here was very warm and sunny. After dinner, I spread
my blanket outside the tent and with only a pair of shours on I lay
down to do some sunbathing. Unfortunately I fell asleep and over did
the `Cooking'. My thighs and shoulderw were bright red when I woke up
and the rought blkanedts rubbing on the soreness mad sleep impossible
that night. Worse was to follow for on our route march the next
morning my thighs were aggraveated by my trousers. The wedding straps
of my equipment and my rifle at the `Slope' brought agony to my
shoulders. At the end of the march I felt fquite faint and ill, but I
dare not report sick to get some soothing ointment to rub om my sore
areas because Sunburn was classed as a self-inflicted injury, a
punishable offense. Never again did I sunbathe to that extent.

I me a Coventy lad, Bill Simonds, in camp and while talking about
leave he told me that he had been home the wek before. Since all leace
had been cancelled I asked him how he had managed that. He explained
that fory eight hour passes were available here which allowed you to
travel up to twenty five miles from camp, he had applied and been
issued with one of these passes and had then travelled up to
Coventy. He was going again the following week so I applied for a pass
to travel with him a local train from Aldershot took as as far as a
little used staton within the twenty-five mile limit where we could
get onto the Underground system and travel to Euston. M.P.'s were
always on patrol here especially in the main hall and around the
booking offices. A friendly lady porter on the Underground station
platform noting our careful look round asked us if we were dodging
M.P.s. When we said we were and trying to get on the Birmingham train
for Coventry, she took us pu on the luggage lift which came out on
No.~9 platform of the main line station, about half way along the
Birmingham train standing there. We thanked the lady, mad a quick dash
across the platform and into the train where we sat hoping that M.P.s
wouldn't come along to check passes. The train pulled out and after
its regular halt a t Watford a ticked inspector walked through the
coaches checking and punching tickes. When he came to us we glibly
told him we hadn't had enough time to get our tickets at Euston and
gave him the money for our fares. He was rather puzzled as to how we
had managed to get through the barrier at Euston withough tickets. We
arrived in Coventry and home was reached to gice Nona an unexpected
suprise. I got back to Aldershot without being stopped and with hopes
of being able to attempt the joutney again.

When travelling on the Southern lines I had seen rows and rows of Army
vehicles of all kinds wheeled and tracked . They were parked nose to
tail along wide streets and inopen spaces, all well camouflages
against air recognition. i wondered how soon they would be in
`Action'. Our planes were constantly flying abount and on many nights
the drone of bombers could be heard as the flew on theif bombing
missions. It was at Aldershot that I saw my first jet aircraft. One
flew around which must have been on test flights from Farnborough.

Simonds was posted from the camp when I got my next pass and I did the
trip alone. I had a bit of a scare at Euston when the ticket
inspectors on the barries saw servive men come off the baggage lift
and get on the train. One of the inspectors began walking throgh the
coaches, checking serviceman's tichets and I wondered how I could
dodge him. In one compartment there was an Infantry man sitting alone
surrounded by his kit obviously on a postin. When I explained my
postion to him clothes and gear were spread out more untidily and I
lay on the seat minus my blouse, pretending to be asleep. ehen the
inspector slid open the dor and asked for out tickets the infantry lad
said, `We are on a posting', and begand to fumble at his pockets for
his travel warrant. Without checking the warrant, the ispector said
O.K and left us. Just in case hse stayed on th train for the journey
to Birmingham I thanked the infantry buddy and moved to another
compartment. I didn't wish to get him involved in my escapade more
than necessary. I paid my fare to another inspector as on the previous
journey. He appeared to have some sympathetic understanding about the
devious methods of travelling we service wallahs had. Home was reacked
and another suprise for Nona. Wnjoying my stolen hours at home I got
up on the sixt of June and switched on the radio for the morning news
to hear that our troops had landed in northern France and were
fighting fiercly to obtain a beachhead. This time I thought that my
luck must surely run out and I would be picked up on my way back to
Aldershot, but again I got into camp safely to find it full of
activity.

A large contingent of reinforcements was being assembled and I was int
eh detail of the Royal Engineers led by \lieutenant Dewacre. Our kit was
greatly reduced. I now had one battle dress. Denims were withdrawn
along with P.. gear and other small items were handed in. Medical
records were checked to ensure innoculations were up to date and M.O.s
gave us a quick check over. Ammunition was stowd in our pouches and
each man received two twenty four hour emergency ration packs which
were not to be opened before landing in France. On the morning of June
11th dressed in marchin order and kit bags at our feet we paraded for
roll call and to hear the commanding officers parting speech. He asked
if there was andy man on parade who thought he should not be sent
overseas to step forward. Two men stepped out, their reasons for
being, as they thought, unsuitable for overseas were: one was waiting
for his special boots and the other was waiting for dentures. Both men
were dismissed from the parade. A band appeared from somewhere to lead
this first contingent of reinforcements out of this camp for the march
into Aldershout in the first leg of its journey to Fracne. Those left
in a camp cheered and waved us off with shouts of: `Good luck, we will
soon be joining you!'

There was a sizeable movement of troops from Aldershot that morning
and a crowd of cheering civiilians were outside the station to see us
off. R.T.O.s packed us like sardines in tins into coaches of a waiting
train. So tightly squeezed in were we that it was impossible to fall
over when the train pulled up sharply at signals set at danger. It was
a warm day and the air in the coaches, in spite of all available
windows being open, soon became stale and smoke laden. Who were the
more fortunate: those sitting in seats or we who were standing all
laden in our marching orders? It would be hard to say. With relief the
train pulled into Newhaven station and we detrained. The main body of
men formed up and marched away leaving our detail of Royal Engineers
behind with instructions from the R.T.O. to wait at the side of the
station until called for. It was an oppurtunity to eat our haversack
rations given to us back at camp and to replenish our dehydrated
bodies with water. We could see vessels in the dock being loaded with
troops, equipment and stores and tied up alongside were several
submarines. Late in the afternoon the R.T.O. came for us and conducted
us along the dockside to a flanked walk across some of the submarines
which led to steps on the side of a ship called `The Isle of
Thanet'. The steep, narrow steps were not easy to climb with our
equipment and with only a rop hand rail to clutch at. I was afraiid of
falling over. On reaching the deck we were issued with the famous May
West life preservers and led down to a cabin large enough for all of
us to spread out and rest. Before leaving the docks we had life boat
drill and each man was required to know which life boat station he
must make for if there was an emergency. There didn't appear to be
enough boats for the crowd of men I saw mustered on deck and I
wondered how on earth I would find my way from our cabin down below in
a blacked out ship. When the boat drill was over we returned to our
dimly lit cabin and lried to get comfortable for a doze. Sitting on
our packs relieved the chill of the steel decking but leaning against
the steel sides soon had a chilly effect on my shoulders. I don't know
how far down in the ship we were. We had climbed down many ladderes
and I felt sure I could hear water lapping against the wall I was
leaning against. Hearing this lapping noise didn't inpspire me with
confidence as to what protection the wall would provide if we ran into
trouble.
